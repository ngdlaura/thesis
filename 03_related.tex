\chapter{Pre-trained Language Models}
\label{chapter:related-pretrained-language-models}


\renewcommand{\leftmark}{\spacedlowsmallcaps{Pre-trained Language Models}}

\ifthenelse{\boolean{skipRelated}}{\endinput}{}

\minitoc

\chapterwithfigures{\nameref*{chapter:related-pretrained-language-models}}
\chapterwithtables{\nameref*{chapter:related-pretrained-language-models}}


Prior to the works of \citet{peters-etal-2018-deep}, \ac{NLP} models were commonly trained \textit{from scratch} in a supervised manner to perform specific tasks, based on (limited) training. As a result, training deep neural networks on such small datasets led to overfitting, making the models sensitive to even slight shifts in the data distribution. 

While word embeddings such as Word2Vec's \citep{mikolov2013efficient} are learned from large corpora, their application in task-specific neural models is restricted to the input layer. Task-specific neural models must be built nearly from scratch, given that the majority of model parameters handling token interactions need to be optimized for the task at hand. This optimization process requires substantial amounts of data to attain a high-performance model.

% The seminal works of \citet{peters-etal-2018-deep, devlin2018bert, radford2018improving} marked a paradigm shift by generalizing the use of \textit{transfer learning} in \ac{NLP}, initiating a new era of \textit{Pre-trained Language Models}, also referred to as \textit{Foundation Models}. Transfer learning avoids building task-specific models from scratch by applying the knowledge acquired from training a model on one task to another task. In the context of pre-trained language models, the model is initially trained in a \textit{self-supervised} way on a large-scale corpus to learn general language representations. 

The seminal works of \citet{peters-etal-2018-deep, devlin2018bert, radford2018improving} marked a paradigm shift by generalizing the use of large unsupervised training datasets in \ac{NLP}, initiating a new era of \textit{Pre-trained Language Models}, also referred to as \textit{Foundation Models}. This shift is driven by the idea that modeling language using large corpora of text is sufficient for training powerful language models capable of performing well across various \ac{NLP} tasks. Rather than building task-specific models from scratch using scarce labeled data, Pre-trained Language Models leverage extensive unsupervised (or \textit{self-supervised}) pre-training to learn general language representations, and apply the knowledge acquired to achieve broad task applicability in \ac{NLP}. The \textit{pre-training} phase allows the model to capture context and a general understanding of syntax and semantics. After pre-training, the model can be \textit{fine-tuned} on specific downstream tasks (\textit{e.g.}, text classification, \ac{NER}, machine translation, and more). Fine-tuning consists in further training the pre-trained model on a smaller, labeled dataset specific to the downstream task. The knowledge acquired during pre-training is leveraged and tailored to the new task, often resulting in improved performance.

% In summary, pre-trained LMs leverage transfer learning by first learning general language representations in an unsupervised manner and then transferring this knowledge to specific tasks through fine-tuning. This approach has proven effective in achieving state-of-the-art results across a variety of NLP applications.

As one of the early endeavors to adopt the "pre-training then fine-tuning" paradigm in \ac{NLP}, \ac{ELMo} generates contextual representations that can be fine-tuned by adding task-specific layers. However, \ac{ELMo}'s effectiveness is somewhat constrained compared to models like \ac{BERT}, in part due to its reliance on \acp{RNN}. The breakthrough in Pre-trained Language Models came with the introduction of the Transformer architecture in the work of \citet{vaswani2017attention}. Transformer-based Pre-trained Language Models \citep{devlin2018bert, radford2018improving} have demonstrated that fine-tuning improves the state of the art across a wide range of language tasks. This suggests that task-specific architectures are no longer a necessity. Furthermore, Transformer-based Pre-trained Language Models perform better with an increased volume of data, model size, and training compute, demonstrating superior scaling behavior \citep{kaplan2020scaling}. Hence, the Transformer has become the go-to component in the modern \ac{NLP} stack, largely replacing other architectures such as \acp{RNN}. Transformer-based Pre-trained Language Models are typically categorized into three main types: \textit{bidirectional} models utilizing only an encoder, \textit{encoder-decoder} models leveraging the entire Transformer architecture, and \textit{generative} models relying on the decoder alone.

In this chapter, we first describe the Transformer architecture, with a focus on its core component — the attention mechanism. Subsequently, we discuss how Transformers can be leveraged to build effective language models, ranging from bidirectional models capable of producing robust, general-purpose word representations to generative models able to create coherent and contextual relevant text, thereby laying the groundwork for \textit{Large Language Models}.

% The advent of pre-trained language models, starting with ELMo and later models like BERT and GPT, marked a paradigm shift by enabling models to learn contextualized representations of words and phrases from large amounts of unlabeled data. 

% Another interesting property of transformer architectures is their structured memory, which allows handling long-term dependencies in text, a problematic issue for recurrent networks like LSTMs. In addition, transformers support parallel processing since they are not sequential models like recurrent networks. 


\section{Transformers}

A Transformer \citep{vaswani2017attention} is an encoder-decoder architecture that establishes a conditional distribution of target vectors $(\bm{y}_1, \ldots, \bm{y}_m)$ given a source sequence $(\bm{x}_1, \ldots, \bm{x}_n)$. The encoder encodes the source sequence $(\bm{x}_1, \ldots, \bm{x}_n)$ into a contextualized sequence of hidden states $(\overline{\bm{x}}_1, \ldots, \overline{\bm{x}}_n)$. The decoder then uses these hidden states to condition the probability distribution of the target vector sequence $(\bm{y}_1, \ldots, \bm{y}_m)$. By Bayes' rule, this distribution can be factorized to a product of conditional probability distribution of the target vector $\bm{y}_i$ given the encoded hidden states $(\overline{\bm{x}}_1, \ldots, \overline{\bm{x}}_n)$ and all previous target vectors $(\bm{y}_0, \ldots, \bm{y}_{i-1})$. Formally:

\begin{equation}
    p_{\theta} \bigl( \bm{y}_1, \ldots, \bm{y}_m \mid \overline{\bm{x}}_1, \ldots, \overline{\bm{x}}_n \bigr) = \prod_{i=1}^{m} p_{\theta}\bigl(\bm{y}_i |\bm{y}_0, \ldots, \bm{y}_{i-1}; \overline{\bm{x}}_1, \ldots, \overline{\bm{x}}_n\bigr).
\end{equation}

\noindent The source sequence $(\bm{x}_1, \ldots, \bm{x}_n)$ and target sequence $(\bm{y}_1, \ldots, \bm{y}_m)$ are embedded and fed to the encoder and the decoder, respectively.

The Transformer tackles two primary limitations of \acp{RNN}. \acp{RNN} suffer from the vanishing/exploding gradient problem, which hinders their ability to capture long-range dependencies. In addition, the sequential processing of input in \acp{RNN} hampers efficient parallelization \citep{vaswani2017attention}. To overcome these limitations, the Transformer removes recurrence altogether by leveraging the \textit{self-attention} mechanism to capture global dependencies between input and output. 

We elaborate on the concept of \textit{attention} and explore a few attention mechanisms, including \textit{self-attention}. We then study the application of self-attention within the Transformer architecture. Finally, we provide details on the processing of sequential information in Transformers.

\subsection{Attention Mechanism} 

The concept of attention can be best explained through an analogy with human biological systems. In various problems involving language, speech, or vision, specific parts of the input are more important than others. For instance, in tasks like machine translation and summarization, only certain words in the input sequence may hold relevance for predicting the next word. An attention mechanism integrates this idea of relevance by allowing the model to dynamically \textit{pay attention} to specific portions of the input that contribute to effectively performing the task at hand. In the following, we formalize the concept of "paying attention", specifically in the context of Transformers.

\subsubsection{Bahdanau's Attention Mechanism} 

The earliest use of attention was proposed by \citet{bahdanau2014neural} for a \textit{sequence-to-sequence} modeling task. A sequence-to-sequence task involves mapping a sequence of $n$ input vectors to a sequence of $m$ target vectors, where $m$ is unknown apriori. A sequence-to-sequence model \citep{sutskever2014sequence} consists of an \textit{encoder-decoder} architecture, where the encoder encodes an input sequence $(\bm{x}_1, \ldots, \bm{x}_n)$ into a sequence of fixed-size vectors $(\bm{h}_1, \ldots, \bm{h}_n)$. The decoder is then fed the fixed-size vector $\bm{h}_n$ and generates an output sequence $(\bm{y}_1, \ldots, \bm{y}_m)$.

In a traditional encoder-decoder architecture (usually based on \acp{RNN}), the encoder must compress all the necessary input information for the task at hand into a single fixed-size vector $\bm{h}_n$ that is fed to the decoder. However, encoding a variable-length input into a fixed-size vector squashes the information of the input sequence, irrespective of its length, causing the performance to deteriorate rapidly as the input sequence length increases \citep{cho2014properties}. In addition, in sequence-to-sequence tasks, each output token is expected to be more influenced by specific parts of the input sequence. However, the decoder lacks any mechanism to selectively focus on relevant input tokens.

To alleviate these challenges, \citet{bahdanau2014neural} introduce the attention mechanism, a principle that allows the decoder to access the entire encoded input sequence $(\bm{h}_1, \ldots, \bm{h}_n)$ and dynamically \textit{attend to} information deemed relevant to generate the next output token. The key idea behind attention is to introduce attention weights $\bm{\alpha}$ over the input sequence, prioritizing positions with relevant information for the generation of the next output token. These attention weights determine the context vector $c$, which is then fed to the decoder. At each decoding position $j$, the context vector $\bm{c}_i$ is defined as a weighted sum of all encoder hidden states $\{\bm{h}_j\}_{j=1, \ldots, n}$ and their corresponding attention weights $\{\alpha_{ij}\}_{i=1, \ldots, n}$: 

\begin{equation}
    \bm{c}_i = A(\bm{s}_{i-1}, \bm{h}) = \sum_{j=1}^n \alpha_{ij} \bm{h}_j.
\end{equation}

\noindent The introduction of the context vector serves as a biased input summary of the encoding states and is re-computed for each output token. This addition enhances the quality of the output by achieving better alignment.

The attention weights $\alpha$ determine the relevance between each encoder hidden state and each decoder hidden state. Each attention weight $\alpha_{ij}$ is computed as a function of the encoder hidden state $\bm{h}_j$ and the decoder hidden state $\bm{s}_{i-1}$, defined as follows:

\begin{equation}
    \alpha_{ij} = \textrm{softmax}\bigl(a(\bm{s}_{i-1}, \bm{h}_j)\bigr),
\end{equation}

% \noindent where $a$ is an alignment function implemented as a feed-forward network, and $p$ is a distribution function. The alignment score $a(\bm{s}_{j-1}, \bm{h}_i)$ defines how relevant $\bm{h}_i$ is for $\bm{s}_{j-1}$.

\noindent where $a$ is an alignment function implemented as a feed-forward network. The alignment score $a(\bm{s}_{i-1}, \bm{h}_j)$ indicates how well the element $\bm{h}_j$ of the input sequence aligns with the current output $\bm{s}_{i-1}$.

\subsubsection{Generalized Attention} 

The attention mechanism can be reformulated into a general form where the alignment does not inherently rely on the hidden \ac{RNN} states. The \textit{generalized attention} model \citep{chaudhari2021attentive} extends the attention mechanism of \citet{bahdanau2014neural} by allowing for more flexibility and adaptability in capturing dependencies between different parts of the input and output sequences. While the original attention mechanism focuses on aligning parts of the input sequence with the current position in the output sequence, the generalized attention model introduces parameters and mechanisms to customize and control the attention process. In the generalized attention model, attention weights are not solely determined by the relevance between the current decoder hidden state and the encoder hidden states. Instead, the model introduces learnable parameters and scoring functions that can be adjusted to capture different types of relationships. This allows the attention mechanism to consider various aspects, such as semantic similarity, positional information, or other task-specific factors.

A generalized attention model is characterized by three components: the queries $\bm{Q}$, the keys $\bm{K}$, and the values $\bm{V}$, all derived from the input sequence using learnable weight matrices $\bm{W}^{(q)}$, $\bm{W}^{(k)}$, and $\bm{W}^{(v)}$. Queries indicate which information is requested from the input sequence, keys are vectors associated with each element in the input sequence and determine which elements are relevant for the queries, while values contain the information to be propagated. Given a set of key-value pairs $(\bm{K}, \bm{V})$ and a query $\bm{q}_i$, generalized attention is defined as follows:

% \begin{equation}
%     A(\bm{q}, \bm{K}, \bm{V}) = \sum_i p(a(\bm{q}, \bm{k}_i)) \cdot \bm{v}_i
% \end{equation}

% \begin{equation}
%     A(\bm{q}_i, \bm{K}, \bm{V}) = \sum_j \alpha_{ij} \cdot \bm{v}_j = \sum_j p(a(\bm{q}_i, \bm{k}_j)) \cdot \bm{v}_j.
% \end{equation}

\begin{equation}
    A(\bm{q}_i, \bm{K}, \bm{V}) = \sum_j \alpha_{ij} \cdot \bm{v}_j,
\end{equation}

\noindent where:

\begin{equation}
    \alpha_{ij} = p(a(\bm{q}_i, \bm{k}_j)).
\end{equation}

\noindent The alignment function $a$ dictates the combination of keys and queries, such as through dot product or cosine similarity. The query vector is compared with each key to compute alignment (or similarity) scores. The distribution function $p$ takes these scores as input and ensures that the attention weights lie between 0 and 1 and are normalized to sum to 1, achieved through mechanisms like logistic sigmoid or softmax function. The final vector $A(\bm{q}_i, \bm{K}, \bm{V})$ is formulated as a weighted sum of all value vectors $\{\bm{v}_j\}$ and their corresponding attention weights $\{\alpha_{ij}\}$.

The attention mechanism of \citet{bahdanau2014neural} can be seen as a special case of generalized attention where the queries and values are analogous to the encoded outputs, \textit{i.e.}, $\bm{K} = \bm{V} = \{\bm{h}_j\}_{j=1, \ldots, n}$, and the query is analogous to the previous decoder output, \textit{i.e.}, $\bm{q} = \bm{s}_{i-1}$. 
% Then, $e = a(\bm{K}, \bm{q})$ and $\alpha = p(e)$.


\subsubsection{Self-Attention} 

One common form of the generalized attention model is \textit{self-attention}, or scaled dot-product attention, introduced by \citet{vaswani2017attention} in the Transformer architecture. The alignment function $a$ is defined by a scaled dot product, while the distribution function $p$ corresponds to the softmax. The scaled dot product between query and key is passed through a softmax function to obtain the final attention weights. Self-attention is defined as follows:

\begin{equation}
    A(\bm{q}_i, \bm{K}, \bm{V}) = \sum_j \textrm{softmax}\left(\frac{\bm{q}_i^{\top} \bm{k}_j}{\sqrt{d}_k}\right) \cdot \bm{v}_j,
\label{equation:self-attention}
\end{equation}

% \begin{equation}
%     \alpha_{ij} = \textrm{softmax}\left(\frac{\bm{q}_i^{\top} \bm{k}_j}{\sqrt{d}}\right) \cdot \bm{v}_j,
% \label{equation:self-attention}
% \end{equation}

\noindent where $d$ is the dimensionality of the key vectors. The alignment score $\dfrac{\bm{q}_i^{\top} \bm{k}_j}{\sqrt{d_k}}$ indicate how to weigh the value $\bm{v}_j$ based on the query vector $\bm{q}_i$. The more similar a key vector $\bm{k}_j$ is to $\bm{q}_i$, the more important is the corresponding value vector $\bm{v}_j$ for the output vector. 

% attention scores are computed by taking the dot product of the query (decoder hidden state) and key (encoder hidden state) vectors

\subsubsection{Muti-Head Self-Attention} 

Rather than computing attention in a single step, \citet{vaswani2017attention} propose to decompose the self-attention operation in multiple heads to capture different aspects of the relationships between tokens. The dimensionality $d$ is divided into $h$ fixed-size segments, with one segment per attention head. Each of the $h$ heads uses three weight matrices (for queries, keys, and values) to project the segment into different subspaces. Self-attention is then computed across each of the $h$ segments in parallel, following Equation~\ref{equation:self-attention}. The outputs of each head are then concatenated to form the complete attention output, and projected back into the original $d$-dimensional representation space.

% Given a query $\bm{q}_i \in \mathbb{R}^{d_q}$, a set of keys $\bm{K} \in \mathbb{R}^{n \times d_k}$, and a set of values $\bm{V} \in \mathbb{R}^{n \times d_v}$, each attention head $r \in [1, \ldots, h]$ is expressed as:

% \begin{equation}
%     \alpha^{l}_{ij} = \textrm{softmax}\left(\frac{(\bm{q}_i \bm{W}_l^{(q)})^{\top} (\bm{k}_j \bm{W}_l^{(k)})}{\sqrt{d}}\right) \cdot \bm{v}_j \bm{W}_l^{(v)},
% \end{equation}

% \begin{equation}
%     A^{(r)}(\bm{q}_i, \bm{K}, \bm{V}) = \sum_j \textrm{softmax}\left(\frac{(\bm{q}_i \bm{W}_r^{(q)})^{\top} (\bm{k}_j \bm{W}_r^{(k)})}{\sqrt{d_k}}\right) \cdot \bm{v}_j \bm{W}_r^{(v)},
% \label{equation:self-attention}
% \end{equation}

% \noindent where $\bm{W}_l^{(q)} \in \mathbb{R}^{d \times d_q}$, $\bm{W}_l^{(k)} \in \mathbb{R}^{d \times d_k}$, and $\bm{W}_l^{(v)} \in \mathbb{R}^{d \times d_v}$ are learnable weight matrices that transform the queries, keys, and values into sub-queries, sub-keys, and sub-values. 

Given a projection matrix $\bm{W}_o \in \mathbb{R}^{d_v \times d}$, multi-head attention is defined as follows:

% \begin{equation}
% \alpha^{(\textrm{MHA})}_{ij} =
% \begin{bmatrix}
%     \alpha^{1}_{ij} \\
%     \alpha^{2}_{ij} \\
%     \ldots \\
%     \alpha^{h}_{ij}
% \end{bmatrix}
% \bm{W}_o.
% \end{equation}

\begin{equation}
    A^{\textrm{(MHA)}}\bigl(\bm{q}_i, \bm{K}, \bm{V}\bigr) = 
    \begin{pmatrix}
        A(\bm{q}_i \bm{W}^{(q)}_1, \bm{K}\bm{W}^{(k)}_1, \bm{V}\bm{W}^{(v)}_1) \\
        A(\bm{q}_i \bm{W}^{(q)}_1, \bm{K}\bm{W}^{(k)}_2, \bm{V}\bm{W}^{(v)}_2) \\
        \ldots \\
        A(\bm{q}_i \bm{W}^{(q)}_h, \bm{K}\bm{W}^{(k)}_h, \bm{V}\bm{W}^{(v)}_h)
    \end{pmatrix}
    \bm{W}_o \\
\end{equation}
    

% The dimension of each head is a subspace of the model's representation space, \textit{i.e.}, $d_k = d_v = \frac{d}{h}$. 
\noindent In addition, multi-head self-attention enables parallelized computation of attention across different representation subspaces.

 
\subsection{Self-Attention in Transformers}

In a Transformer, both encoder and decoder are composed by stacking a series of Transformer layers on top of each other. Each Transformer layer is characterized by a \textit{multi-head self-attention} module and two position-wise feed-forward networks. To help the model train faster and more accurately, a residual connection \citep{he2016deep} is added to all sublayers, followed by layer normalization.

Given two sequences $\bm{Z} = (\bm{z}_1, \ldots, \bm{z}_n)$ and $\bm{Z'} = (\bm{z'}_1, \ldots, \bm{z'}_m)$, we use the following notation for multi-head self-attention:

\begin{equation}
    \textrm{MHA}\bigl(\bm{Z}; \bm{Z'}\bigr) = 
    {\begin{pmatrix}
    A^{\textrm{(MHA)}}\bigl(\bm{z}_1 \bm{W}^{(q)}, \bm{Z'}\bm{W}^{(k)}, \bm{Z}'\bm{W}^{(v)}\bigr)\\ 
    A^{\textrm{(MHA)}}\bigl(\bm{z}_2 \bm{W}^{(q)}, \bm{Z'}\bm{W}^{(k)}, \bm{Z}'\bm{W}^{(v)}\bigr)\\ 
    \ldots \\
    A^{\textrm{(MHA)}}\bigl(\bm{z}_n \bm{W}^{(q)}, \bm{Z'}\bm{W}^{(k)}, \bm{Z}'\bm{W}^{(v)}\bigr) \\
    \end{pmatrix}}^{\top}
\end{equation}

\noindent where $\bm{W}^{(q)} \in \mathbb{R}^{d \times d_q}$, $\bm{W}^{(k)} \in \mathbb{R}^{d \times d_k}$, and $\bm{W}^{(v)} \in \mathbb{R}^{d \times d_v}$ are layer-specific weight matrices for queries, keys, and values, respectively. In general, the dimensions $d_q$, $d_k$, $d_v$ are set $\dfrac{d}{h}$.

% \paragraph{Transformer Architecture} 

% A Transformer \citep{vaswani2017attention} is an encoder-decoder architecture that establishes a conditional distribution of target vectors $(\bm{y}_1, \ldots, \bm{y}_m)$ given a source sequence $(\bm{x}_1, \ldots, \bm{x}_n)$. The encoder encodes the source sequence $(\bm{x}_1, \ldots, \bm{x}_n)$ into a contextualized sequence of hidden states $(\overline{\bm{x}}_1, \ldots, \overline{\bm{x}}_n)$. The decoder then uses these hidden states to condition the probability distribution of the target vector sequence $(\bm{y}_1, \ldots, \bm{y}_m)$. By Bayes' rule, this distribution can be factorized to a product of conditional probability distribution of the target vector $\bm{y}_i$ given the encoded hidden states $(\overline{\bm{x}}_1, \ldots, \overline{\bm{x}}_n)$ and all previous target vectors $(\bm{y}_0, \ldots, \bm{y}_{i-1})$. Formally:

% \begin{equation}
%     p_{\theta} \bigl( \bm{y}_1, \ldots, \bm{y}_m \mid \overline{\bm{x}}_1, \ldots, \overline{\bm{x}}_n \bigr) = \prod_{i=1}^{m} p_{\theta}\bigl(\bm{y}_i |\bm{y}_0, \ldots, \bm{y}_{i-1}; \overline{\bm{x}}_1, \ldots, \overline{\bm{x}}_n\bigr).
% \end{equation}

% The source sequence $(\bm{x}_1, \ldots, \bm{x}_n)$ and target sequence $(\bm{y}_1, \ldots, \bm{y}_m)$ are embedded and fed to the encoder and the decoder, respectively. 

% In a Transformer, both encoder and decoder are composed by stacking a series of Transformer layers on top of each other. Each Transformer layer is characterized by a multi-head self-attention module and two position-wise feed-forward networks. The input of each encoder layer corresponds to the previous layer's output. To help the model train faster and more accurately, a residual connection \citep{he2016deep} is added to all sublayers, followed by layer normalization. In the decoder, self-attention is \textit{unidirectional} to prevent tokens from attending to future tokens. Furthermore, an additional sublayer, the \textit{cross-attention} module, is inserted between the self-attention module and the feed-forward networks. Cross-attention takes as inputs both the encoder's outputs and the outputs of the previous decoder layer. Finally, the outputs of the final decoder layer are fed to a feed-forward network to obtain, for each target position, a probability distribution over the whole vocabulary.

% Using the full set of attention scores $A(\bm{Q}^{(l)}, \bm{K}^{(l)}, \bm{V}^{(l)})$, token representations $(\bm{x}^{(l+1)}_1, \ldots, \bm{x}^{(l+1)}_n)$ are computed by building the corresponding weighted sum over every other token, \textit{i.e.},

% \begin{equation}
%     \bm{x}^{(l+1)}_i = \bm{x}^{(l)}_i + \sum_j \textrm{softmax} \left(\dfrac{\bm{q}^{{(l)_i}^\top} \bm{k}^{(l)}_j}{\sqrt{d_k}}\right) \cdot \bm{v}^{(l)}_j.
% \end{equation}

% Authors demonstrated that Transformer architecture achieved significant parallel processing, shorter training time and higher accuracy for Machine Translation without any recurrent component

\subsubsection{Self-Attention in the Encoder} 

In the encoder, self-attention is used to map the input sequence $(\bm{x}_1, \ldots, \bm{x}_n)$ to a sequence of context-dependent vectors $(\overline{\bm{x}}_1, \ldots, \overline{\bm{x}}_n)$. Each attention layer builds the queries, keys and values from the outputs of the previous encoder layer, and uses \textit{bidirectional} self-attention to put each input token in relation with all input tokens in the sequence. Given $(\bm{x}^{(l)}_1, \ldots, \bm{x}^{(l)}_n)$ the input sequence to the $l$-th encoder layer, the outputs $(\bm{x}^{(l+1)}_1, \ldots, \bm{x}^{(l+1)}_n)$ constructed using bidirectional self-attention can be expressed as:

\begin{equation}
    \bm{x}^{(l+1)}_1, \ldots, \bm{x}^{(l+1)}_n = \bigl(\bm{x}^{(l)}_1, \ldots, \bm{x}^{(l)}_n\bigr) + \textrm{SA}\bigl(\bm{x}^{(l)}_1, \ldots, \bm{x}^{(l)}_n\bigr), \\
    % \bm{x}^{(l+1)}_i = \bm{x}^{(l)}_i + \textrm{MultiHeadAttention}\left(\bm{q}^{(l)}_i, \bm{K}^{(l)}, \bm{V}^{(l)}\right), \qquad \forall \quad 1 \leq i \leq n,
    % \bm{x}^{(l+1)}_i = \bm{x}^{(l)}_i + \sum_{j=1}^{n} \textrm{softmax} \left(\dfrac{\bm{q}_i^{{(l)}^\top} \bm{k}^{(l)}_j}{\sqrt{d_k}}\right) \cdot \bm{v}^{(l)}_j, \qquad \forall \quad 1 \leq i \leq n.
\end{equation}

\noindent where:

\begin{equation}
    \textrm{SA}\bigl(\bm{x}^{(l)}_1, \ldots, \bm{x}^{(l)}_n\bigr) = \textrm{MHA}\bigl(\bm{x}^{(l)}_1, \ldots, \bm{x}^{(l)}_n ; \bm{x}^{(l)}_1, \ldots, \bm{x}^{(l)}_n\bigr).
\end{equation}

% where $\bm{q}^{(l)}_i$, $\bm{K}^{(l)}$, and $\bm{V}^{(l)}_i$ are the query vector, key, and value matrices obtained by projecting $\bm{x}^{(l)}_i$ using three weight matrices $\bm{W}^{(l)}_Q \in \mathbb{R}^{n \times d_q}$, $\bm{W}^{(l)}_K \in \mathbb{R}^{n \times d_k}$ and $\bm{W}^{(l)}_V \in \mathbb{R}^{n \times d_v}$ (with $d_q = d_k = d$).

Each encoder layer builds a contextualized representation of its input sequence, and the following layer further refines this context-dependent representation. Compared to \acp{RNN}, bidirectional self-attention reduces the amount of computation steps that information needs to flow from one point to another. Therefore, information loss is reduced, making long-range dependencies more easily learnable. 

\subsubsection{Self-Attention in the Decoder} 

The decoder models the distribution of a target sequence $(\bm{y}_1, \ldots, \bm{y}_m)$ conditioned on the input sequence $(\bm{x}_1, \ldots, \bm{x}_n)$. Each decoder layer contains three sublayers: \textit{decoder self-attention}, \textit{cross-attention}, and a module made of two position-wise feed-forward networks. The final decoder layer is followed by a feed-forward network which produces a probability distribution over the whole vocabulary. 

The decoder self-attention layer conditions each decoder output vector on all previous decoder input vectors. As opposed to the encoder, self-attention in the decoder is masked (\textit{unidirectional}) to ensure that each vector attends only to the previous positions. The output vectors $\bm{y}'^{(l)}_0, \ldots, \bm{y}'^{(l)}_i$ generated by unidirectional self-attention are defined as follows:

% Given $\bm{y}^{(l)}_i$ a target vector fed to the $l$-th decoder layer, the output vector $\bm{y}^{\prime(l)}_i$ generated by unidirectional self-attention is defined as follows:

% \begin{equation}
%     \bm{y}^{\prime(l)}_i = \bm{y}^{(l)}_i + \textrm{MultiHeadAttention}\left(\bm{q}^{(l)}_i, \bm{K}^{(l)}_{0:i}, \bm{V}^{(l)}_{0:i}\right), \qquad \forall \quad 1 \leq i \leq m,
% \end{equation}

\begin{equation}
        \bm{y}'^{(l)}_1, \ldots, \bm{y}'^{(l)}_{i} = \bigl(\bm{y}^{(l)}_1, \ldots, \bm{y}^{(l)}_{i}\bigr) + \textrm{SA}^{(unidirectional)}\bigl(\bm{y}^{(l)}_1, \ldots, \bm{y}^{(l)}_{i}\bigr), 
        % \bm{x}^{(l+1)}_i = \bm{x}^{(l)}_i + \textrm{MultiHeadAttention}\left(\bm{q}^{(l)}_i, \bm{K}^{(l)}, \bm{V}^{(l)}\right), \qquad \forall \quad 1 \leq i \leq n,
        % \bm{x}^{(l+1)}_i = \bm{x}^{(l)}_i + \sum_{j=1}^{n} \textrm{softmax} \left(\dfrac{\bm{q}_i^{{(l)}^\top} \bm{k}^{(l)}_j}{\sqrt{d_k}}\right) \cdot \bm{v}^{(l)}_j, \qquad \forall \quad 1 \leq i \leq n.
\end{equation}

\noindent where:

\begin{equation}
    \textrm{SA}^{(unidirectional)}\bigl(\bm{y}^{(l)}_1, \ldots, \bm{y}^{(l)}_i\bigr) =
    {\begin{pmatrix}
        A^{\textrm{(MHA)}}\bigl(\bm{y}^{(l)}_1 \bm{W}^{(q)}, \bm{Y}_{0:1}\bm{W}^{(k)}, \bm{Y}_{0:1}\bm{W}^{(v)}\bigr)\\ 
        A^{\textrm{(MHA)}}\bigl(\bm{y}^{(l)}_2 \bm{W}^{(q)}, \bm{Y}_{0:2}\bm{W}^{(k)}, \bm{Y}_{0:2}\bm{W}^{(v)}\bigr)\\ 
        \ldots \\
        A^{\textrm{(MHA)}}\bigl(\bm{y}^{(l)}_i \bm{W}^{(q)}, \bm{Y}_{0:i}\bm{W}^{(k)}, \bm{Y}_{0:i}\bm{W}^{(v)}\bigr). \\
    \end{pmatrix}}^{\top}
\end{equation}


% \noindent where $\bm{q}^{(l)}_i$, $\bm{K}^{(l)}_{0:i}$, and $\bm{V}^{(l)}_{0:i}$ are projections of $(\bm{y}^{(l)}_0, \ldots, \bm{y}^{(l)}_i)$.

To condition the probability distribution of the next target vector on the encoder's input, \textit{cross-attention} is applied to put ($\bm{y}^{(l)}_1, \ldots, \bm{y}^{(l)}_i)$ into relation with all contextualized input vectors $(\overline{\bm{x}}_1, \ldots, \overline{\bm{x}}_n)$. The output vectors $(\bm{y}^{(l+1)}_1, \ldots, \bm{y}^{(l+1)}_i)$ built using cross-attention are expressed as:

\begin{equation}
    \bm{y}^{(l+1)}_1, \ldots, \bm{y}^{(l+1)}_{i} = \bigl(\bm{y}'^{(l)}_1, \ldots, \bm{y}'^{(l)}_{i}\bigr) + \textrm{MHA}\bigl(\bm{y}'^{(l)}_1, \ldots, \bm{y}'^{(l)}_{i}; \overline{\bm{x}}_1, \ldots, \overline{\bm{x}}_n\bigr),
\end{equation}

% \noindent where:

% \begin{equation}
%     \textrm{CA}\bigl(\bm{y}'^{(l)}_1, \ldots, \bm{y}'^{(l)}_{i}; \overline{\bm{x}}_1, \ldots, \overline{\bm{x}}_n\bigr) = \textrm{MHA}\bigl(\bm{y}'^{(l)}_1, \ldots, \bm{y}'^{(l)}_{i} ; \overline{\bm{x}}_1, \ldots, \overline{\bm{x}}_n\bigr).
% \end{equation}

% \noindent While $\bm{q}^{\prime(l)}_i$ is computed from the output $\bm{y}^{\prime(l)}_i$ of the unidirectional self-attention module, $\overline{\bm{K}}^{(l)}$ and $\overline{\bm{V}}^{(l)}$ are built from the contextualized input sequence $(\overline{\bm{x}}_1, \ldots, \overline{\bm{x}}_n)$. 
\noindent Cross-attention ensures that, the more similar a decoder input representation is to an encoder input representation, the more does the input representation influence the decoder output representation.

In contrast to the encoder, the output vector $\bm{y}^{(l+1)}_i$ represents the next target vector $\bm{y}_{i+1}$, and not $\bm{y}_{i}$ itself.


\subsection{Sequential Information in Transformers}

The position and order of words deeply impact the semantics of a sentence. By processing sequences token by token in a sequential manner, \acp{RNN} inherently integrate the order of the sequence. Unlike \acp{RNN}, Transformers simultaneously process each token in the sequence, making them entirely invariant to sequence ordering. Consequently, there is a need to explicitly incorporate the order of tokens into the Transformer.

\subsubsection{Positional Encodings} 

% There are many reasons why assigning a single number (\textit{e.g.}, the index value) to each time step is not used to represent a token's position in Transformer models. For long sequences, the indices can grow large in magnitude. If the index value is normalized to lie between 0 and 1, it can create problems for variable length sequences, as they would be normalized differently. 

A satisfactory positional encoding method must be deterministic, produce a unique encoding at each time step, generalize to longer sequences, and ensure that distance between any two elements are consistent across sequences with different lengths. Instead of integrating this encoding into the model itself, the dominant approach for preserving information about the sequence order is to modify the initial token representation with information about the token's position in the sequence. These inputs are called positional encodings (or embeddings) and can either be learned or fixed a priori. % In other words, we enhance the model’s input to inject the order of words.


\paragraph{Absolute Position Encodings}

Absolute position encodings encode the absolute position of a token within a sequence, meaning that each token is assigned a fixed vector based on its position in the sequence. \citet{vaswani2017attention} propose a simple scheme for fixed absolute positional encodings, where each position is mapped to a vector. Given $t$ a position in an input sequence, $d$ the encoding dimension, and $k \in \{1, \ldots, d/2\}$, the function $f: \mathbb{N} \rightarrow \mathbb{R}^d$ produces the positional encoding $\bm{p}_t$ as follows:

\begin{equation}
    p_{t,i} = f(t)_i = 
\begin{cases}
    \sin(\omega_k t), & \text{if } i=2k\\
    \cos(\omega_k t),              & \text{otherwise},
\end{cases}
\end{equation}

\noindent where $\omega_k =\dfrac{1}{10000^{2k/d}}$. This encoding scheme is called \textit{sinusoidal} positional encoding. The positional embedding matrix $\bm{P} \in \mathbb{R}^{n \times d}$, obtained by encoding every position $i \in {1, \ldots, n}$, is added to the input representation matrix $\bm{X} \in \mathbb{R}^{n \times d}$ and fed to the Transformer.

% Given $t \in \{1, \ldots, n\}$ a position in the input sequence and $k \in \{0,1, \cdots, d/2-1\}$ the index of an element in the vector space, the positional encoding is defined as a function of type $f:\mathbb {R} \to \mathbb {R} ^{d}$:

% Transformers use a smart positional encoding scheme, where each position/index is mapped to a vector. Hence, the output of the positional encoding layer is a matrix, where each row of the matrix represents an encoded object of the sequence summed with its positional information.

% \paragraph{Relative Positional Biases} 
Besides capturing absolute positional information, sinusoidal positional encoding also allows the model to learn to attend by relative positions. This is because, for any offset $\delta$, the positional encoding at position $i + \delta$ can be computed by a linear projection of the encoding at position $i$. Formally, any pair of $(p_{i, 2k}, p_{i, 2k+1})$ can be linearly projected to $(p_{i + \delta, 2k}, p_{i + \delta, 2k+1})$ for any offset $\delta$:

\begin{equation}
    \begin{bmatrix}
        \cos(\delta \omega_k)  & \sin(\delta \omega_k) \\
        -\sin(\delta \omega_k) & \cos(\delta \omega_k)
    \end{bmatrix}
    \begin{bmatrix}
        p_{t, 2k}   \\
        p_{t, 2k+1}
    \end{bmatrix}
    = \begin{bmatrix}
        p_{i + \delta, 2k}   \\
        p_{i + \delta, 2k+1}.
    \end{bmatrix}
\end{equation}

An alternative form of absolute positional encoding involves learning position embeddings jointly with the model during training \citep{devlin2018bert}. Instead of using fixed functions, learned positional encodings introduce trainable parameters. Each position in the sequence is associated with a unique vector, and these vectors are treated as parameters that the model can learn during training. Learned positional encodings offer flexibility as the model can adapt to various tasks and datasets. The embeddings are not predetermined by fixed functions, enabling the model to learn patterns related to position.

\paragraph{Relative Position Encodings}

Although absolute positional encodings show satisfactory performance, they still face limitations. First, absolute positional encodings do not generalize well to sequences of lengths not seen at training time \citep{dai2019transformer}. Another limitation of absolute positional encoding lies in its lack of translation invariance. Shifting the same sequence alters the absolute positions, leading to different attention patterns. In the case of learned positional encodings,  there is a restriction on the number of tokens a model can handle. For instance, if a language model can only encode up to 1,024 positions, any sequence longer than 1,024 tokens cannot be processed by the model. Relative positional encoding address these issues by using a different vector for each pair of tokens, based on their relative distance \citep{shaw2018self, huang2018music, ke2020rethinking}. \citet{shaw2018self} are the first to leverage pairwise distances to create positional encodings. To account for pairwise distances when computing alignment scores, the relative position embeddings representing the distance between tokens $\bm{x}_i$ and $\bm{x}_j$ is added to the keys as follows:

\begin{equation}
    \alpha'_{ij} = \mathrm{Softmax}\left(\frac{{\bm{q}_i}^{\top} (\bm{k}_j + \bm{b}^K_{ij})}{\sqrt{d_k}}\right).
\end{equation}

\noindent Pairwise distances are clipped beyond a maximum distance to allow for generalization to sequence lengths not seen during training. The relative position embedding $\bm{b}^K_{ij}$ is expressed as:

\begin{equation}
\begin{aligned}
    b^K_{ij} &= \bm{w}^K_{\text{clip}(j-i, k)} \\
    \text{clip}(x, k) &= \text{max}(-k, \text{min}(k, x)),
\end{aligned}
\end{equation}

\noindent where $\bm{w}^K = (\bm{w}^K_{-k}, \ldots, \bm{w}^K_{k}) \in \mathbb{R}^{2k \times d_a}$ is a learned vector.

\citet{raffel2020exploring} introduce a simplified form of relative position embeddings where the distance between every pair of tokens is mapped to a scalar and added to the corresponding logit used for computing the attention weights. Formally:

\begin{equation}
    \alpha'_{ij} = \mathrm{Softmax}\left(\frac{{(\bm{q}_i}^{\top} \bm{k}_j) + \bm{b}_{ij}}{\sqrt{d_k}}\right).
\end{equation}

\noindent To build the relative position bias $\bm{b}_{ij}$, the distance $\delta_{ij}$ between tokens at positions $i$ and $j$ is first mapped to a bucket index $b \in [0, \ldots, B]$ using a bucketing function $f$, where $B$ is a hyperparameter specifying the number of buckets. The bucket $b$ is then mapped to a scalar using an embedding matrix. Hence, distances belonging to the same bucket share the same embedding. For any pair of positions $(i, j)$, the bucketing function $f$ is expressed as:

\begin{equation}
    f(\delta_{ij}) = 
    \begin{cases}
        g\left(-\delta_{ij}, \frac{B}{2}\right), & \text{if } \delta_{ij} \leq 0 \\
        g\left(\delta_{ij}, \frac{B}{2}\right) + \frac{B}{2}, & \text{otherwise.} \\
        % d, & \text{if } 0 \leq d < \dfrac{K}{2} \\
        % \min \left(\dfrac{K}{2} + \left\lfloor \dfrac{\log 2d - \log K}{\log 2M - \log K} \cdot \dfrac{K}{2}\right\rfloor, K-1 \right), & \text{otherwise.}
    \end{cases}
\end{equation}

\noindent If attention is unidirectional, the bucket index is set to $0$ for positions $(i, j)$ where $j > i$. Given an integer $M > B$ and a distance $d$, the function $g$ is defined as:

\begin{equation}
    g(d, K) = 
    \begin{cases}
        d, & \text{if } 0 \leq d < \dfrac{K}{2} \\
        \min \left(\dfrac{K}{2} + \left\lfloor \dfrac{\log 2d - \log K}{\log 2M - \log K} \cdot \dfrac{K}{2}\right\rfloor, K-1 \right), & \text{otherwise.}
    \end{cases}
\end{equation}

% During attention calculation, relative positional information is added on the fly to keys and values. Equation ~\ref{equation:self-attention} is reformulated as follows: 

% Given a query $\bm{q}_i$ computed from token $\bm{x}_i$, and a set of key-value pairs $(\bm{K}, \bm{V})$ computed from the input sequence $(\bm{x}_1, \ldots, \bm{x}_n)$, attention is reformulated as follows:

% Given a query $\bm{q}_i$ computed from token $\bm{x}_i$ and a key $\bm{k}_j$ calculated from token $\bm{x}_j$, the attention score between tokens $i$ and $j$ is reformulated as follows:

% \begin{equation}
%     \alpha_{ij} = \mathrm{Softmax}\left(\frac{\bm{q}_i (\bm{k}_j + \bm{r}^K_{ij})^{\top}}{\sqrt{d_k}}\right).
% \end{equation}

% \begin{equation}
%     A(\bm{q}_i, \bm{K}, \bm{V}) = \sum_{j=1}^n \alpha'_{ij} \cdot \left(\bm{v}_j + \bm{r}^V_{ij}\right),
% \end{equation}

% \noindent where $\bm{r}^V_{ij}$ is the pairwise distance representation for tokens $\bm{x}_i$ and $\bm{x}_j$ used to propagate edge information to the layer output. To account for pairwise distances when computing alignment scores, the relative distance representation for tokens $\bm{x}_i$ and $\bm{x}_j$ is added to the keys:

% \begin{equation}
%     \alpha'_{ij} = \mathrm{Softmax}\left(\frac{{\bm{q}_i}^{\top} (\bm{k}_j + \bm{r}^K_{ij})}{\sqrt{d_k}}\right).
% \end{equation}

Relative positional encodings offer the advantage of generalizing to sequences of unseen lengths. Theoretically, they encode only the relative pairwise distance between two tokens, allowing adaptability to various sequence lengths.

\subsection{Segment Information in Transformers}

Segment embeddings, as introduced by \citet{devlin2018bert}, inform the model about sentence delineations within a sequence. These embeddings, not part of the original Transformer architecture, encode to which of two sentences (\textit{segments})
a word belongs, aiding the model in distinguishing between two segments within the same input sequence. 
Unlike word embeddings and position encodings, segment embeddings remain identical across all tokens within a segment, and vary only between segments. These embeddings are then added to the input representations.

\section{Bidirectional Models}

\textit{Bidirectional} Pre-trained Language Models refer to models that employ the Transformer encoder. These models are pre-trained on large corpora and learn deep contextualized representations of words and phrases by jointly conditioning on left and right context in all self-attention layers. Bidirectional Transformer-based models are effective for capturing dependencies and contextual information in both directions, making them suitable for various natural language understanding tasks including sentiment analysis, \ac{NER}, text classification, and more. The exploration of bidirectional Transformer-based Pre-trained Language Models began with \ac{BERT}, introduced by \citet{devlin2018bert}, and has led to the development of a myriad of variants.

\subsection{BERT}

\ac{BERT} marked a paradigm shift in the construction of word representations.  
Prior to \ac{BERT}, language models typically processed text in a unidirectional manner. Bidirectional models, such as those obtained using stacked bidirectional \ac{LSTM} layers \citep{peters-etal-2018-deep}, also processed the sequence in a fixed order, capturing information from both directions but not simultaneously. \ac{BERT} extended the concept of bidirectional processing by employing bidirectional self-attention.

% Using bidirectional self-attention, \ac{BERT} extended the concept of bidirectional processing. For each token in the sequence, the model can simultaneously consider both preceding and following tokens, enabling a more comprehensive understanding of context.

In \ac{NLP}, some tasks (\textit{e.g.}, sentiment analysis) take a single sequence as input, while others (\textit{e.g.}, natural language inference) require a pair of sequences. \ac{BERT} can represent both single text and text pairs. In both cases, special tokens are inserted into the tokenized text to cater for different \ac{NLP} tasks. A classification token \texttt{[CLS]} is prepended to the input sequence and is used to represent the whole sequence. In addition, a special separation token \texttt{[SEP]}, indicating the end of the sequence, is added to the end of the sequence. In the case of text pairs, an extra \texttt{[SEP]} token is added between the pair to separate the texts. Additionally, segment embeddings are used and trained to distinguish text pairs. To encode positions, \ac{BERT} departs from the fixed positional encodings used in the original Transformer and employs learnable positional embeddings. To sum up, text is tokenized into subwords using WordPiece \citep{wu2016google}, and special tokens are added accordingly to the aforementioned scenarios. The final input embeddings of \ac{BERT} are the sum of the token embeddings, positional embeddings, and segment embeddings. The input embeddings $(\bm{x}_1, \ldots, \bm{x}_n)$ are passed through a Transformer encoder that generates a sequence of contextualized token representations $(\overline{\bm{x}}_1, \ldots, \overline{\bm{x}}_n)$.

\subsubsection{Pre-training BERT}

\ac{BERT} is trained on a large-scale dataset \citep{zhu2015aligning}, as a language model that operates at both the word-level and the sentence-level. The training involves two unsupervised tasks: \ac{MLM} and \ac{NSP}. 

The \ac{MLM} task consists in randomly masking out tokens and using all remaining tokens to recover the masked-out tokens in a self-supervised fashion. Instead of following the same probability distribution as standard language models that process text in a unidirectional manner, \ac{BERT} uses the following approximation:

\begin{equation}
    P(\bm{w}) \propto \prod_{w \in C} P_{\theta} \left(w \mid \tilde{\bm{w}}\right),
\end{equation}

\noindent where $C$ is a random set of tokens, with 15\% of tokens selected to be in $C$, and $\tilde{\bm{w}}$ is the input sequence $\bm{w}$ corrupted as follows:

\begin{equation}
    \tilde{w} = 
\begin{cases}
    w_t,               & \text{if } w_t \notin C\\
    \text{mask token}       & \text{if } w_t \in C, \text{ with probability 80\%} \\
    \text{random token}       & \text{if } w_t \in C, \text{ with probability 10\%} \\
    w_t       & \text{if } w_t \in C, \text{ with probability 10\%.} \\
\end{cases}
\end{equation}

\noindent Because the mask token is never used during fine-tuning, a discrepancy between pre-training and fine-tuning can occur. The process involves masking a selected token with the mask token 80\% of the time, substituting it with a random token 10\% of the time, and leaving it unchanged 10\% of the time. The cross-entropy loss between the masked tokens and their predictions is minimized during pre-training. The primary benefit of the \ac{MLM} task, in contrast to a causal language model, is that token representations are parameterized by the whole sequence.

While \ac{MLM} effectively captures bidirectional context to represent words, it does not explicitly capture the logical correlation between pairs of texts. To address this, the \ac{NSP} task is introduced. This task involves determining whether two sentences follow each other and helps in modeling the relationship between texts. The training dataset is constructed such that half of the pairs are made of consecutive sequences, while for the other half the second sequence is randomly sampled from the corpus. Given a pair of sequences $(s_1, s_2)$, a binary single-layer feed-forward network classifier is trained to determine whether $s_2$ follows $s_1$ in the corpus. The classifier is fed with the \ac{BERT} representation of the \texttt{[CLS]} token, which encodes both sequences, and outputs the probability that the sequences are successive sentences. 


\subsubsection{Fine-tuning BERT}

% The outcome of this pre-training process is a language model able to comprehend context, semantics, and relationships between words and sentences. 
The knowledge gained during pre-training can then transferred to various downstream tasks through fine-tuning. The contextualized token representations obtained by the pre-trained \ac{BERT} are fed to a feed-forward network built over the last encoder layer. In this layer, predictions are generated for either individual tokens or the entire sequence. While the parameters of the pre-trained encoder are reused and fine-tuned for the task, the additional layer is initialized randomly and trained from scratch. However, it contains significantly fewer parameters. This layer can output predictions for individual tokens or the entire sequence, rendering the model suitable for tasks involving token classification and sequence classification, respectively. \\

The bidirectional capability of \ac{BERT} addressed a crucial limitation in previous models, especially for tasks requiring a deep understanding of context and relationships between words. \ac{BERT} demonstrated remarkable performance across various natural language understanding benchmarks (\textit{e.g.}, sentiment analysis, question answering, text classification), showcasing the potential of bidirectional Transformer-based PLMs.
 
\subsection{BERT Variants}

The success of \ac{BERT} has expanded exploration of bidirectional Transformer-based Pre-trained Language Models, with researchers and practitioners building upon the foundation laid by the model. 

Some extensions of \ac{BERT} have introduced different optimization objectives to further improve the quality of the contextual representations. To enhance the robustness and generalization capability of the model, \ac{RoBERTa} \citep{liu2019roberta} removes the \ac{NSP} objective of \ac{BERT}, uses dynamic masking, and employs a bigger dataset. \ac{ALBERT} \citep{lan2019albert} introduces a variant of \ac{NSP} where negative examples correspond to two consecutive segments with their order reversed. This modification of the \ac{NSP} objective, emphasizing coherence, makes the task more challenging, thereby improving the robustness and generalization of the model. SpanBERT \citep{joshi2020spanbert} modify the masking strategy in \ac{MLM} by masking contiguous random spans, rather than random tokens. Pre-trained using three type of language modeling tasks (unidirectional, bidirectional, and sequence-to-sequence prediction), UniLM \citep{dong2019unified} can be fine-tuned on both natural language understanding and generation tasks. 

Furthermore, several adaptations of \ac{BERT}, pre-trained on specific datasets tailored to particular languages or domains, have been proposed. SciBERT \citep{beltagy2019scibert} has been pre-trained on scientific texts and is specialized for tasks in the scientific research domain. Similarly, BioBERT \citep{lee2020biobert} has been designed for biomedical texts. CamemBERT \citep{martin2019camembert}, on the other hand, is a French version of \ac{BERT}, pre-trained on French texts, and adapted for French language understanding tasks. For usage across different languages, multilingual pre-trained models such as mBERT \citep{devlin2018bert}, XLM \citep{lample2019cross}, and XLM-RoBERTa \citep{conneau2019unsupervised} have been proposed. 

To enable the use of models in resource-constrained scenarios, and given the quadratic complexity of \ac{BERT} with respect to the sequence length in both memory and time, researchers have explored approaches to reduce the number of parameters. \ac{ALBERT} achieves efficiency by sharing parameters across layers and reducing the rank of the embedding matrix. DistilBERT \citep{sanh2019distilbert} simplifies \ac{BERT} by using parameter reduction techniques such as factorized embedding parameterization and cross-layer parameter sharing. TinyBERT \citep{jiao2019tinybert} further simplifies the model for increased efficiency. This research is part of a broader effort to make Transformers more computationally efficient, which will be further discussed in Chapter~\ref{chapter:related-long-range-modeling}.

\section{Generative Models} 

In contrast to bidirectional Pre-trained Language Models that focus on predicting labels for given inputs, the goal of \textit{generative} Pre-trained Language Models is to produce new text of arbitrary length that resembles human language. These models employ the Transformer decoder to generate content in an autoregressive fashion. Given their ability to capture contextual information and generate diverse and contextually fitting text, generative Pre-trained Language Models play a pivotal role in improving the quality of text generation for natural language generation tasks (\textit{e.g.}, machine translation, summarization, text completion). Generative Transformer-based Pre-trained Language Models can either use the entire Transformer architecture (\textit{encoder-decoder} models) or solely the decoder component (\textit{decoder-only} models).

\subsection{Training and Inference Framework}

Training and inference for generative models differ from those of bidirectional models in several aspects.

\subsubsection{Training} 

While bidirectional models are typically trained for specific pre-defined tasks, generative language models are trained to predict the next token in a sequence, given the preceding ones (\textit{autoregressive language modeling}). However, compounding errors might occur from incorrect predictions, leading the decoder to potentially drift too far away from the target. To mitigate this issue and guide the training process, the \textit{teacher forcing} strategy is used: to predict the next token, the decoder is fed with the ground-truth target tokens, rather than its own predictions from the previous step. Furthermore, using ground-truth target tokens as inputs accelerates convergence, as stronger gradient signals are provided during backpropagation.

\subsubsection{Inference} 

Teacher forcing is obviously not applicable at inference time. Instead, the sequential generation process requires the model to be fed its own predictions from the previous step as inputs to generate the next token. Different strategies can be used to determine how the model predicts the next element of the sequence. 

\textit{Greedy search} consists in selecting the token with the highest probability at each step. While simple and computationally efficient, it misses high probabilities that can be found in posterior tokens. 

To reduce this risk, \textit{beam search} extends greedy search by maintaining a fixed number $K$  of sequences with the highest probabilities. At each step, it picks the $K$ best sequences so far based on their joint probabilities. Finally, the sequence with the highest probability is selected as the output sequence.

To ensure that the less probable tokens should not have any chance of being selected, \textit{top-k sampling} \citep{fan2018hierarchical} filters the $k$ most likely next tokens and redistributes the probability mass among those $k$ tokens only. Alternatively, \textit{nucleus sampling} chooses from the smallest possible set of tokens whose cumulative probability exceeds a certain threshold $p$. The probability mass is then redistributed among this set of tokens. Nucleus sampling balances randomness and predictability better than traditional sampling.

\subsection{Encoder-decoder Models}

Encoder-decoder Transformer-based Pre-trained Language Models use the original Transformer architecture, consisting of both an encoder and a decoder. The encoder processes an input sequence, producing contextualized representations of each token. The decoder attends to these representations and generates the output tokens one at a time, considering the context provided by the encoder. This architecture is particularly suited for sequence-to-sequence tasks, where the goal is to transform a source sequence into a corresponding target sequence. The encoder-decoder setup is commonly used in sequence-to-sequence tasks like machine translation and text summarization.

\paragraph{BART} \ac{BART} \citep{lewis2019bart} is pre-trained with a denoising objective, where the model is trained to reconstruct the original sequence from a corrupted version. \ac{BART} extends the \ac{MLM} approach by adding more perturbations: replacing text spans with a single mask token (\textit{text infilling}), permuting sentences, deleting or replacing tokens, and rotating documents. The corrupted sequence is encoded using the bidirectional encoder, and the decoder is trained to reconstruct the original sequence. When fine-tuned, \ac{BART} shows remarkable results for natural language generation tasks such as text summarization, machine translation, question answering. Additionally, the model also works well for natural language understanding tasks, \textit{e.g.}, \ac{NER}, \ac{NLI}, and coreference resolution. Inspired by the success of \ac{BART}, \citet{liu2020multilingual} introduce mBART, a multilingual version of \ac{BART} pre-trained on large-scale monolingual corpora in many languages. mBART can be fine-tuned for any of the language pairs, whether in supervised or unsupervised settings, without necessitating task-specific or language-specific adjustments or initialization methods.

\paragraph{Pegasus} \citep{zhang2020pegasus} is specifically tailored for abstractive text summarization. It is trained using the \ac{MLM} strategy coupled with the \textit{Gap-Sentences Generation} task, a novel pre-training approach intentionally similar to summarization. For each document, a subset comprising 15\% of sentences is selected and masked from the original documents.These masked sentence (\textit{gap sentences}) are masked and combined to form a pseudo-summary, which serves as the training label during pre-training. To refine the pseudo-summary and approach summary-like quality, the top-$m$ sentences are selected based on their importance, determined by the \ac{ROUGE}-1 score between each gap sentence and the rest of the document. The model is then tasked to generate the gap sentences using the remaining sentences.

% The Gap-Sentences Generation task consists in masking whole sentences important to an input sequence and generating them together as one output sequence using the remaining sentences, similar to an extractive summary. 

\paragraph{T5} \ac{T5} \citep{raffel2020exploring} converts all \ac{NLP} tasks into a sequence-to-sequence problem: for any task, the input of the encoder is a task-specific prefix (\textit{e.g.}, \say{Summarize:}) followed by the task's input (\textit{e.g.}, a sequence of tokens from an article), and the decoder predicts the task's output (\textit{e.g.}, a sequence of tokens summarizing the input article). The pre-training includes a mixture of both supervised and unsupervised tasks. Supervised pre-training is conducted on downstream tasks (translation, question answering, and more). Unsupervised pre-training uses corrupted tokens, by randomly removing 15\% of the tokens and replacing them with individual (sentinel) tokens. Given the corrupted sequence encoded by the encoder and the original sequence fed to the decoder, \ac{T5} has to reconstruct the dropped out tokens. Casting all \ac{NLP} tasks into the same sequence-to-sequence problem allows for the use of the same model, loss function, and hyperparameters across a diverse set of tasks. 

\paragraph{Limitations of encoder-decoder models} Tasks such as machine translation and summarization commonly favor encoder-decoder models, as  a holistic understanding of the input sequence is crucial for generating accurate and coherent outputs. However, encoder-decoder models face several challenges compared to decoder-only models. The introduction of an encoder adds more parameters to the model, which can be a limitation in scenarios where model size is a critical consideration. Furthermore, the incorporation of bidirectional self-attention and cross-attention introduces additional computational overhead. The interaction between the encoder and decoder can render training more complex, potentially making the model more susceptible to overfitting. Lastly, the encoder may encode redundant or irrelevant information from the input sequence, and the decoder has to learn to filter and use this information effectively.

\subsection{Decoder-only Models}


The use of decoder-only Transformer architectures in Pre-trained Language Models has seen a recent surge, with several groundbreaking models \citep{radford2018improving, brown2020language, ouyang2022training, touvron2023llama} emerging. Decoder-only Pre-trained Language Models, also referred to as \textit{autoregressive} or \textit{causal} language models, remove the Transformer encoder and cross-attention layers. Each source sequence is concatenated with the corresponding target sequence to form a single input sequence, which is then used to train a language model. This design significantly simplifies the architecture and provides potential advantages in computational efficiency and ease of training. 

Among the most performant generative models of the decade is the series of \ac{GPT} models introduced by \citep{radford2018improving}. \ac{GPT} is the first autoregressive language model that uses a Transformer decoder as its backbone. It learns to predict the next word in a sequence using an autoregressive language modeling. Suppose $\bm{w} = \{w_1, \ldots, w_n\}$ an unsupervised corpus of tokens, $k$ the size of the context window, and $\theta$ the parameters of the decoder. 
\ac{GPT} uses the following approximation:

\begin{equation}
    P(\bm{w}) \propto \prod_{i=1}^n P_{\theta}(w_i \mid w_{i-k}, \ldots, w_{i-1}).
\end{equation}

\noindent The pre-training objective can therefore be expressed as follows:

\begin{equation}
    L_1(\bm{w}) = \sum_{i=1}^n \log P_{\theta}(w_i \mid w_{i-k}, \ldots, w_{i-1}).
\end{equation}

\noindent During fine-tuning, the parameters are adjusted to the supervised downstream task. Given a labeled dataset $\mathcal{C}$, where each instance consists of a sequence of input tokens $\bm{w} = (w_1, \ldots, w_m)$ and its label $y$, the following objective is maximized:

\begin{equation}
    L_2(\mathcal{C}) = \sum_{(\bm{w}, y) \in \mathcal{C}} \log P(y \mid w_1, \ldots, w_m) + \lambda L_1(\mathcal{C}),
\end{equation}

\noindent where $\lambda$ is the weight given to the auxiliary language modeling objective.

\noindent \ac{GPT} surpassed state-of-the-art \acp{NLP} models that were trained in a supervised fashion with task-specific architectures. In addition, it improved zero-shot performance in various \ac{NLP} tasks such as question answering, schema resolution, and sentiment analysis. \ac{GPT} established the core architecture for the \ac{GPT}-series models and laid down the fundamental principle to model natural language text, \textit{i.e.}, predicting the next word.

To learn an even stronger language model, \citet{radford2019language} propose \ac{GPT}-2, a much larger version of \ac{GPT} that increases the number of parameters from 100 million to 1.5 billion. Similar to T5 \citep{raffel2020exploring}, \ac{GPT}-2 seeks to perform tasks via self-supervised language modeling, without explicit fine-tuning with labeled data. To achieve this, \citet{radford2019language} introduce \textit{task conditioning}, a probabilistic form for multi-task learning, which consists in predicting the output based on the input and task information, \textit{i.e.}, $P(output \mid input, task)$. Task conditioning is performed by providing examples of natural language instructions to perform a task, \textit{e.g.}, for English to French translation, the model is given an English sentence followed by \say{French: }. Therefore, input to \ac{GPT}-2 is given in a format which expects the model to understand the nature of the task. Trained on a sufficiently extensive and large dataset \citep{radford2019language}, \ac{GPT}-2 achieved state-of-the-art performance on language modeling benchmarks \citep{marcus1993building, chelba2013one, merity2016pointer} in zero-shot scenarios. On downstream tasks such as question answering, summarization, and translation, \ac{GPT}-2 demonstrates the ability to learn these tasks directly from raw text, without relying on task-specific training data. The model's versatility in handling various tasks in a zero-shot setting suggests that high-capacity models, trained to optimize the likelihood of diverse text corpora, inherently learn how to perform a remarkable range of tasks without the need for explicit supervision.


% Task conditioning forms the basis for zero-shot task transfer 


\subsection{Large Language Models}

These decoder-only Pre-trained Language Models have showcased their ability to generate coherent and contextually relevant text, estabilishing them as versatile tools for various \ac{NLP} tasks. Researchers have observed that scaling Pre-trained Language Models, whether by increasing model size or training data, frequently results in enhanced model capacity for a variety of downstream tasks. This phenomenon aligns with the scaling law, as suggested by \citet{kaplan2020scaling}. The success of decoder-only Pre-trained Language Models has spurred further development of larger and more sophisticated decoder-only Pre-trained Language Models, now referred to as Large Language Models.

In particular, the next iteration in the \ac{GPT} series, \ac{GPT}-3 \citep{brown2020language}, represents a significant milestone in the progression from Pre-trained Language Models to Large Language Models. \ac{GPT}-3 is a slightly modified version of \ac{GPT}-2 that demonstrates a significant capacity leap by scaling to a staggering size of 175 billion of parameters. \ac{GPT}-3 introduced the concept of \textit{in-context} learning, which allows the model to perform specific tasks by conditioning its responses on context provided in the prompt. In-context learning, also referred to as \textit{prompting}, encompasses zero-shot, one-shot, and few-shot learning. \ac{GPT}-3 uses the same architecture as \ac{GPT}-2 with the exception that attention patterns are sparse at alternating layers. Pre-trained on an even larger dataset, \ac{GPT}-3 has empirically shown that scaling Pre-trained Language Models to a significant size and formulating text to guide models to perform specific tasks (in-context learning) can lead to a huge increase in model capacity, especially in few and zero-shot learning scenarios.

% It has empirically demonstrated that scaling neural networks to a significant size and formulating text to induce models to perform desired tasks (in-context learning) can result in a huge increase in model capacity, especially in few and zero-shot learning.

% The idea is that, during pre-training, language models develop pattern recognition while learning to predict the following word conditioned on the context. Therefore, Pre-trained Language Models may be able to generate the correct task solution (formatted as a text sequence) given the task desk description, task-specific input-output examples, and a prompt. In-context learning, also referred to as \textit{prompting}, encompasses zero-shot, one-shot, and few-shot learning. 

The series of \ac{GPT} models has allowed significant progress in the field of \ac{NLP} by demonstrating the power of Large Language Models. Building on the success of the \ac{GPT} series, a myriad of Large Language Models have been released \citep{scao2022bloom, chowdhery2022palm, touvron2023llama}. The progress in Large Language Models has expanded into more specific domains, with models tailored for specialized tasks such as medical language processing \citep{thirunavukarasu2023large}, scientific research \citep{wang2023scientific}, website development \citep{wang2023software}, and code generation \citep{xu2022systematic}. A groundbreaking application of Large Language Models is ChatGPT \footnote{\url{https://openai.com/blog/chatgpt}}, an adaptation of the \ac{GPT}-series designed for dialogue. ChatGPT exhibits exceptional conversational abilities with humans and has sparked a wealth of reviews and discussions on the advancements of Large Language Models \citep{zhao2023survey, mohamadi2023chatgpt, hadi2023large}. 

In recent years, the size of Pre-trained Language Models has been scaled from a few million parameters (\acp{BERT}, 110M) to hundreds of billions of parameters (PaLM, 540B). This scaling has boosted capabilities, enabling models to generate even more coherent and natural-sounding text and further pushing the boundaries of \ac{NLP}. Despite substantial progress, there are still limitations associated with Large Language Models. First, pre-training effective Large Language Models is challenging due to the very high computational needs and the sensitivity to data quality and training settings. In addition, these have a tendency to generate inaccurate information, either conflicting with existing sources or unverifiable by available sources \citep{bang2023multitask}. Even robust Large Language Models such as ChatGPT encounter significant difficulties in managing hallucinations within generated texts. Another significant challenge for Large Language Models is their limitation in solving tasks that require the latest knowledge beyond what was available in their training data. Fine-tuning such large models with new data is a costly process and can potentially lead to catastrophic forgetting when incrementally training these models. Keeping up with and incorporating information from real-time or rapidly evolving context remains an open research problem \citep{yao2023editing}. Besides, Large Language Models may struggle with tasks that demand generating structured data \citep{jiang2023structgpt} or require domain-specific knowledge \citep{ye2023comprehensive}. Incorporating this knowledge into these models while preserving their original capabilities is non-trivial. Furthermore, Large Language models have been shown to internalize, spread, and potentially amplify harmful information present in the training data they are exposed to. This often includes toxic language such as offensiveness, hate speech, and insults \citep{gehman2020realtoxicityprompts}, as well as social biases such as stereotypes directed towards individuals with specific demographic identities \citep{sheng2021societal}. Addressing all of these limitations remains an ongoing area of research to enhance the robustness and ethical deployment of Large Language Models in various applications.

% Large Language Models have opened up new possibilities for \ac{NLP}, further pushing the boundaries of \ac{NLP}.

% During the past few years, GenAI models size has been scaled from a few million parameters(BERT [45], 110M) to hundreds of billions of parameters (GPT [112], 175B). Generally speaking, as the size of the model (number of parameters) increases, the performance of the model also increases [113], and it can be generalized for a variety of tasks [114], for example, Foundation models [115]. However, smaller models can also be fine-tuned for a more focused task [116]


% Large language models offer an exciting prospect of formulating text input to induce models to perform desired tasks via in-context learning, which is also known as prompting. For example, chain-of-thought prompting (Wei et al., 2022), an in-context learning method with few-shot "question, intermediate reasoning steps, answer" demonstrations, elicits the complex reasoning capabilities of large language models to solve mathematical, commonsense, and symbolic reasoning tasks. Sampling multiple reasoning paths (Wang et al., 2023), diversifying few-shot demonstrations (Zhang et al., 2023), and reducing complex problems to sub-problems (Zhou et al., 2023) can all improve the reasoning accuracy. In fact, with simple prompts like "Let’s think step by step" just before each answer, large language models can even perform zero-shot chain-of-thought reasoning with decent accuracy (Kojima et al., 2022). Even for multimodal inputs consisting of both text and images, language models can perform multimodal chain-of-thought reasoning with further improved accuracy than using text input only (Zhang et al., 2023).

\acresetall

