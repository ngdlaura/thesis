\chapter{Related Work}
\label{chapter:related}


\renewcommand{\leftmark}{\spacedlowsmallcaps{Related work}}

\ifthenelse{\boolean{skipRelated}}{\endinput}{}

\minitoc

\chapterwithfigures{\nameref*{chapter:related}}
\chapterwithtables{\nameref*{chapter:related}}


% “Language is essentially a complex, intricate system of human expressions governed by grammatical rules. It poses a significant challenge to develop capable artificial intelligence (AI) algorithms for comprehending and grasping a language. As a major approach, language modeling has been widely studied for language understanding and generation in the past two decades.” (https://browse.arxiv.org/pdf/2303.18223.pdf)

% In recent years, the AI technology that has arguably advanced the most is foundation models (Bommasani et al., 2021), headlined by the rise of language models (LMs; Peters et al., 2018; Devlin et al., 2019; Brown et al., 2020; Rae et al., 2021; Chowdhery et al., 2022).

% Since their emergence, language models have constantly improved the state of the art in most NLP benchmarks. In this chapter we characterize the main approaches to ...

% The contemporary landscape of natural language processing (NLP) has witnessed a paradigm shift, propelled by advancements in language models

Language, a complex and intricate system guided by grammatical rules, stands out as a distinctive human ability that develops in early childhood and undergoes continuous evolution throughout a lifetime \citep{hauser2002faculty}. However, unlike humans, comprehending and communicating in human language poses a significant challenge for \ac{AI} algorithms. Achieving this goal has been a longstanding research challenge, aiming to empower machines with the capability to read, write, and communicate in a manner akin to humans. In particular, \textit{language modeling} has emerged as a major approach, extensively studied for language understanding and generation over the past two decades. Since their emergence, language models have consistently pushed the boundaries of state-of-the-art performance across various \ac{NLP} benchmarks. In recent years, \textit{foundation models}, notably represented by the ascent of language models \citep{peters-etal-2018-deep, devlin2018bert, brown2020language}, stand out as a category of \ac{AI} technology that has arguably witnessed the most significant advancements. As a key application area, Document Understanding, \textit{i.e.}, automated information processing, emerges as a critical domain where these advancements find substantial utility.

%Notably, these models have contributed significantly to automated information processing, \textit{i.e.}, Document Understanding, thereby propelling digital transformation.

In this chapter, we explore the literature landscape that forms the foundation for our research. We begin by delving into the extensive literature that underpins the evolution of language models. We explore the foundational aspects of language models, tracing their trajectory from the early era of statistical language models to the advent of neural language models and the contemporary era of foundation models. We then delve into long-range modeling and recent developments in modeling techniques that address the challenges posed by long sequences. Finally, we provide an overview of Document Understanding, a pivotal application domain where intricate interplay of text, layout, and visual elements within documents poses unique challenges.

\section{Language Modeling}

Language modeling stands out as the major approach to advancing language understanding and generation. A language model is a probabilistic model designed to capture the probability distribution of words within a given language, thereby constructing effective representations of text. Originally conceived for text generation, language models have recently emerged as a powerful means to establish parametric models that can be fine-tuned on a wide range of tasks. 
In this section, we explore the diverse tasks in \ac{NLP}, discuss the building blocks and historical approaches for Language Modeling, and describe how language models are evaluated, examining both automatic and human evaluation methods.

\subsection{Tasks} 

The primary goal of language models is to understand and generate human-like text. Playing a pivotal role in numerous \ac{NLP} tasks encompassing both text understanding and generation, language models are crucial in advancing our understanding of language. We explore several \ac{NLP} tasks that have received extensive attention due to their practical significance and their role in advancing the understanding on language. Addressing these challenges requires models that can comprehend semantics, context, and syntactic structures in text, making them central to the development of sophisticated \ac{NLP} systems.

\subsubsection{Natural Language Understanding}

Comprising a broad array of tasks, \textit{Natural Language Understandign} focuses on the ability of machines to process written language. 

\paragraph{Text Classification} involves categorizing text into one or more pre-defined classes or categories. This task finds applications in various scenarios, including sentiment analysis, spam detection, and content moderation. Automating these processes through language models can streamline data management, decrease manual workload, and enhance the accuracy and efficiency of analysis. 

\paragraph{Information Extraction} consists in automatically extracting structured information from unstructured and/or semi-structured documents, primarily texts. The goal of information extraction is to convert large volumes of textual data into a more organized and usable format, enabling machines to  understand the content. Information extraction involves identifying specific pieces of information, such as entities (\textit{e.g.}, person names, organizations, and quantities), relationships between entities, and events, within a given text. Key tasks include \ac{NER}, which seeks to identify and classify named entities into pre-defined categories, and Relationship Extraction, where the goal is to determine relationships or connections between different entities mentioned in the text. 

\paragraph{\ac{NLI}} is the task of determining the relationship between two given texts. Given a source text and a target text, the relationship between them can be categorized into three classes: \textit{entailment} (the target text implies the source), \textit{contradiction} (the source is false), and \textit{neutral} (there is no relation between the source and the target texts).

\paragraph{Coreference Resolution} is the task of finding all linguistic expressions (or \textit {mentions}) that refer to the same entity in a text. It constitutes a crucial stage for many higher level \ac{NLP} tasks that require a deep understanding of natural language, \textit{e.g.}, information extraction.

\subsubsection{Natural Language Generation} 

\textit{Natural Language Generation} focuses on the automatic generation of human-like language. The primary goal of natural language generation is to enable machines to produce coherent and contextually appropriate text.

\paragraph{Text Generation} refers to the process of automatically creating human-like text for diverse purposes, such as articles, blogs, research papers, social media posts, source codes, and more.

\paragraph{Text Summarization} is a generation task that aims to generate concise and coherent summaries from lenghty texts. Summarization can be categorized into two categories: \textit{extractive} summarization and \textit{abstractive summarization}. Extractive summarization consists in selecting and combining existing sentences from the text to create the summary. On the other hand, abstractive summarization goes beyond verbatim copying and may generate new phrases and sentences that are not present in the source text. Abstractive summarization, with its ability to generate more concise and coherent summaries, has the potential to capture the overall meaning of a text. Overall, text summarization is a crucial component in the development of applications that require efficient information processing, allowing users to access relevant information more quickly and effectively. It plays a significant role in reducing information overload and improving the accessibility of large volumes of text.

\paragraph{Machine Translation} is the automated process of translating text from one language to another. The aim of machine translation is to produce translations that are linguistically accurate and convey the intended meaning of the source text in the target language. Machine translation finds application in a range of domains and industries, including language service providers, global businesses, content localization and information access. 

\paragraph{Question Answering} involves providing accurate and relevant answers to questions posed in natural language. It encompasses various types of questions, and the answers can be generated from a variety of sources, including knowledge bases, databases, documents, or a combination of sources. Question answering tasks can be open-domain or closed-domain, fact-based or reasoning-based. The emphasis is on formulating an appropriate response to a question. \textit{Extractive} and \textit{abstrative} question answering are two different approaches to formulating answers for questions. Extractive question answering involves selecting and extracting a span of text from a document as the answer to a question. While maintaining factual accuracy, this approach is limited to information explicitly present in the document. To produce concise answers and allow for potential novel insights, abstractive question answering consists in generating a response that may not be explicitly stated in the document. As such, it potentially involves rephrasing or synthesizing information to provide a concise and coherent response. This approach is more challenging as it requires ensuring the generated answers are accurate and contextually appropriate. It has found wide application in scenarios such as search engines and customer support.

Machine Reading Comprehension is a specific type of question answering task that focuses specifically on questions related to a given passage of text, and requires comprehending and extracting information from that passage. The questions are typically formulated based on the content of the provided text, and the goal is to understand the text and extract relevant information to answer the questions. Usually, the passage is provided and the goal is to extract the answer directly from it.

\paragraph{Dialog Systems} (also known as chatbots), are designed to engage in natural language conversations with users. They play a crucial role in human-machine interaction, facilitating effective communication between humans and machines. Dialog systems are required to comprehend and interpret user input, keep track of the conversation context, create responses that are appropriate and linguistically coherent, and maintain an understanding of the state of the conversation and user preferences throughout the interaction. Their applications span various domains such as customer service, education, and entertainment. \\

\subsection{Language Modeling}

Language Models have incited substantial interest across both academic and industrial domains, owing to their unprecedented performance in various tasks and domains, including medical language processing \citep{thirunavukarasu2023large}, scientific research \citep{wang2023scientific}, and code generation \citep{xu2022systematic}.

In the early days of \ac{NLP}, researchers developed rule-based systems to process language \citep{manning1999foundations}. These systems relied on handcrafted linguistic rules to analyze and generate text. While these approaches were valuable for specific tasks, they lacked the ability to capture the richness and variability of natural language. This limitation paved the way for the emergence of \textit{Statistical Language Models}. 

Language modeling aims to predict the next element in a given sequence of text. We begin by discussing text representation units and the methods employed to obtain them. We then explain how probabilities over text sequences are calculated, before delving into the early iterations of language models, \textit{i.e.}, Statistical Language Models.

\subsubsection{Text Representation Units}

Natural language inputs can be commonly modeled as sequences of tokens, with various levels of granularities — characters, words, sentences, and more. To pass these sequences to a model, text is usually tokenized. Tokenization is a crucial pre-processing step that consists in splitting the input text into smaller units, \textit{i.e.}, tokens. Tokens serve as the fundamental components of language modeling, and all models operate on raw text at the token level. These tokens are used to build the vocabulary, which represents a set of unique tokens within a corpus. A token can be a character, a word, or a subword. Various algorithms adopt distinct processes to perform tokenization. 

\paragraph{Word-based Tokenization} divides a text into words using a delimiter, with space and punctations being the most commonly employed. Rules are added into the tokenization process to deal with special cases such as negative forms (for instance, space and punctuation-based tokenization generates three tokens for the word “don't”:  “don”, “'”, and “t”, whereas a more effective tokenization using specific rules would  break it into “do”, and “n't”).

In English, words like “helps”, “helped”, and “helping” are derived forms of the base word “help”. Similarly, the relationship between “dog” and “dogs” is analogous to that between “cat” and “cats”, and “boy” and “boyfriend” show the same relationship as “girl” and “girlfriend”. In some other languages like French and Spanish, verbs can have more than 40 inflected forms. However, word-based tokenization does not cater for the internal structure of words, as morphological information, i.e., word formation and relationships, are not taken into account by the tokenization process. Instead, different inflected forms of the same word (e.g., “cat” and “cats”) are tokenized into two distinct tokens. Consequently, models would fail to recognize the similarity between those words. In addition, word-based tokenization leads to a very large vocabulary. Furthermore, at inference, words not included in the vocabulary must be handled. This typically involves the use of an \ac{OOV} token, a practice that often contributes to sub-optimal results.

\paragraph{Character-based Tokenization} \citep{wehrmann2017character} can be used to alleviate the vocabulary problem. This tokenization process splits the raw text into individual characters, resulting in a very small vocabulary with little to no \ac{OOV} words. 

However, few languages convey a significant amount of information within each character. Therefore, character-based tokenization suffers from a weak correlation between characters and semantic/syntactic aspects of the language. Furthermore, working at the character level results in much longer sequences, which are more challenging to deal with.

\paragraph{Subword-based Tokenization} Modern \ac{NLP} models address both word and character-based tokenization issues by tokenizing a text into \textit{subword} units, a solution between word and character-based tokenization. Subword-based tokenization algorithms use the following principles: 1) frequently used words should not be split into smaller subwords, and 2) rare words should be split into smaller, meaningful words. 

\citet{gage1994new} proposed the \ac{BPE} method, a compression algorithm that breaks down words into subwords to form a compact, fixed-size vocabulary with subwords of varying lengths. The \ac{BPE} algorithm performs a statistical analysis of the training dataset to identify common symbols within words, e.g., consecutive characters of arbitrary lengths. It starts with an initial vocabulary consisting of symbols of length 1 (characters), and iteratively merges the most frequent pairs of adjacent symbols to produce new, longer symbols. The process stops until a specified number of iterations or a predefined vocabulary size is reached. The resulting symbols can be used as subwords to segment words. \ac{BPE} is widely used for input representations in \ac{NLP} models, and has contributed significantly to improving their performance by enhancing their ability to handle morphologically-rich languages and \ac{OOV} words.

WordPiece \citep{wu2016google} is another subword segmentation algorithm. Similar to \ac{BPE}, WordPiece learns merge rules. To build the vocabulary, it starts from a word unit inventory including individual characters in the language and special tokens used by the model. Using this inventory, a language model is built on the training data. A new word unit is obtained by combining two units out of the current word inventory. This increments the word unit inventory by one. From all possible combinations, the new word unit is selected such that it yields the highest increase in the likelihood on the training data after its addition to the model. From the updated inventory, a new language model is built and the process is repeated until a predefined limit of word units is reached or the likelihood increase falls below a certain threshold. 

% Tokenization differs in WordPiece and \ac{BPE} in that WordPiece only saves the final vocabulary, not the merge rules learned. Starting from the word to tokenize, WordPiece finds the longest subword that is in the vocabulary, then splits on it. 

Subword-based tokenization often maintains linguistic meaning, such as morphemes. Consequently, even though a word may be unknown to the model, individual subword tokens may retain enough information for the model to deduce its meaning to a certain degree. Additionally, using subword units helps keeping the vocabulary at a reasonable size.


\subsubsection{Language Model Definition}

% Language models give us the ability to assign such a conditional probability to every possible next word, giving us a distribution over the entire vocabulary. We can also assign probabilities to entire sequences by combining these conditional probabilities with the chain rule

A language model is a probabilistic model of a natural language that predicts probability distributions over sequences of tokens. Given a sequence of tokens $w_1, w_2, ..., w_n$, a language model aims to calculate the joint probability $P(w_1, w_2, ..., w_n)$ of the whole sequence. Using the chain rule, the probability of the sequence can be decomposed into a product of conditional distribution on tokens. Most commonly, the probability $P$ of a sequence of words can be obtained from the probability of each word given the preceding ones:

\begin{equation}
    P(w_1, ..., w_n) = \prod_{t=1}^{n} P\bigl(w_t \mid w_1, ..., w_{t-1}\bigr).
\label{equation:causal-distribution}
\end{equation}

In other words, the probability of a sequence is estimated as a product of each token's probability given its preceding tokens. \textit{Causal}\footnote{This name is common in the literature but is misleading as it has little connection to the proper study of causality.}, or \textit{autoregressive} language models use this decomposition.

A successful language model estimates the distribution across text sequences, encoding not only the grammatical structure, but also the potential knowledge embedded in the training corpora \citep{jozefowicz2016exploring}.

\subsubsection{Statistical Language Models} The history of language models can be traced back to the 1990s, a period that marked the emergence of Statistical Language Models. Such language models are rooted in probabilistic approaches to predict word sequences. The underlying idea is to simplify the word prediction model using the Markov assumption, \textit{e.g.}, approximating the probability of the next word using the most recent context. Prominent examples including $n$-gram models \citep{brown1992class, omar2018arabic} and \acp{HMM} \citep{petrushin2000hidden}.

\paragraph{$N$-gram Models} simplify the calculation of the joint probability by operating on the assumption that the likelihood of the next token in a sequence is solely dependent on a fixed-size window spanning the $n-1$ previous adjacent tokens (\textit{$n$-grams}). If only one prior token is considered, it is termed a bigram model; with two words, a trigram model; and with $n-1$ words, an n-gram model. Given a window size $k$, the calculation of the joint probability is simplified as follows:

\begin{equation}
    P(w_1, ..., w_n) \approx \prod_{t=1}^{n} P\bigl(w_t \mid w_{t-k}, ..., w_{t-1}\bigr).
    \label{equation:lm-likelihood-markov}
\end{equation}

$N$-grams models calculate Equation~\ref{equation:lm-likelihood-markov} using frequency counts based on $n$-grams. 

\paragraph{Hidden Markov Models} are latent-variable models that are able to fully separate the process of generating hidden states from observations, while allowing for exact posterior inference. Given a sequence of observed tokens $\bm{w} = (w_1, \ldots, w_n)$, \acp{HMM} specify a joint distribution over observed tokens $\bm{x}$ and discrete latent states $\bm{z} = (z_1, \ldots, z_n)$:

\begin{equation}
    P(\bm{w}, \bm{z}; \theta) = \prod_{t=1}^{n} P\bigl(w_t \mid z_t \bigr) P\bigl(z_t \mid z_{t-1} \bigr). \\
\end{equation}


\paragraph{On the Curse of Dimensionality} Statistical Language Models represent tokens through one-hot encoding, where each token is represented as a sparse binary vector, with a dimension for each unique token in the vocabulary. In this encoding, all dimensions are zero except for the one corresponding to the token, which is set to one. Hence, one-hot encoding leads to very high-dimensional and sparse representations. This often hinders the accurate estimation of language models, as one-hot encoding requires estimating an exponential number of transition probabilities. Furthermore, one-hot encoding introduces greater difficulty in capturing semantic relationships between tokens (each individual token is treated independently of the others) and handling \ac{OOV} tokens efficiently. This phenomenon is referred to as the \textit{curse of dimensionality}. To tackle this issue, specific smoothing strategies, including backoff estimation \citep{katz1987estimation} and Good-Turing estimation \citep{gale1995good}, have been introduced to alleviate the problem of data sparsity. \\

% Nevertheless, the curse of dimensionality often hinders the performance of Statistical Language Models, making the accurate estimation of high-order language models challenging. This difficulty arises from the necessity to estimate an exponential number of transition probabilities. To tackle this issue, specific smoothing strategies, including backoff estimation \citep{katz1987estimation} and Good-Turing estimation \citep{gale1995good}, have been introduced to alleviate the problem of data sparsity.

Statistical Language Models have found extensive application in boosting performance across \ac{NLP} tasks \citep{bahl1989tree, thede1999second}. While these models may appear rudimentary by today's standards, they represent a pivotal starting point in the field of \ac{NLP}. Although capable of basic text generation and word prediction, their limitations become apparent when attempting to capture complex contextual relationships \citep{rosenfeld2000two, arisoy2012deep}.


\subsection{Evaluation of Language Models}

As language models play an increasingly critical role in both research and daily applications, the importance of their evaluation grows significantly. The evaluation of language models stands as a crucial phase in assessing their efficacy and performance, bridging the gap between theoretical advancements and practical utility. We explore \textit{automatic evaluation} with computational metrics, and \textit{human evaluation} using qualitative assessments. 

\subsubsection{Automatic Evaluation} 

Several key metrics can be employed to provide valuable insights into the capacities and limitations of a language model. Language models can be evaluated using \textit{intrinsic} or \textit{extrinsic} evaluation. 

\paragraph{Intrinsic Evaluation} An intrinsic evaluation metric measures the quality of the language model independently of any application, and can be used to quickly assess potential improvements in the model.

Perplexity is a widely used intrinsic metric that measures how well a language model predicts a sample. Given an input sequence $\bm{w} = (w_1, \ldots, w_n)$, and $P(w_1, \ldots, w_n)$ the probability assigned to $\bm{w}$ by the model, the perplexity of $\bm{w}$ can be defined as the multiplicative inverse of $P(w_1, \ldots, w_n)$, normalized by the number of words in the test set:

\begin{equation}
    \text{PPL}(\bm{w}) = P(w_1, \ldots, w_n)^{\frac{1}{n}}
\end{equation}

Perplexity quantifies how uncertain a model is about the predictions it makes. The lower the perplexity of a language model, the more confident (but not necessarily accurate) it is. Perplexity often correlates well with the model's performance on the target tasks, and it can be easily computed from the probability distribution learned during training. Hence, perplexity is a reliable metric to filter out models that are unlikely to perform well in real-world scenarios, where computing is costly and testing is time-consuming. However, comparing perplexity across different datasets, context lengths, vocabulary sizes, and tokenization procedures is challenging. These differences can significantly influence model performance, necessitating careful consideration and adjustment for fair evaluation.

Cross-entropy is another intrinsic metric used to measure the performance of a language model. Suppose $n$ the number of tokens, $m$ the vocabulary size, $\bm{y}$ the ground-truth vector, and $\bm{p}$ the vector of output probabilities. Cross-entropy can be calculated as:

\begin{equation}
    \text{CE}(\bm{y}, \bm{p}) = - \dfrac{1}{n} \sum_{i}^n \sum_{j}^m y_{ij} \log (p_{ij}).
\end{equation}

\noindent When $m = 2$, binary cross-entropy can be computed as:

\begin{equation}
    \text{BCE}(\bm{y}, \bm{p}) = - \dfrac{1}{n} \sum_{i}^n (y_i \log(p_i) + (1-y_i) \log (1-p_i))
\end{equation}

\noindent Cross-entropy loss increases as the predicted probability diverges from the actual label. Hence, it is minimized when adjusting model weights during training. 

\ac{BPC} is a measurement used to quantify the efficiency of encoding text using a specific model. It calculates the average number of bits needed to represent each character in a text using the model's encoding scheme. The lower the \ac{BPC} value, the more efficient the model is at encoding the text, indicating that the model is effectively capturing the patterns and structure of the language. This metric is often used to assess the performance and compression capabilities of language models. Given an input sequence $\bm{w} = (w_1, \ldots, w_n)$, \ac{BPC} is defined as:

\begin{equation}
    \text{BPC}(\bm{w}) = - \dfrac{1}{n} \sum_{i=1}^n \log_2 P(w_i).
\end{equation}

\noindent Notably, \ac{BPC} serves as a metric for evaluating models in the Hutter Prize contest and its associated enwiki8 benchmark on data compression.\footnote{\url{http://prize.hutter1.net/}}

\paragraph{Extrinsic Evaluation and Benchmarks} However, good scores during intrinsic evaluation do not always translate to better performance in downstream tasks. Therefore, extrinsic evaluation, also called task-based evaluation, is used to gauge how useful the language model is in a particular task. As proper evaluation is a challenging task \citep{jones2005some}, benchmarking emerged as a prominent methodology in the 1980-1990s to address this challenge. 

The Penn Treebank corpus \citep{marcus1993building}, specifically the section dedicated to Wall Street Journal articles, stands out as one of the most widely used annotated English dataset for evaluating models on sequence labelling. The dataset is renowned for its detailed syntatic annotations, providing a tree-like structure that represents the grammatical structure of sentences. The task involves assigning each word a \ac{POS} tag. 

More recently, the \ac{SNLI} dataset, a larger corpus of sentence-pairs annotated from Flickr30k image captions, has been proposed to train and evaluate models for \ac{NLI} tasks. Additionally, the \ac{SQuAD} \citep{rajpurkar2016squad} dataset, a collection of question-answer pairs derived from Wikipedia articles, serves as an evaluation benchmark for question answering models. 

With the rise of more general-purpose methods in \ac{NLP}, often replacing task-specific methods, the emergence of new and exhaustive benchmarks followed suit. SentEval \citep{conneau2018senteval} is a toolkit crafted for evaluating the quality of universal sentence representations. It covers an array of tasks, including binary and multi-class classification, \ac{NLI}, and sentence similarity. Simultaneously, the \ac{GLUE} benchmark \citep{wang2018glue} has been developped to train and assess the performance of natural language understanding models across a diverse set of language tasks. \ac{GLUE} covers nine sentence/sentence-pair language understanding tasks (\textit{e.g.}, grammaticality judgments, sentence similarity, \ac{NLI}) selected to cover a broad array of dataset sizes, text genres, degrees of difficulty, and various linguistic aspects. The goal of \ac{GLUE} is to encourage the development of models that can generalize well, exhibit a broad understanding of natural language, and demonstrate robust performance across different tasks. 
These benchmarks offer both a training set and an evaluation set for each task, enabling researchers to train models on one subset of the data and evaluate their performance on another, ensuring fair assessments of generalization. Additionally, unlike earlier benchmarks, they assign each model a vector of scores to gauge accuracy across a range of scenarios.


\subsubsection{Human Evaluation} 

Human evaluation consists in having human annotators evaluate the quality of generated text on specific tasks. Annotators can rate the generated text based on its fluency, coherence, and relevance to the given output. Human evaluation considers factors that might be difficult to quantify, \textit{e.g.}, the overall quality of the generated text, creativity, or the ability to handle ambiguous or nuanced language. While it can be time-consuming and subjective, human evaluation offers valuable insights into how language models perform in real-world scenarios. Integrating human judgment helps uncovering potential limitations, biases, or domains where models might struggle. The \ac{GEM} benchmark \citep{gehrmann2021gem} introduces a set of natural language generation tasks in diverse languages, emphasizing evaluation through both automated metrics and human annotations. 


%%%%%%%%%%%%%%%%%%%%%%%%%%%%%%%%%%%%%%%%%%%%%%%%%%%%%%%%%%%%%%%%%%%%%%%%%%%%%%%%%%%%%%%%%%%%%%%%%%%%
%%%%%%%%%%%%%%%%%%%%%%%%%%%%%%%%%%%%%%%%%%%%%%%%%%%%%%%%%%%%%%%%%%%%%%%%%%%%%%%%%%%%%%%%%%%%%%%%%%%%

\section{Neural Language Models}

Starting in the 2000s, neural networks began to be used for language modeling \citep{bengio2000neural}, and representation of text shifted from being non-continuous to being continuous (\textit{distributed}). The mid-2010s marked a significant milestone in language modeling with the emergence of Deep Learning, laying foundation for the developement of \textit{Neural Language Models}. A Neural Language Model is a language model that exploits the ability of neural networks to learn distributed representations of text. Neural Language Models delve into vast amounts of data to learn the intricate patterns and structures of language, allowing them to significantly improve their ability to understand context. In this section, we first describe how distributed representations of textual data can be obtained from neural architectures, before exploring notable word embedding models.

% NLMs characterize the probability of word sequences by neural networks

% Starting in the 2000s, neural networks begin to be used for language modeling, a task which aims at predicting the next word in a text given the previous words. In 2003, Bengio et al. proposed the first neural language model, that consists of a one-hidden layer feed-forward neural network. They were also one of the first to introduce what is now referred as word embedding, a real-valued word feature vector in R^d. More precisely, their model took as input vector representations of the n previous words, which were looked up in a table learned together with the model. The vectors were fed into a hidden layer, whose output was then provided to a softmax layer that predicted the next word of the sequence. 

% As a remarkable contribution, the work in [15] introduced the concept of distributed representation of words and built the word prediction function conditioned on the aggregated context features (i.e., the distributed word vectors).

\subsection{Distributed Representations of Text}

A fundamental challenge that renders language modeling challenging is the curse of dimensionality, primarily stemming from the sparse and high-dimensional nature of one-hot encodings. Text representation has therefore evolved from a non-continuous to a continuous (\textit{i.e.}, distributed) form.

Initial attempts to estimate continuous word representations, as seen in methods like \ac{PLSA} \citep{hofmann2001unsupervised} and \ac{LDA} \citep{blei2003latent}, centered around extracting embeddings from co-occurrence matrices to represent words and documents in a latent topical space. Such continuous representations, while effective for their intended purposes such as document clustering and term analysis, have limited success when applied to a broader range of \ac{NLP} tasks. These limitations arise from a couple of key factors. Firstly, these methods overlook the context and relationships between words, essential for understanding natural language in diverse applications. Second, the representations derived lack the adaptability needed for the specific requirements of various \ac{NLP} tasks.

% \textit{Distributed word representations}, or \textit{word embeddings}, are dense and low-dimensional continuous-valued representations that serve as the building blocks for contemporary \ac{NLP}. 

Due to these limitations, researchers and practitioners have shifted towards using embeddings generated by neural network-based models, \textit{i.e.}, \textit{distributed representations}. The concept of distributed representations of words, also known as \textit{word embeddings} was introduced by \citet{bengio2000neural}. Distributed word representations are dense, low-dimensional, continuous-valued representations that rely on the distributional hypothesis, which posits that words with similar contexts have similar (or related) meaning. Using distributed representations requires a much smaller number of features than the size of the vocabulary. Additionally, the continuous space in which the vocabulary is embedded can represent the similarity structure between words, \textit{i.e.}, words with related meanings appear in the same region in the embedding space \citep{shazeer2016swivel}. Therefore, generalization can be obtained more easily: a sequence of words that has never been encountered before is assigned a high probability if it consists of words with representations similar to those forming an already seen sentence. 

% A word embedding is a learned representation in which, ideally, words with related meanings or contextual relationships become highly correlated in the representation space. One of the main incentives behind word embeddings is their high generalization power, as opposed to sparse, higher dimensional representations \citep{bengio2000neural}.

% Distributed word representations are based on the distributional hypothesis where words that co-occur in similar context are considered to have similar (or related) meaning.

The same work by \citet{bengio2000neural} introduces Neural Language Models. Using neural networks, Neural Language Models learn the probability distribution of a word sequence given the previous context, while embedding the vocabulary in a continuous space to obtain distributed representations. Given a context of size $k$, and a training sequence of words $(w_1, \ldots, w_n) \in V^n$, where $V$ is the vocabulary, the objective is to estimate $P(w_t \mid w_1, \ldots, w_{t-1})$ through a neural network $f(w_t, \ldots, w_{t-k+1})$. The distributed word vectors are first built using a matrix $C \in \mathbb{R}^{\mid V \mid \times h}$ whose parameters are simply the distributed word vectors themselves: the $i$-th row in $C$ is the distributed representation $C_i \in \mathbb{R}^h$ for word $i$. To obtain the next word $w_t$, the probability function over words is expressed as a function $G$ that maps the input sequence of feature vectors in the context, $\bigl(C_{w_{t-k+1}}, \ldots, C_{w_{t-1}}\bigr)$ to a conditional probability distribution over words in $V$. The output of $G$ is a vector whose $i$-th element provides an estimate of the probability $P(w_t = i \mid w_1, \ldots, w_{t-1})$. The function $F$ is a composition of $C$ and $G$, expressed as follows:

\begin{equation}
    F(i, w_{t-1}, \ldots, w_{t-k+1}) = G\bigl(i, C(w_{t-1}), \ldots, C(w_{t-k+1})\bigr).
\label{eq:nlm-bengio}
\end{equation}

The model is trained by searching for the parameter set $\theta = (C, \omega)$ that maximizes the following penalized log-likelihood:

\begin{equation}
    L = \frac{1}{n} \sum_{t=1}^n \log F(w_t, w_{t-1}, \ldots, w_{t-k+1}; \theta) + R(\theta),
\label{eq:nlm-log-likelihood}
\end{equation}

\noindent Where $R(\theta)$ is a regularization term.

The function $G$ can be implemented as a feed-forward network or any another parametrized function. More specifically, Neural Language Models leverage the capabilities of the prevailing architectures of the 2010s, \acp{RNN} and \acp{CNN}.

% Neural language models are typically trained as probabilistic classifiers that learn to predict a probability distribution over a vocabulary $V$, given the context features:

\subsubsection{Recurrent Neural Networks}

\acp{RNN} are neural networks designed to deal with sequential data. The key feature of \acp{RNN} is their ability to maintain “memory” across time steps, allowing them to process each token of a sequence using information from prior tokens. This has allowed them to reach superior performance in handling sequential data and capturing the context and semantics present in natural language.  

\noindent Formally, an \ac{RNN} is a function parameterized by a set of parameters $\theta$ shared across all time steps. At each time step $t$, the model receives an input $\bm{x}_t$ and a fixed-size hidden state vector $\bm{h}_{t-1}$ from the previous time step $t-1$. The hidden state at time $t$ acts as a “memory” summarizing information from previous words $(w_1, \ldots, w_{t-1})$, and is computed as follows::

\begin{equation}
    \bm{h}_{t} = f_{\bm{\theta}}(\bm{x}_t, \bm{h}_{t-1}),  
\end{equation}

The distribution of probabilities over the output is computed as:

\begin{equation}
    \hat{\bm{y}}_t = g_{\bm{\theta}}(\bm{h}_t).
\end{equation}

\noindent $f_{\bm{\theta}}$ and $g_{\bm{\theta}}$ are activation functions, usually sigmoid, hyperbolic tangent, and \ac{ReLU} functions.

The recurrent connections allow information to be retained and updated over time, enabling \acp{RNN} to capture dependencies and temporal patterns in sequential data. \acp{RNN} are trained by minimizing the negative log-likelihood (\textit{i.e.}, maximizing the log-likelihood as in \ref{eq:chapter2-nlm-log-likelihood}) using the \ac{BPTT} algorithm \citep{werbos1990backpropagation}. \ac{BPTT} unrolls the computational graph of an \ac{RNN} one time step at a time, resulting in a feed-forward network with the special property that the same parameters are repeated throughout the unrolled network. Gradients are then backpropagated through the unrolled net, and accumulated in order to update $\bm{\theta}$. 

Complications arise because sequences can be rather long. This is linked to the vanishing and exploding gradients problems that come from repeated application of recurrent connections \citep{hochreiter2001gradient}. When gradients are backpropagated through multiple time steps, gradients can become too small, leading to slow learning or information disappearing over time. Hence, this can affect the ability of \acp{RNN} to capture long-term dependencies. On the contrary, gradients can also become too large, causing the model's parameters to be updated dramatically, resulting in unstable training. 

To alleviate the vanishing gradient problem and better model long-range dependencies, several variants of \acp{RNN} have been introduced, the most common one being \ac{LSTM} \citep{hochreiter1997long}. The key idea behind an \ac{LSTM} is to introduce special gated structures that allow the network to selectively remember or forget information over time. While a vanilla \ac{RNN} is a chain of very simple, repeated modules, an \ac{LSTM} is made of more complex modules, or \textit{cells}. At step $t$, a cell consists in a cell state $\bm{c}_t$, a cell candidate $\tilde{\bm{c}}_t$, a forget gate $\bm{f}_t$, an input gate $\bm{i}_t$, and an output gate $\bm{o}_t$. Given the input $\bm{x}_{t-1}$, the previous hidden state $\bm{h}_{t-1}$, along with the previous cell state $\bm{c}_{t-1}$, an \ac{LSTM} cell at step $t$ is defined by:

\begin{equation}
\begin{aligned}
    \tilde{\bm{c}_t} &= \tanh \left( \bm{W}_c \bm{h}_{t-1} + \bm{U}_c \bm{x}_t + \bm{b}_c \right)\\
    \bm{i}_t         &= \sigma \left( \bm{W}_i \bm{h}_{t-1} + \bm{U}_i \bm{x}_t + \bm{b}_i \right) \\
    \bm{f}_t         &= \sigma \left( \bm{W}_f \bm{h}_{t-1} + \bm{U}_f \bm{x}_t] + \bm{b}_f \right) \\
    \bm{o}_t         &= \sigma \left( \bm{W}_o \bm{h}_{t-1} + \bm{U}_o \bm{x}_t] + \bm{b}_o \right) \\
    \bm{c}_t         &= \bm{f}_t \odot \bm{c}_{t-1} + \bm{i}_t \odot \tilde{\bm{c}_t} \\
    \bm{h}_t         &= \bm{o}_t \odot \tanh (\bm{c}_t),
\end{aligned}
\end{equation}

\noindent where $\odot$ is the element-wise multiplication. More intuitively, the cell state $\bm{c}_t$ runs across the entire sequence and serves as an internal memory. The LSTM has the ability to remove or add information to the cell state by using gates. Gates are composed of a sigmoid activation function which outputs values between zero and one, describing how much of each component should be let through: a gate lets everything pass through if the value is one, and lets nothing through if the value is zero. The forget gate $\bm{f}_t$ determines what information to throw away from the cell state, the input gate $\bm{i}_t$ decides what new information to store in the cell state, while the output gate $\bm{o}_t$ controls what information from the cell state to pass to the next time step $t+1$. This gating mechanism allows \acp{LSTM} to control the flow of information and mitigates the vanishing gradient problem by breaking the multiplicative sequential gradient dependence. Hence, \acp{LSTM} are more robust at handling long sequences and preserving long-term dependencies.

\ac{RNN}-based language models process the input sequence one token at a time, predicting the next token from the current one and the previous hidden state. Unlike $n$-grams and feed-forward networks, \acp{RNN} do not have limited or fixed context in theory, as the hidden state should ideally encapsulate information about all preceding words, extending back to the beginning of the sequence.
\ac{RNN}-based (along with feed-forward network-based) language models have demonstrated superior performance over $n$-gram models in diverse setups \citep{mikolov2010recurrent}, producing more naturally-sounding text than previous language models \citep{kovavcevic2022bidirectional}. In addition, \ac{RNN}-based language models can be tailored for specific tasks. After training, the \ac{RNN} state $\bm{h}_t$ can serve as a representation of a text up to the word $w_t$. The final state $\bm{h}_{n}$ of a text corresponds to the representation of the entire text $(w_1, \ldots, w_n)$. Given that states have fixed dimensions, they can be applied to classification and regression tasks. \citet{schwenk2007continuous} have shown that \acp{RNN} provide significant improvements in speech recognition, while \citet{collobert2011deep} obtain close to state-of-the-art results on diverse morpho-syntactical labeling tasks.

\subsubsection{Convolutional Neural Networks}

\acp{CNN} constitute a family of neural network models characterized by a specific layer known as the convolutional layer. In this layer, features are extracted by convolving a learnable filter (or kernel) across various positions of a vectorial input.
\acp{CNN} were initially designed to deal with the hierarchical representations inherent to the Computer Vision field \citep{lecun1989backpropagation}. They are built upon two fundamental concepts: (1) the processing of an image region should not depend on its specific location (two-dimensional equivariance of data), (2) given the hierarchical nature of images, patterns should be captured at various levels of abstraction, progressing from regions composed of basic shapes to larger ones representing real-world objects. 

These concepts can also be applied to texts, where the translation equivariance is unidimensional rather than bi-dimensional. Let $(\bm{x}_1, \ldots, \bm{x}_n) \in \mathbb{R}^d$ be the input sequence of a convolutional layer, $\bm{K} \in \mathbb{R}^{w \times d' \times d}$ a kernel of width $w$, and $b \in \mathbb{R}^{d'}$ the bias term. One-dimensional convolution can be defined as:

\begin{equation}
    \bm{y}_i = \sum_{j=1}^w \bm{K}_j \bm{x}_{i-j+1} + \bm{b},
\end{equation}

\noindent where

\[
    \bm{x}_{i-j+1} = 
        \begin{cases}
            \bm{x}_{i-j+1}, & \text{if } 0 \leq i-j \leq n \\
            0,              & \text{otherwise}
        \end{cases}
\]

In the first layers of a \ac{CNN}, convolution is applied to word representations, \textit{i.e.}, $(\bm{x}_1, \ldots, \bm{x}_n)$ corresponds to a sequence of distributed representation of words. 

In \ac{NLP}, \acp{CNN} have mostly found application in static classification tasks for discovering latent structures in text, including sentiment analysis \citep{kalchbrenner2014convolutional}, topic categorization \citep{kim2014convolutional}, relation extraction \citep{nguyen2015relation}, and entity recognition \citep{adel2016comparing}. Additionally, they have demonstrated potential in sequential prediction tasks, such as language modeling \citep{pham2016convolutional} and \ac{POS} tagging \citep{collobert2011natural}. The popularity of \acp{CNN} is attributed to two key properties \citep{pham2016convolutional}: their ability to integrate information from larger context windows and their capacity to learn specific patterns at a high level of abstraction.

This analogy extends to the role of convolutions in the Computer Vision field, but it has its limitations. For instance, in language modeling, stacking convolution layers in a deeper model tends to harm performance \citep{pham2016convolutional}, unlike in Computer Vision where it significantly improves results. This difference is attributed to the nature of visual versus linguistic data. While convolution creates abstract images that retain crucial properties in the visual domain, when applied to language, it detects important textual features but distorts the input to the extent that it is no longer recognizable as text.

\subsection{Word Embedding Models}

Word embeddings have been shown to significantly improve and simplify many \ac{NLP} applications \citep{collobert2011natural}. However, the approach introduced by \citet{bengio2000neural} requires calculating a probability distribution for all words in the vocabulary (Equation~\ref{eq:nlm-bengio}). As a result, their embeddings are slow to compute and cannot be effectively learned from large datasets. 

The work of \citet{bengio2000neural} laid the foundation for \ac{NLP} researchers to explore modifications to develop computationally more efficient methods. Subsequent research in \ac{NLP} primarily focused on unsupervised learning of word representations from large corpora, with the intent of leveraging them across diverse \ac{NLP} tasks.


\subsubsection{Static Word Embedding Models} 

\paragraph{Word2Vec}

\citet{mikolov2013efficient} introduced Word2Vec, a shallow neural network designed to efficiently learn continuous word embeddings by grouping semantically similar words in the same region of the vector space. While the model may not represent data as precisely as a neural network with limited data, its efficiency improves significantly when trained on larger datasets, enabling more accurate data representation. 

Skip-gram is the simplest and most widely used model proposed by \citet{mikolov2013efficient}. The idea behind Skip-gram is to learn word representations in a manner that allows the context to be inferred from these representations. Therefore, words that co-occur in similar contexts have similar representations. Given a word, the Skip-gram model tries to predict the words that are likely to appear around it. The training objective of the model consists in maximizing the following log-probability:

\begin{equation}
    \sum_{(t, c)} \log P(t \text{ appears in the context } c) = \sum_{(t, c)} \log P(t \mid c),
\end{equation}

\noindent where (t, c) corresponds to the set of terms $t$ associated with the context $c$. The context is defined by a fixed-sized window centered on $t$, such that any word in the window but $t$ are part of the context. Given $\bm{\underline{c}}$ the embedding for the context $c$ and $\bm{\underline{t}}$ the embedding for the target word $t$, the probability of $t$ to appear in $c$ is expressed as follows:

\begin{equation}
    P(t \mid c) = \frac{\exp (\bm{\underline{t}} \cdot \bm{\underline{c}})}{\sum_{t' \in V} \exp (\bm{\underline{t'}} \cdot \bm{\underline{c}})}
\end{equation}

To maximize the similarity of words appearing in the same context and minimizing it when they occur in different contexts, it is necessary to sample pairs of words that should not appear in the same context, \textit{i.e.}, negative samples. Instead of minimizing the objective for all words in the dictionary except the context words, the model randomly selects a limited number of words and uses them to optimize the objective. 

One major limitation of Word2Vec lies in its inability to effectively handle \ac{OOV} words. Additionally, Word2Vec lacks shared representations at the subword level, which can become a challenge when dealing with morphologically rich languages such as Arabic or German.

\paragraph{FastText}  \citet{bojanowski2017enriching} mitigate these issues by introducing subword information into the model, allowing for better handling of morphologically rich languages and enhanced understanding of subword structures. While Word2Vec treats each word as an atomic unit, fastText represents words as bags of character $n$-grams. Rather than learning word embeddings, the model generates subword representations, and words are represented by the sum of their subword vectors. Formally, given $\mathcal{G}_t$ the set of all subwords of the word $t$, the target word embedding $\bm{\underline{t}}$ in the Skip-gram model can be defined as:

\begin{equation}
    \bm{\underline{t}} = \sum_{g \in \mathcal{G}_t} \bm{z}_g,
\end{equation}

where $\bm{z}_g$ is the vector of subword $g$ in the dictionary. The rest of the process is identical to the Skip-gram model.

The significant advantage of fastText over Word2Vec lies in its ability to comprehend the structure of words, including uncommon or unseen ones during training, by sharing parameters among words with similar structures. This capability is particularly valuable for languages with complex word forms.

\subsubsection{Contextualized Word Embedding Models}

% These advancements made it feasible to utilize LMs for representation learning beyond mere word sequence modeling.
% These studies have initiated the use of language models for representation learning (beyond word sequence modeling), having an important impact on the field of NLP.

The success of the Word2Vec algorithm \citep{mikolov2013efficient} has sparked immense interest in word embeddings within \ac{NLP} researchers and practitioners, leading to the development of a myriad of alternative models \citep{pennington2014glove, shazeer2016swivel, bojanowski2017enriching}. However, a significant drawback of such word embedding algorithms is that they produce static, context-independent embeddings. In other words, a word embedding remains the same across linguistic context — for instance, “jaguar” would have the same embedding whether referring to an animal or a car brand. Moreover, these approaches overlook multi-word expressions, grammatical nuances, and word-sense information, which can be crucial for handling polysemy and homonymy.

To address these limitations, \textit{contextual} word embedding models have been introduced, initiating the use of language models for representation learning beyond mere word sequence modeling. These methods use language modeling to generate contextual word embeddings that adapt to the word's usage by considering the complete context within a sentence. 

\paragraph{ELMo} \ac{ELMo} \citep{peters-etal-2018-deep} is a language model that learns contextual word embeddings by pre-training a stack of \acp{RNN} using the standard (\textit{i.e.}, autoregressive) language modeling task on a large-scale corpora. To capture the influence of both preceding and succeeding words on the target word, \ac{ELMo} adopts a \textit{bidirectional} approach, employing a forward language model to process the input sequence and predict the next word, and a backward language model that runs the input sequence in reverse, predicting the previous token given the future ones. \ac{ELMo} introduced layered word representations where each layer captures different aspects of linguistic information, ranging from syntax to semantics. The model produces a series of contextualized embeddings for each token, with one embedding per layer. They can be used in a downstream model by aggregating representations from all layers, or \textit{fine-tuned} for specific downstream tasks by adding task-specific layers on top of the model.

\paragraph{BERT} To learn a language model, the \ac{BERT} model \citep{devlin2018bert} moves away from \acp{RNN} by adopting a Transformer \citep{vaswani2017attention} architecture. A Transformer consists of parametric functions that iteratively improve the representation of a sequence of embeddings. Specifically, each layer $l$ transforms the sequence $\bm{x} = (x^l_1, \ldots, x^l_n)$ into a sequence $\bm{y} = (y^l_1, \ldots, y^l_n)$ of the same length, using \textit{self-attention}. The key idea behind self-attention is for each element in the sequence to learn to gather from other elements in the sequence (we will provide a detailed exploration of the attention mechanism in Section~\ref{sec:chapter2-plms}). Therefore, in contrast to \ac{ELMo}, bidirectionality is achieved by training a single model. \ac{BERT} produces fixed-size contextualized embeddings for each token, which can be fine-tuned for downstream tasks by adding a classifier atop of the model.

% \ac{BERT} uses two unsupervised tasks: \ac{MLM} and \ac{NSP}, where the former consists in predicting masked-out words in a sentence, while the latter involves determining whether two sentences follow each other in the original text.

\paragraph{GPT} Similar to \ac{BERT}, the \ac{GPT} model \citep{radford2018improving} leverages the Transformer architecture. However, \ac{GPT} is trained using an autoregressive language model objective. In this configuration, \ac{GPT} captures contextual information in a unidirectional manner: the sequence is processed from left to right, and each token can only attend to previous tokens to predict the next one in the sequence. \ac{GPT} is often used for tasks that require generating coherent and contextually relevant text, \textit{e.g.}, text completion, dialogue generation, and creative writing. \\

These pre-trained context-aware representations, learned on large unlabeled corpora, serve as highly effective general-purpose semantic features, significantly raising the performance bar of \ac{NLP} tasks. These studies have inspired numerous follow-up work, estabilishing the “pre-training then fine-tuning” paradigm as the prevailing learning approach. 

%enabling models to learn contextualized representations of words and phrases from large amounts of unlabeled data

\section{Pre-trained Language Models}
\label{sec:chapter2-plms}

Prior to the works of \citet{peters-etal-2018-deep}, \ac{NLP} models were commonly trained in a supervised manner \textit{from scratch} to perform specific tasks, for which labeled training data is limited. As a result, training deep neural networks on such small datasets led to overfitting, making the models sensitive to even slight shifts in the data distribution. 

While word embeddings such as Word2Vec's \citep{mikolov2013efficient} are learned from large corpora, their application in task-specific neural models is restricted to the input layer. Consequently, task-specific neural models must be built nearly from scratch, given that the majority of model parameters need to be optimized for the task at hand. This optimization process requires substantial amounts of data to attain a high-performance model.

The seminal works of \citet{peters-etal-2018-deep, devlin2018bert, radford2018improving} marked a paradigm shift by generalizing the use of \textit{tranfer learning} in \ac{NLP}, initiating a new era of \textit{Pre-trained Language Models}, also referred to as \textit{Foundation Models}. Transfer learning avoids building task-specific models from scratch by applying the knowledge acquired from training a model on one task to another task. In the context of pre-trained language models, the model is initially trained in a \textit{self-supervised} way on a large-scale corpus to learn general language representations. This \textit{pre-training} phase allows the model to capture context and a general understanding of syntax and semantics. After pre-training, the model can be \textit{fine-tuned} on specific downstream tasks (\textit{e.g.}, text classification, \ac{NER}, machine translation, and more). Fine-tuning consists in further training the pre-trained model on a smaller, labeled dataset that is specific to the downstream task. The knowledge acquired during pre-training is leveraged and tailored to the new task, often resulting in improved performance.

% In summary, pre-trained LMs leverage transfer learning by first learning general language representations in an unsupervised manner and then transferring this knowledge to specific tasks through fine-tuning. This approach has proven effective in achieving state-of-the-art results across a variety of NLP applications.

As an early attempt to leverage transfer learning in \ac{NLP}, \ac{ELMo} generates contextual representations that can be fine-tuned by adding task-specific layers. However, the need for specific architectures to solve different tasks still remains. The breakthrough in Pre-trained Language Models came with the introduction of the Transformer architecture in the work of \citet{vaswani2017attention}. Transformer-based Pre-trained Language Models \citep{devlin2018bert, radford2018improving} have demonstrated that fine-tuning the internal self-attention blocks along with a shallow feed-forward network is enough to improve the state of the art across a wide range of language tasks. This suggests that task-specfific architectures are no longer a necessity. Further, given more pre-training data, Transformer-based Pre-trained Language Models perform better with an increased model size and training compute, demonstrating superior scaling behavior \citep{kaplan2020scaling}. Hence, the Transformer has become the go-to component in the modern \ac{NLP} stack, largely replacing other architectures such as \acp{RNN}. Transformer-based Pre-trained Language Models are typically categorized into three main types: \textit{bidirectional} models utilizing only the encoder, \textit{encoder-decoder} models leveraging the entire Transformer architecture, and \textit{generative} models relying on the decoder alone.

In this section, we first explore the Transformer architecture, with a focus on its core component — the attention mechanism. Subsequently, we discuss how Transformers can be leveraged to build effective language models, ranging from bidirectional models capable of producing robust, general-purpose word representations to generative models able to create coherent and contextual relevant text, thereby laying the groundwork for \textit{Large Language Models}.

% The advent of pre-trained language models, starting with ELMo and later models like BERT and GPT, marked a paradigm shift by enabling models to learn contextualized representations of words and phrases from large amounts of unlabeled data. 

% Another interesting property of transformer architectures is their structured memory, which allows handling long-term dependencies in text, a problematic issue for recurrent networks like LSTMs. In addition, transformers support parallel processing since they are not sequential models like recurrent networks. 


\subsection{Transformer Architecture}

\acp{RNN} suffer from the vanishing/exploding gradient problem, which hinders their ability to capture long-range dependencies. In addition, the sequential processing of input in \acp{RNN} hampers efficient parallelization \citep{vaswani2017attention}. To address these limitations, \citet{vaswani2017attention} propose to remove recurrence altogether and introduce the Transformer, an architecture solely based on the \textit{self-attention} mechanism to capture global dependencies between input and output. 

We elaborate on the concept of \textit{attention} and explore a few attention mechanisms, including \textit{self-attention}. We then study the application of self-attention within the Transformer architecture. Finally, we provide details on the processing of sequential information in Transformers.

\subsubsection{Attention Mechanism} 

The concept of attention can be best explained through an analogy with human biological systems. In various problems involving language, speech, or vision, specific parts of the input are more important than others. For instance, in tasks like machine translation and summarization, only certain words in the input sequence may hold relevance for predicting the next word. An attention mechanism integrates this idea of relevance by allowing the model to dynamically \textit{pay attention} to specific portions of the input that contribute to effectively performing the task at hand.

\paragraph{Bahdanau's Attention Mechanism} The earliest use of attention was proposed by \citet{bahdanau2014neural} for a \textit{sequence-to-sequence} modeling task. A sequence-to-sequence task involves mapping a sequence of $n$ input vectors to a sequence of $m$ target vectors, where $m$ is unknown apriori. A sequence-to-sequence model \citep{sutskever2014sequence} consists of an \textit{encoder-decoder} architecture, where the encoder encodes an input sequence $(\bm{x}_1, \ldots, \bm{x}_n)$ into a sequence of fixed-size vectors $(\bm{h}_1, \ldots, \bm{h}_n)$. The decoder is then fed the fixed-size vector $\bm{h}_n$ and generates an output sequence $(\bm{y}_1, \ldots, \bm{y}_m)$.

In a traditional encoder-decoder architecture (usually based on \acp{RNN}), the encoder must compress all input information into a single fixed-size vector $\bm{h}_n$ that is fed to the decoder. However, encoding a variable-length input into a fixed-size vector squashes the information of the input sequence, irrespective of its length, causing the performance to deteriorate rapidly as the input sequence length increases \citep{cho2014properties}. In addition, in sequence-to-sequence tasks, each output token is expected to be more influenced by specific parts of the input sequence. However, the decoder lacks any mechanism to selectively focus on relevant input tokens.

To alleviate these challenges, \citet{bahdanau2014neural} introduce the attention mechanism, a principle that allows the decoder to access the entire encoded input sequence $(\bm{h}_1, \ldots, \bm{h}_n)$ and dynamically \textit{attend to} information deemed relevant to generate the next output token. The key idea behind attention is to introduce attention weights $\bm{\alpha}$ over the input sequence, prioritizing positions with relevant information for the generation of the next output token. These attention weights determine the context vector $c$, which is then fed to the decoder. At each decoding position $j$, the context vector $\bm{c}_j$ is updated as a weighted sum of all encoder hidden states $\{\bm{h}_i\}_{i=1, \ldots, n}$ and their corresponding attention weights $\{\alpha_{ij}\}_{i=1, \ldots, n}$: 

\begin{equation}
    \bm{c}_j = \sum_{i=1}^n \alpha_{ij} \bm{h}_i.
\end{equation}

\noindent The introduction of the context vector serves as a mechanism for the decoder to access the entire input sequence and selectively attend to relevant posititons within it. It acts as a representation of the input sequence and is re-computed for each output token. This addition enhances the quality of the output by achieving better alignment.

The attention weights $\alpha$ determine the relevance between each encoder hidden state and each decoder hidden state. Each attention weight $\alpha_{ij}$ is computed as a function of the encoder hidden state $\bm{h}_i$ and the decoder hidden state $\bm{s}_{j-1}$, defined as follows:

\begin{equation}
    \alpha_{ij} = p(e_{ij}) = a(\bm{s}_{j-1}, \bm{h}_i),
\end{equation}

\noindent where $a$ is an alignment function implemented as a feed-forward network, and $p$ is a distribution function. The alignment score $a(\bm{s}_{j-1}, \bm{h}_i)$ defines how relevant $\bm{h}_i$ is for $\bm{s}_{j-1}$.

\paragraph{Generalized Attention} The \textit{generalized attention} model \citep{chaudhari2021attentive} is an extension of the attention mechanism of \citet{bahdanau2014neural} that allows for more flexibility and adaptability in capturing dependencies between different parts of the input and output sequences. While the original attention mechanism focuses on aligning parts of the input sequence with the current position in the output sequence, the generalized attention model introduces parameters and mechanisms to customize and control the attention process. In the generalized attention model, attention weights are not solely determined by the relevance between the current decoder hidden state and the encoder hidden states. Instead, the model introduces learnable parameters and scoring functions that can be adjusted to capture different types of relationships. This allows the attention mechanism to consider various aspects, such as semantic similarity, positional information, or other task-specific factors.

A generalized attention model $A$ is characterized by a set of key-value pairs $(\bm{K}, \bm{V})$ and a query $\bm{q}$ such that:

\begin{equation}
    A(\bm{q}, \bm{K}, \bm{V}) = \sum_i p(a(\bm{q}, \bm{k}_i)) \cdot \bm{v}_i
\end{equation}

\noindent The alignment function $a$ determines how keys and queries are combined (\textit{e.g.}, dot product or cosine similarity), while the distribution function $p$ ensures that the attention weights lie between 0 and 1 and are normalized to sum to 1 (\textit{e.g.}, logistic sigmoid or softmax function).

The attention mechanism of \citet{bahdanau2014neural} can be seen as a special case of generalized attention where $\bm{K} = \bm{V} = \{\bm{h}_i\}_{i=1, \ldots, n}$ and $\bm{q} = \bm{s}_{j-1}$. 
% Then, $e = a(\bm{K}, \bm{q})$ and $\alpha = p(e)$.


\paragraph{Self-Attention} One common form of the generalized attention model is \textit{self-attention}, or scaled dot-product attention, introduced by \citet{vaswani2017attention} in the Transformer architecture. In this mechanism, the alignment function is defined by a scaled dot product, while the distribution function corresponds to the softmax. The scaled dot product between query and key is passed through a softmax function to obtain the final attention weights. Self-attention is defined as follows:

\begin{equation}
    A(\bm{q}, \bm{K}, \bm{V}) = \sum_i \textrm{softmax}\left(\frac{\bm{q}^{\top} \bm{k}_i}{\sqrt{d_k}}\right) \cdot \bm{v}_i,
\end{equation}

\noindent where $d_k$ is the dimensionality of the key vectors.

The alignment score $\frac{\bm{q}^{\top} \bm{k}_i}{\sqrt{d_k}}$ indicate how to weigh the value $\bm{v}_i$ based on the query vector $\bm{q}$. The more similar a key vector $\bm{k}_i$ is to $\bm{q}$, the more important is the corresponding value vector $\bm{v}_j$ for the output vector. 

% attention scores are computed by taking the dot product of the query (decoder hidden state) and key (encoder hidden state) vectors

\paragraph{Muti-Head Self-Attention} Rather than computing attention in a single step, \citet{vaswani2017attention} propose to decompose the self-attention operation in $h$ heads. The feature dimension $d$ is divided into $h$ fixed-size segments. Self-attention is then computed over each segment in parallel, using different linear transformations of the same input. The outputs of each head are then concatenated to form the complete attention output. \textit{Multi-head} self-attention is then expressed as:

\begin{flalign}
    \text{MultiHeadAttention}(\bm{q}, \bm{K}, \bm{V}) &= 
    \begin{bmatrix}
        \mathrm{head}_1(\bm{q}, \bm{K}, \bm{V}) \\
        \mathrm{head}_2(\bm{q}, \bm{K}, \bm{V}) \\
        \ldots \\
        \mathrm{head}_h(\bm{q}, \bm{K}, \bm{V})
    \end{bmatrix}
    \bm{W}_o \\
    \mathrm{where} \quad \mathrm{head}_i(\bm{q}, \bm{K}, \bm{V}) &= \text{Self-Attention}\left(\bm{qW}^{(i)}_q, \bm{KW}^{(i)}_k, \bm{VW}^{(i)}_v\right).
    \end{flalign}
    

% The dimension of each head is a subspace of the model's representation space, \textit{i.e.}, $d_k = d_v = \frac{d}{h}$. 
\noindent For each head $i$, query, key and value matrices are transformed into sub-queries, sub-keys, and sub-values using the learned projection matrices $\bm{W}_q$, $\bm{W}_k$ and $\bm{W}_v$. The matrix $\bm{W}_o$ then projects the concatenation of head-attentions back into the original $d$-dimensional representation space.

The motivation behind multi-head attention is to ensure different views of the same sequence and enable parallelized computation of attention across different representation subspaces

 
\subsubsection{Self-Attention in Transformers}

\paragraph{Transformer Architecture} A Transformer \citep{vaswani2017attention} is an encoder-decoder architecture that defines a conditional distribution of target vectors $(\bm{y}_1, \ldots, \bm{y}_m)$ given an input sequence $(\bm{x}_1, \ldots, \bm{x}_n)$. The encoder encodes the input sequence $(\bm{x}_1, \ldots, \bm{x}_n)$ into a contextualized sequence of hidden states $(\overline{\bm{x}}_1, \ldots, \overline{\bm{x}}_n)$. The decoder then uses these hidden states to condition the probability distribution of the target vector sequence $(\bm{y}_1, \ldots, \bm{y}_m)$. By Bayes' rule, this distribution can be factorized to a product of conditional probability distribution of the target vector $\bm{y}_i$ given the encoded hidden states $(\overline{\bm{x}}_1, \ldots, \overline{\bm{x}}_n)$ and all previous target vectors $(\bm{y}_0, \ldots, \bm{y}_{i-1})$. Formally:

\begin{equation}
    p_{\theta} \bigl( \bm{y}_1, \ldots, \bm{y}_m \mid \overline{\bm{x}}_1, \ldots, \overline{\bm{x}}_n \bigr) = \prod_{i=1}^{m} p_{\theta}\bigl(\bm{y}_i |\bm{y}_0, \ldots, \bm{y}_{i-1}; \overline{\bm{x}}_1, \ldots, \overline{\bm{x}}_n\bigr).
\end{equation}

The input and target sequences $(\bm{x}_1, \ldots, \bm{x}_n)$ and $(\bm{y}_1, \ldots, \bm{y}_m)$ are embedded and fed to the encoder and the decoder. Both encoder and decoder are composed by stacking a series of Transformer layers on top of each other. Each Transformer layer is characterized by a multi-head self-attention module and two position-wise feed-forward networks. The input of each encoder layer corresponds to the previous layer's output. To help the model train faster and more accurately, a residual connection \citep{he2016deep} is added to all sublayers, followed by layer normalization. In the decoder, the self-attention module is masked to enforce \textit{unidirectional} self-attention, preventing tokens from attending to future tokens. Furthermore, an additional sublayer, the \textit{cross-attention} module, is inserted between the self-attention module and the feed-forward networks. Cross-attention takes as inputs both the encoder's outputs and the outputs of the previous decoder layer. Finally, the outputs of the final decoder layer are fed to a feed-forward network to obtain, for each target position, a probability distribution over the whole vocabulary.

% Using the full set of attention scores $A(\bm{Q}^{(l)}, \bm{K}^{(l)}, \bm{V}^{(l)})$, token representations $(\bm{x}^{(l+1)}_1, \ldots, \bm{x}^{(l+1)}_n)$ are computed by building the corresponding weighted sum over every other token, \textit{i.e.},

% \begin{equation}
%     \bm{x}^{(l+1)}_i = \bm{x}^{(l)}_i + \sum_j \textrm{softmax} \left(\dfrac{\bm{q}^{{(l)_i}^\top} \bm{k}^{(l)}_j}{\sqrt{d_k}}\right) \cdot \bm{v}^{(l)}_j.
% \end{equation}

% Authors demonstrated that Transformer architecture achieved significant parallel processing, shorter training time and higher accuracy for Machine Translation without any recurrent component

\paragraph{Self-Attention in the Encoder} In the encoder, self-attention is used to map the input sequence $(\bm{x}_1, \ldots, \bm{x}_n)$ to a sequence of context-dependent vectors $(\overline{\bm{x}}_1, \ldots, \overline{\bm{x}}_n)$. Each attention layer builds the queries, keys and values from the outputs of the previous encoder layer, and uses \textit{bidirectional} self-attention to put each input token in relation with all input tokens in the sequence. Given $(\bm{x}^{(l)}_1, \ldots, \bm{x}^{(l)}_n)$ the input sequence to the $l$-th encoder layer, the outputs $(\bm{x}^{(l+1)}_1, \ldots, \bm{x}^{(l+1)}_n)$ constructed using bidirectional self-attention can be expressed as:

\begin{equation}
    \bm{x}^{(l+1)}_i = \bm{x}^{(l)}_i + \textrm{MultiHeadAttention}\left(\bm{q}^{(l)}_i, \bm{K}^{(l)}, \bm{V}^{(l)}\right), \qquad \forall \quad 1 \leq i \leq n,
    % \bm{x}^{(l+1)}_i = \bm{x}^{(l)}_i + \sum_{j=1}^{n} \textrm{softmax} \left(\dfrac{\bm{q}_i^{{(l)}^\top} \bm{k}^{(l)}_j}{\sqrt{d_k}}\right) \cdot \bm{v}^{(l)}_j, \qquad \forall \quad 1 \leq i \leq n.
\end{equation}

\noindent where $\bm{q}^{(l)}_i$, $\bm{K}^{(l)}$, and $\bm{V}^{(l)}_i$ are the query vector, key, and value matrices obtained by projecting $\bm{x}^{(l)}_i$ using three weight matrices $\bm{W}^{(l)}_Q \in \mathbb{R}^{n \times d_q}$, $\bm{W}^{(l)}_K \in \mathbb{R}^{n \times d_k}$ and $\bm{W}^{(l)}_V \in \mathbb{R}^{n \times d_v}$ (with $d_q = d_k = d$).

Each encoder layer builds a contextualized representation of its input sequence, and the following layer further refines this context-dependent representation. Compared to \acp{RNN}, bidirectional self-attention reduces the amount of computation steps that information needs to flow from one point to another. Therefore, information loss is reduced, making long-range dependencies more easily learnable. 

\paragraph{Self-Attention in the Decoder} The decoder models the distribution of a target sequence $(\bm{y}_1, \ldots, \bm{y}_m)$ conditioned on the input sequence $(\bm{x}_1, \ldots, \bm{x}_n)$. Each decoder layer contains three sublayers: \textit{decoder self-attention}, \textit{cross-attention}, and a module made of two position-wise feed-forward networks. The final decoder layer is followed by a feed-forward network which produces a probability distribution over the whole vocabulary. 

The decoder self-attention layer conditions each decoder output vector on all previous decoder input vectors. As opposed to the encoder, self-attention in the decoder is masked to ensure that each vector attends only to the previous positions, making predictions depend only on the tokens that have already been generated. Given $\bm{y}^{(l)}_i$ a target vector fed to the $l$-th decoder layer, the output vector $\bm{y}^{\prime(l)}_i$ generated by unidirectional self-attention is defined as follows:

\begin{equation}
    \bm{y}^{\prime(l)}_i = \bm{y}^{(l)}_i + \textrm{MultiHeadAttention}\left(\bm{q}^{(l)}_i, \bm{K}^{(l)}_{0:i}, \bm{V}^{(l)}_{0:i}\right), \qquad \forall \quad 1 \leq i \leq m,
\end{equation}

\noindent where $\bm{q}^{(l)}_i$, $\bm{K}^{(l)}_{0:i}$, and $\bm{V}^{(l)}_{0:i}$ are projections of $(\bm{y}^{(l)}_0, \ldots, \bm{y}^{(l)}_i)$.

To condition the probability distribution of the next target vector on the encoder's input, \textit{cross-attention} is applied to put each of the target vectors $\bm{y}'^{(l)}_i$ into relation with all contextualized input vectors $(\overline{\bm{x}}_1, \ldots, \overline{\bm{x}}_n)$. The output $\bm{y}^{(l+1)}_i$ built using cross-attention is expressed as:

\begin{equation}
    \bm{y}^{(l+1)}_i = \bm{y}^{\prime(l)}_i + \textrm{MultiHeadAttention}\left(\bm{q}'^{(l)}_i, \overline{\bm{K}}^{(l)}, \overline{\bm{V}}^{(l)}\right), \qquad \forall \quad 1 \leq i \leq m.
\end{equation}

\noindent While $\bm{q}^{\prime(l)}_i$ is computed from the output $\bm{y}^{\prime(l)}_i$ of the unidirectional self-attention module, $\overline{\bm{K}}^{(l)}$ and $\overline{\bm{V}}^{(l)}$ are built from the contextualized input sequence $(\overline{\bm{x}}_1, \ldots, \overline{\bm{x}}_n)$. Cross-attention ensures that, the more similar a decoder input representation is to an encoder input representation, the more does the input representation influence the decoder output representation.



\subsubsection{Sequential Information in Transformers}

The position and order of words form the semantics of a sentence and thus are a fundamental component of any language. By processing sequences token by token in a sequential manner, \acp{RNN} inherently integrate the order of the sequence in their backbone. Unlike \acp{RNN}, Transformers simultaneously process each token in the sequence, hence losing any sense of position and order. Consequently, there is a need to explicitly incorporate the order of tokens into the Transformer.

\paragraph{Positional Encodings} There are many reasons why assigning a single number (\textit{e.g.}, the index value) to each time step is not used to represent a token's position in Transformer models. For long sequences, the indices can grow large in magnitude. If the index value is normalized to lie between 0 and 1, it can create problems for variable length sequences, as they would be normalized differently. 

A satisfactory positional encoding method must be deterministic, produce a unique encoding at each time step, generalize to longer sequences, and ensure that distance between two time steps are consistent across sequences with different lengths. Instead of integrating this encoding into the model itself, the dominant approach for preserving information about the sequence order is to equip each token with information about its position in the sequence. These inputs are called positional encodings (or embeddings) and can either be learned or fixed a priori. % In other words, we enhance the model’s input to inject the order of words.

Absolute position encodings encode the absolute position of a token within a sequence, meaning that each token is assigned a fixed vector based on its position in the sequence. \citet{vaswani2017attention} propose a simple scheme for fixed absolute positional encodings, where each position is mapped to a vector. Given $t$ a position in an input sequence, $d$ the encoding dimension, and $k \in \{1, \ldots, d/2\}$, the function $f: \mathbb{N} \rightarrow \mathbb{R}^d$ produces the positional encoding $\bm{p}_t$ as follows:

\begin{equation}
    p_{t,i} = f(t)_i = 
\begin{cases}
    \sin(\omega_k t), & \text{if } i=2k\\
    \cos(\omega_k t),              & \text{otherwise},
\end{cases}
\end{equation}

where $\omega_k =\dfrac{1}{10000^{2k/d}}$. This encoding scheme is called \textit{sinusoidal} positional encoding.

The positional embedding matrix $\bm{P} \in \mathbb{R}^{n \times d}$, obtained by encoding every position $i \in {1, \ldots, n}$, is added to the input representation matrix $\bm{X} \in \mathbb{R}^{n \times d}$ and fed to the Transformer.

% Given $t \in \{1, \ldots, n\}$ a position in the input sequence and $k \in \{0,1, \cdots, d/2-1\}$ the index of an element in the vector space, the positional encoding is defined as a function of type $f:\mathbb {R} \to \mathbb {R} ^{d}$:

% Transformers use a smart positional encoding scheme, where each position/index is mapped to a vector. Hence, the output of the positional encoding layer is a matrix, where each row of the matrix represents an encoded object of the sequence summed with its positional information.

\paragraph{Relative Positional Biases} Besides capturing absolute positional information, sinusoidal positional encoding also allows the model to learn to attend by relative positions. This is because, for any offset $\delta$, the positional encoding at position $i + \delta$ can be represented by a linear projection of the encoding at position $i$. Formally, any pair of $(p_{i, 2k}, p_{i, 2k+1})$ can be linearly projected to $(p_{i + \delta, 2k}, p_{i + \delta, 2k+1})$ for any offset $\delta$:

\begin{equation}
    \begin{bmatrix}
        \cos(\delta \omega_k)  & \sin(\delta \omega_k) \\
        -\sin(\delta \omega_k) & \cos(\delta \omega_k)
    \end{bmatrix}
    \begin{bmatrix}
        p_{t, 2k}   \\
        p_{t, 2k+1}
    \end{bmatrix}
    = \begin{bmatrix}
        p_{i + \delta, 2k}   \\
        p_{i + \delta, 2k+1}.
    \end{bmatrix}
\end{equation}

Although absolute positional encodings show satisfactory performance, they still face limitations. First, relying on absolute positional information imposes a constraint on the number of tokens a model can handle. For instance, if a language model can only encode up to 1,024 positions, any sequence longer than 1,024 tokens cannot be processed by the model. Besides, absolute positional encodings do not generalize well to sequences of unseen lengths. Relative positional encoding address these issues by using a different vector for each pair of tokens, based on their relative distance \citep{shaw2018self, huang2018music, ke2020rethinking}.  \citet{shaw2018self} are the first to leverage pairwise distances to create positional encodings. During attention calculation, relative positional information is added on the fly to keys and values. Given a query $\bm{q}_i$ computed from token $\bm{x}_i$ and a key $\bm{k}_j$ calculated from token $\bm{x}_j$, the attention score between tokens $i$ and $j$ is reformulated as follows:

\begin{equation}
    \alpha_{ij} = \mathrm{Softmax}\left(\frac{\bm{q}_i (\bm{k}_j + \bm{r}^K_{ij})^{\top}}{\sqrt{d_k}}\right).
\end{equation}

Let $\bm{v}_j$ be the value vector corresponding to token $j$. Relative positional information is supplied again as a sub-component of the values matrix:

\begin{equation}
    \bm{y}_i = \sum_{j=1}^n \alpha_{ij} (\bm{v}_j + \bm{r}^V_{ij}).
\end{equation}

Relative positional encodings offer the advantage of generalizing to sequences of unseen lengths. Theoretically, they encode only the relative pairwise distance between two tokens, allowing adaptability to various sequence lengths.

\subsection{Bidirectional Models}

\textit{Bidirectional} Pre-trained Language Models refer to models that employ the Transformer encoder. These models are pre-trained on large corpora and learn deep contextualized representations of words and phrases by jointly conditioning on left and right context in all layers. Bidirectional Transformer-based models are effective for capturing dependencies and contextual information in both directions, making them suitable for various natural language understanding tasks including sentiment analysis, \ac{NER}, text classification, and more. The exploration of bidirectional Transformer-based Pre-trained Language Models began with \ac{BERT}, introduced by \citet{devlin2018bert}, and has led to the development of a myriad of variants.

\subsubsection{BERT}

\ac{BERT} marked a paradigm shift in the construction of word representations.  
Prior to \ac{BERT}, language models typically processed text in a unidirectional manner. Bidirectional models, such as those obtained using stacked bidirectional \ac{LSTM} layers \citep{peters-etal-2018-deep}, also processed the sequence in a fixed order, capturing information from both directions but not simultaneously. 
Using bidirectional self-attention, \ac{BERT} extended the concept of bidirectional processing. For each token in the sequence, the model can simultaneously consider both preceding and following tokens, enabling a more comprehensive understanding of context.

In \ac{NLP}, some tasks (\textit{e.g.}, sentiment analysis) take a single sequence as input, while others (\textit{e.g.}, natural language inference) require a pair of sequences. \ac{BERT} can represent both single text and text pairs. In both cases, a classification token \texttt{[CLS]} is prepended to the input sequence and is used to represent the whole sequence. In addition, a special separation token \texttt{[SEP]}, indicating the end of the sequence, is added to the end of the sequence. In the case of text pairs, an extra \texttt{[SEP]} token is added between the pair to separate the texts. Additionally, segment embeddings are used and trained to distinguish text pairs. To encode positions, \ac{BERT} departs from the fixed positional encodings used in the original Transformer and employs learnable positional embeddings. To sum up, text is tokenized into subwords using WordPiece \citep{wu2016google}, and special tokens are added accordingly to the aforementioned scenarios. The final input embeddings of \ac{BERT} are the sum of the token embeddings, positional embeddings, and segment embeddings. The input embeddings $(\bm{x}_1, \ldots, \bm{x}_n)$ are passed through a Transformer encoder that generates a sequence of contextualized token representations $(\overline{\bm{x}}_1, \ldots, \overline{\bm{x}}_n)$.

\paragraph{Pre-training BERT}

\ac{BERT} is trained on a large-scale dataset \citep{zhu2015aligning}, as a language model that operates at both the word-level and the sentence-level. The training involves two unsupervised tasks: \ac{MLM} and \ac{NSP}. 

The \ac{MLM} task consists in randomly masking out tokens and using all remaining tokens to recover the masked-out tokens in a self-supervised fashion. Instead of following the same probability distribution as causal language models (Equation~\ref{equation:causal-distribution}), \ac{BERT} uses the following approximation:

\begin{equation}
    P(\bm{w}) \propto \prod_{w \in C} P_{\theta} \left(w \mid \tilde{\bm{w}}\right),
\end{equation}

\noindent where $C$ is a random set of tokens, with 15\% of tokens selected to be in $C$, and $\tilde{\bm{w}}$ is the input sequence $\bm{w}$ corrupted as follows:

\begin{equation}
    \tilde{w} = 
\begin{cases}
    w_t,               & \text{if } w_t \notin C\\
    \text{mask token}       & \text{if } w_t \in C, \text{ with probability 80\%} \\
    \text{random token}       & \text{if } w_t \in C, \text{ with probability 10\%} \\
    w_t       & \text{if } w_t \in C, \text{ with probability 10\%.} \\
\end{cases}
\end{equation}

\noindent Because the mask token is never used during fine-tuning, a discrepancy between pre-training and fine-tuning can occur. For 10\% of 15\% time, the masked token is replaced with a random token. The cross-entropy loss between the masked tokens and their predictions is minimized during pre-training. The primary benefit of the \ac{MLM} task, in contrast to a causal language model, is that token representations are parameterized by the whole sequence.

While \ac{MLM} effectively captures bidirectional context to represent words, it does not explicitly capture the logical correlation between pairs of texts. To address this, the \ac{NSP} task is introduced. This task involves determining whether two sentences follow each other and helps in modeling the relationship between texts. The training dataset is constructed such that half of the pairs are made of consecutive sequences, while for the other half the second sequence is randomly sampled from the corpus. Given a pair of sequences $(s_1, s_2)$, a binary single-layer feed-forward network classifier is trained to determine whether $s_2$ follows $s_1$ in the corpus. The classifier is fed with the \ac{BERT} representation of the \texttt{[CLS]} token, which encodes both sequences, and outputs the probability that the sequences are successive sentences. 


\paragraph{Fine-tuning BERT}

% The outcome of this pre-training process is a language model able to comprehend context, semantics, and relationships between words and sentences. 
The knowledge gained during pre-training can then transferred to various downstream tasks through fine-tuning. The contextualized token representations obtained by the pre-trained \ac{BERT} are fed to a shallow feed-forward network built over the last encoder layer. In this layer, predictions are generated for either individual tokens or the entire sequence. While the parameters of the pre-trained encoder are reused and fine-tuned for the task, the additional layer is initialized randomly and trained from scratch. This layer can output predictions for individual tokens or the entire sequence, rendering the model suitable for tasks involving token classification and sequence classification, respectively. While all the parameters of the pre-trained encoder are re-used and adjusted to the task, the additional layer is randomly initialized and trained from scratch. \\

The bidirectional capability of \ac{BERT} addressed a crucial limitation in previous models, especially for tasks requiring a deep understanding of context and relationships between words. \ac{BERT} demonstrated remarkable performance across various natural language understanding benchmarks (\textit{e.g.}, sentiment analysis, question answering, text classification), showcasing the potential of bidirectional Transformer-based PLMs.
 
\subsubsection{BERT Variants}

The success of \ac{BERT} has expanded exploration of bidirectional Transformer-based Pre-trained Language Models, with researchers and practitioners building upon the foundation laid by the model. 

Some extensions of \ac{BERT} have introduced different optimization objectives to further improve the quality of the contextual representations. To enhance the robustness and generalization capability of the model, \ac{RoBERTa} \citep{liu2019roberta} removes the \ac{NSP} objective of \ac{BERT}, uses dynamic masking, and employs a bigger dataset. \ac{ALBERT} \citep{lan2019albert} introduces a variant of \ac{NSP} where negative examples correspond to two consecutive segments with their order reversed. This modification of the \ac{NSP} objective, emphasizing coherence, makes the task more challenging, thereby improving the robustness and generalization of the model. SpanBERT \citep{joshi2020spanbert} modify the masking strategy in \ac{MLM} by masking contiguous random spans, rather than random tokens. Pre-trained using three type of language modeling tasks (unidirectional, bidirectional, and sequence-to-sequence prediction), UniLM \citep{dong2019unified} can be fine-tuned on both natural language understanding and generation tasks. 

Furthermore, several adaptations of \ac{BERT}, pre-trained on specific datasets tailored to particular languages or domains, have been proposed. SciBERT \citep{beltagy2019scibert} has been pre-trained on scientific texts and is specialized for tasks in the scientific research domain. Similarly, BioBERT \citep{lee2020biobert} has been designed for biomedical texts. CamemBERT \citep{martin2019camembert}, on the other hand, is a French version of \ac{BERT}, pre-trained on French texts, and adapted for French language understanding tasks. For usage across different languages, multilingual pre-trained models such as mBERT \citep{devlin2018bert}, XLM \citep{lample2019cross}, and XLM-RoBERTa \citep{conneau2019unsupervised} have been proposed. 

To enable the use of models in resource-constrained scenarios, and given the quadratic complexity of \ac{BERT} with respect to the sequence length in both memory and time, researchers have explored approaches to reduce the number of parameters. \ac{ALBERT} achieves efficiency by sharing parameters across layers and reducing the rank of the embedding matrix. DistilBERT \citep{sanh2019distilbert} simplifies \ac{BERT} by using parameter reduction techniques such as factorized embedding parameterization and cross-layer parameter sharing. TinyBERT \citep{jiao2019tinybert} further simplifies the model for increased efficiency. This research is part of a broader effort to make Transformers more computationally efficient, which will be further discussed in Section~\ref{sec:chapter2-long-range}.

\subsection{Generative Models} 

In contrast to bidirectional Pre-trained Language Models that focus on predicting labels for given inputs, the goal of \textit{generative} Pre-trained Language Models is to produce new text of arbitrary length that resembles human language. These models employ the Transformer decoder to generate content in an autoregressive fashion. Given their ability to capture contextual information and generate diverse and contextually fitting text, generative Pre-trained Language Models play a pivotal role in improving the quality of text generation for natural language generation tasks (\textit{e.g.}, machine translation, summarization, text completion). Generative Transformer-based Pre-trained Language Models can either use the entire Transformer architecture (\textit{encoder-decoder} models) or solely the decoder component (\textit{decoder-only} models).

\subsubsection{Training and Inference Framework}

Training and inference for generative models differ from those of bidirectional models in several aspects.

\paragraph{Training} While bidirectional models are typically trained for specific pre-defined tasks, generative language models are trained to predict the next token in a sequence, given the context of the preceding tokens (\textit{autoregressive language modeling}). However, compounding errors might occur from incorrect predictions, leading the decoder to potentially drift too far away from the target. To mitigate this issue and guide the training process, the \textit{teacher forcing} strategy is used: to predict the next token, the decoder is fed with the ground-truth target tokens, rather than its own predictions from the previous step. Furthermore, using ground-truth target tokens as inputs accelerates convergence, as stronger gradient signals are provided during backpropagation.

\paragraph{Inference} However, teacher forcing is not applicable at inference time. Instead, the sequential generation process requires the model to be fed its own predictions from the previous step as inputs to generate the next token. Different strategies can be used to determine how the model predicts the next element of the sequence. 

\textit{Greedy search} consists in selecting the token with the highest probability at each step. While simple and computationally efficient, it misses high probabilities that can be found in posterior tokens. 

To reduce this risk, \textit{beam search} extends greedy search by maintaining a fixed number $K$  of sequences with the highest probabilities. At each step, it picks the $K$ best sequences so far based on their combined probabilities. Finally, the sequence with the highest probability is selected as the output sequence.

To ensure that the less probable tokens should not have any chance of being selected, \textit{top-k sampling} \citep{fan2018hierarchical} filters the $k$ most likely next tokens and redistributes the probability mass among those $k$ tokens only. 

\textit{Nucleus sampling} chooses from the smallest possible set of tokens whose
cumulative probability exceeds a certain threshold $p$. The probability mass
is then redistributed among this set of tokens. Nucleus sampling balances randomness and predictability better than traditional sampling.

\subsubsection{Encoder-decoder Models}

Encoder-decoder Transformer-based Pre-trained Language Models use the original Transformer architecture, consisting of both an encoder and a decoder. The encoder processes an input sequence, producing contextualized representations of each token. The decoder attends to these representations and generates the output tokens one at a time, considering the context provided by the encoder. This architecture is particularly suited for sequence-to-sequence tasks, where the goal is to transform a source sequence into a corresponding target sequence. The encoder-decoder setup is commonly used in sequence-to-sequence tasks like machine translation and text summarization.

\paragraph{BART} \ac{BART} \citep{lewis2019bart} is pre-trained with a denoising objective, where the model is trained to reconstruct the original sequence from a corrupted version. \ac{BART} extends the \ac{MLM} approach by adding more perturbations: replacing text spans with a single mask token (\textit{text infilling}), permuting sentences, deleting or replacing tokens, and rotating documents. The corrupted sequence is encoded using the bidirectional encoder, and the decoder is trained to reconstruct the original sequence. When fine-tuned, \ac{BART} shows remarkable results for natural language generation tasks such as text summarization, machine translation, question answering. Additionally, the model also works well for natural language understanding tasks, \textit{e.g.}, \ac{NER}, \ac{NLI}, and coreference resolution. Inspired by the success of \ac{BART}, \citet{liu2020multilingual} introduce mBART, a multilingual version of \ac{BART} pre-trained on large-scale monolingual corpora in many languages. mBART can be fine-tuned for any of the language pairs, whether in supervised or unsupervised settings, without necessitating task-specific or language-specific adjustments or initialization methods.

\paragraph{Pegasus} \citep{zhang2020pegasus} is specifically tailored for abstractive text summarization. It is trained using the \ac{MLM} strategy coupled with the \textit{Gap-Sentences Generation} task, a novel pre-training approach intentionally similar to summarization. The Gap-Sentences Generation task consists in masking whole sentences important to an input sequence and generating them together as one output sequence using the remaining sentences, similar to an extractive summary. 

\paragraph{T5} \ac{T5} \citep{raffel2020exploring} converts all \ac{NLP} tasks into a sequence-to-sequence problem: for any task, the input of the encoder is a task-specific prefix (\textit{e.g.}, \say{Summarize:}) followed by the task's input (\textit{e.g.}, a sequence of tokens from an article), and the decoder predicts the task's output (\textit{e.g.}, a sequence of tokens summarizing the input article). The pre-training includes a mixture of both supervised and unsupervised tasks. Supervised pre-training is conducted on downstream tasks (translation, question answering, and more). Unsupervised pre-training uses corrupted tokens, by randomly removing 15\% of the tokens and replacing them with individual sentinel tokens. Given the corrupted sequence encoded by the encoder and the original sequence fed to the decoder, \ac{T5} has to reconstruct the dropped out tokens. Casting all \ac{NLP} tasks into the same sequence-to-sequence problem allows for the use of the same model, loss function, and hyperparameters across a diverse set of tasks. 

\subsubsection{Decoder-only Models}

The use of decoder-only Transformer architectures in Pre-trained Language Models has seen a recent surge, with several groundbreaking models \citep{radford2018improving, brown2020language, ouyang2022training, touvron2023llama} emerging. Decoder-only Pre-trained Language Models, also referred to as \textit{autoregressive} or \textit{causal} language models, remove the Transformer encoder and cross-attention layers, leading to a substantial reduction in model size. Each source sequence is concatenated with the corresponding target sequence to form a single input sequence, which is then used to train a language model. 

Among the most revolutionaly generative models of the decade is the series of \ac{GPT} models introduced by \citep{radford2018improving}. \ac{GPT} is the first autoregressive language model that uses a Transformer decoder as its backbone. It learns to predict the next word in a sequence using an autoregressive language modeling. Suppose $\bm{w} = \{w_1, \ldots, w_n\}$ an unsupervised corpus of tokens, $k$ the size of the context window, and $\theta$ the parameters of the decoder. 
\ac{GPT} uses the following approximation:

\begin{equation}
    P(\bm{w}) \propto \prod_{i=1}^n P_{\theta}(w_i \mid w_{i-k}, \ldots, w_{i-1}).
\end{equation}

\noindent The pre-training objective can therefore be expressed as follows:

\begin{equation}
    L_1(\bm{w}) = \sum_{i=1}^n \log P_{\theta}(w_i \mid w_{i-k}, \ldots, w_{i-1}).
\end{equation}

\noindent During fine-tuning, the parameters are adjusted to the supervised downstream task. Given a labeled dataset $\mathcal{C}$, where each instance consists of a sequence of input tokens $\bm{w} = (w_1, \ldots, w_m)$ and its label $y$, the following objective is maximized:

\begin{equation}
    L_2(\mathcal{C}) = \sum_{(\bm{w}, y) \in \mathcal{C}} \log P(y \mid w_1, \ldots, w_m) + \lambda L_1(\mathcal{C}),
\end{equation}

\noindent where $\lambda$ is the weight given to the auxiliary language modeling objective.

\noindent \ac{GPT} surpassed state-of-the-art \acp{NLP} models that were trained in a supervised fashion with task-specific architectures. In addition, it improved zero-shot performance in various \ac{NLP} tasks such as question answering, schema resolution, and sentiment analysis. \ac{GPT} established the core architecture for the \ac{GPT}-series models and laid down the fundamental principle to model natural language text, \textit{i.e.}, predicting the next word.

To learn an even stronger language model, \citet{radford2019language} propose \ac{GPT}-2, a much larger version of \ac{GPT} that increases the number of parameters from 100 million to 1.5 billion. Similar to T5 \citep{raffel2020exploring}, \ac{GPT}-2 seeks to perform tasks via self-supervised language modeling, without explicit fine-tuning with labeled data. To achieve this, \citet{radford2019language} introduce \textit{task conditioning}, a probabilistic form for multi-task learning, which consists in predicting the output based on the input and task information, \textit{i.e.}, $P(output \mid input, task)$. Task conditioning is performed by providing examples of natural language instructions to perform a task, \textit{e.g.}, for English to French translation, the model is given an English sentence followed by \say{French: }. Therefore, input to \ac{GPT}-2 is given in a format which expects the model to understand the nature of the task. Trained on a sufficiently extensive and large dataset \citep{radford2019language}, \ac{GPT}-2 achieved state-of-the-art performance on language modeling benchmarks \citep{marcus1993building, chelba2013one, merity2016pointer} in zero-shot scenarios. On downstream tasks such as question answering, summarization, and translation, \ac{GPT}-2 demonstrates the ability to learn these tasks directly from raw text, without relying on task-specific training data. The model's versatility in handling various tasks in a zero-shot setting suggests that high-capacity models, trained to optimize the likelihood of diverse text corpora, inherently learn how to perform a remarkable range of tasks without the need for explicit supervision.


% Task conditioning forms the basis for zero-shot task transfer 


\subsubsection{Large Language Models}

These decoder-only Pre-trained Language Models have showcased their ability to generate coherent and contextually relevant text, estabilishing them as versatile tools for various \ac{NLP} tasks. Researchers have observed that scaling Pre-trained Language Models, whether by increasing model size or training data, frequently results in enhanced model capacity for a variety of downstream tasks. This phenomenon aligns with the scaling law, as suggested by \citet{kaplan2020scaling}. The success of decoder-only Pre-trained Language Models has spurred further development of larger and more sophisticated decoder-only Pre-trained Language Models, now referred to as Large Language Models.

In particular, the next iteration in the \ac{GPT} series, \ac{GPT}-3 \citep{brown2020language}, represents a significant milestone in the progression from Pre-trained Language Models to Large Language Models. \ac{GPT}-3 is a slightly modified version of \ac{GPT}-2 that demonstrates a significant capacity leap by scaling to a staggering size of 175 billion of parameters. \ac{GPT}-3 introduced the concept of \textit{in-context} learning, which allows the model to perform specific tasks by conditioning its responses on context provided in the prompt. In-context learning, also referred to as \textit{prompting}, encompasses zero-shot, one-shot, and few-shot learning. \ac{GPT}-3 uses the same architecture as \ac{GPT}-2 with the exception that attention patterns are sparse at alternating layers. Pre-trained on an even larger dataset, \ac{GPT}-3 has empirically shown that scaling Pre-trained Language Models to a significant size and formulating text to guide models to perform specific tasks (in-context learning) can lead to a huge increase in model capacity, especially in few and zero-shot learning scenarios.

% It has empirically demonstrated that scaling neural networks to a significant size and formulating text to induce models to perform desired tasks (in-context learning) can result in a huge increase in model capacity, especially in few and zero-shot learning.

% The idea is that, during pre-training, language models develop pattern recognition while learning to predict the following word conditioned on the context. Therefore, Pre-trained Language Models may be able to generate the correct task solution (formatted as a text sequence) given the task desk description, task-specific input-output examples, and a prompt. In-context learning, also referred to as \textit{prompting}, encompasses zero-shot, one-shot, and few-shot learning. 

The series of \ac{GPT} models has allowed significant progress in the field of \ac{NLP} by demonstrating the power of Large Language Models. Building on the success of the \ac{GPT} series, a myriad of Large Language Models have been released \citep{scao2022bloom, chowdhery2022palm, touvron2023llama}. The progress in Large Language Models has expanded into more specific domains, with models tailored for specialized tasks such as medical language processing \citep{thirunavukarasu2023large}, scientific research \citep{wang2023scientific}, website development \citep{wang2023software}, and code generation \citep{xu2022systematic}. A groundbreaking application of Large Language Models is ChatGPT \footnote{\url{https://openai.com/blog/chatgpt}}, an adaptation of the \ac{GPT}-series designed for dialogue. ChatGPT exhibits exceptional conversational abilities with humans and has sparked a wealth of reviews and discussions on the advancements of Large Language Models \citep{zhao2023survey, mohamadi2023chatgpt, hadi2023large}. 

In recent years, the size of Pre-trained Language Models has been scaled from a few million parameters (\acp{BERT}, 110M) to hundreds of billions of parameters (PaLM, 540B). This scaling has boasted capabilities, enabling models to generate even more coherent and natural-sounding text and further pushing the boundaries of \ac{NLP}. 

% Large Language Models have opened up new possibilities for \ac{NLP}, further pushing the boundaries of \ac{NLP}.

% During the past few years, GenAI models size has been scaled from a few million parameters(BERT [45], 110M) to hundreds of billions of parameters (GPT [112], 175B). Generally speaking, as the size of the model (number of parameters) increases, the performance of the model also increases [113], and it can be generalized for a variety of tasks [114], for example, Foundation models [115]. However, smaller models can also be fine-tuned for a more focused task [116]


% Large language models offer an exciting prospect of formulating text input to induce models to perform desired tasks via in-context learning, which is also known as prompting. For example, chain-of-thought prompting (Wei et al., 2022), an in-context learning method with few-shot “question, intermediate reasoning steps, answer” demonstrations, elicits the complex reasoning capabilities of large language models to solve mathematical, commonsense, and symbolic reasoning tasks. Sampling multiple reasoning paths (Wang et al., 2023), diversifying few-shot demonstrations (Zhang et al., 2023), and reducing complex problems to sub-problems (Zhou et al., 2023) can all improve the reasoning accuracy. In fact, with simple prompts like “Let’s think step by step” just before each answer, large language models can even perform zero-shot chain-of-thought reasoning with decent accuracy (Kojima et al., 2022). Even for multimodal inputs consisting of both text and images, language models can perform multimodal chain-of-thought reasoning with further improved accuracy than using text input only (Zhang et al., 2023).

%%%%%%%%%%%%%%%%%%%%%%%%%%%%%%%%%%%%%%%%%%%%%%%%%%%%%%%%%%%%%%%%%%%%%%%%%%%%%%%%%%%%
%%%%%%%%%%%%%%%%%%%%%%%%%%%%%%%%%%%%%%%%%%%%%%%%%%%%%%%%%%%%%%%%%%%%%%%%%%%%%%%%%%%%
\section{Long-range modeling}
\label{sec:chapter2-long-range}

% Due to the ever-growing volume, it is difficult for humans to read, process, and extract vital and pertinent information from large-scale long texts. 

In real-world scenarios, long text serves as a major information medium documenting human activities, \textit{e.g.}, academic articles, official reports, and meeting transcripts. Consequently, a compelling need arises for \ac{NLP} systems to model long texts and extract information of human interest. Broadly, the objective of long-range modeling is to capture salient semantics from text through informative representations, which hold utility for diverse downstream applications.
 
Furthermore, computational efficiency cannot be overlooked. As the document's length increases, the time and memory requirements needed to model the text with standard Transformers increase quadratically, adding a substantial burden for practical applications. This high computational cost originates from various factors, with the computation of self-attention being a major contributor. In order to calculate the dot product $\bm{Q}\bm{K}^{\top} \in \mathbb{R}^{n \times n}$, the inner product of every single key with every single query must be computed, for each layer and each attention head.  In detail, the computational complexity for a self-attention operation on a single sequence is $\mathcal{O}(hdn^2)$. The memory complexity to compute the attention matrix is $\mathcal{O}(hdn + hn^2)$, the first term being the memory required to store keys and queries, and the second term referring to the scalar attention values produced by each head. Hence, the $\bm{Q}\bm{K}^{\top}$ matrix multiplication alone results in $n^2$ time and memory requirements, constraining the use of Transformers models to short sequences. Furthermore, the two feed-forward network components in each Transformer block also significantly contribute to the cost of Transformers. While having a linear complexity with respect to sequence length, feed-forward networks are still, in practice, resource-intensive.

Additionally, long document harbor distinct attributes when compared to shorter texts. As long texts are typically domain-specific articles with complex hierarchical structures, there is a need to consider long-range dependency, inter-sentence relations, and discourse structure.

In this section, we focus mostly on modeling advances and architectural innovations that tackle the quadratic complexity issue of the self-attention mechanism. Furthermore, we explore benchmarks specifically designed for evaluating long-range modeling capabilities.

\subsection{Long-range Models}

To alleviate the cost of Transformers, a diversity of efficient self-attention model variants \citep{tay2020efficient} have been proposed over the past few years. Termed as \textit{Long-range Transformers}, these variants play a vital role in applications that model long sequences. Based on their core techniques and primary use case, long-range Transformers can be grouped into three categories \citep{qin2022nlp}: \textit{sparse patterns}, \textit{recurrence}, and \textit{low-rank and kernel} methods. While the goal of most of these models is to improve the complexity of the self-attention mechanism, we also include methods that improve the general efficiency of the Transformer architecture. Most of these models can be used both as an encoder-only and an encoder-decoder model. 

%Enhancements made to self-attention are only applied at the encoder-level.

\subsubsection{Sparse Patterns}

The earliest modifications to self-attention apply pattern-based methods to sparsify the attention matrix. The key idea is to relax the constraint that a single layer is necessary to aggregate information from any two tokens. Although the attention of each layer is not full, the receptive field can be increased as multiple layers are stacked. Pattern-based methods reduce the dense attention matrix to a sparse version by only computing attention on a sparse number of query-key pairs, hence restricting the field of view to patterns. 

\paragraph{Longformer}

Longformer \citep{beltagy2020longformer} uses three patterns: \textit{sliding window attention} restricts each token's field of view to a local window, \textit{dilated window attention} makes each token only attend at fixed intervals, and \textit{global attention} allows some fixed, user-defined tokens to attend to every other token and vice-versa. The key concept underlying the first two patterns is similar to convolution: the most important information is supposedly contained in the neighbourhoods of the tokens. Thus, in one layer, a single token can only attend to itself and its neighbours. However, dilated sliding window attention alone does not suffice to produce task-specific representations: some tokens are so important that it is highly beneficial that each token is connected to them and conversely (\textit{e.g.}, through a single layer, the \texttt{[CLS]} token needs to have access to all input tokens for sequence classification tasks). Global attention addresses this issue by allowing the model to learn task-specific representations. Overall, the time and memory complexity of Longformer is $\mathcal{O}(2sn)$, where $s$ is the number of global tokens. 

The pre-trained checkpoint for the encoder-only model has been trained using \ac{MLM} on sequences of 4,096 tokens extracted from long documents \citep{trinh2018simple, zellers2019defending}. Longformer was evaluated on three \ac{NLP} tasks: question answering \citep{welbl2018constructing, joshi2017triviaqa, yang2018hotpotqa}, coreference resolution \citep{pradhan2012conll}, and document classification \citep{maas2011learning, kiesel2019semeval}. 

Given the global attention mechanism, computing causal masks becomes unfeasible. Consequently, Longformer's attention cannot be used in an autoregressive setting. However, it can still be leveraged for sequence-to-sequence modeling. \ac{LED}, a Longformer variant for supporting long document generative sequence-to-sequence tasks, was proposed for summarization. In this configuration, Longformer's attention is used in the encoder while standard self-attention is employed in the decoder. 

A notable limitation of Longformer is its reliance on custom CUDA kernels to implement a block-sparse variance of matrix-matrix multiplication, rendering it impractical for use on TPUs.

\paragraph{BigBird}

\citet{zaheer2020big} introduced BigBird, an extension to Longformer that adds a \textit{random pattern attention}, by which tokens can attend to any other tokens randomly. Each query attends to $r$ random keys, where $r$ is a small constant number, chosen randomly. The intuition behind this mechanism is that the path lengths in a randomly connected graph are on average logarithmic. BigBird has linear time and memory complexity, and does not introduce new parameters beyond the Transformer model. 

BigBird has been evaluated on question answering \citep{yang2018hotpotqa, , welbl2018constructing, kwiatkowski2019natural} and classification \citep{zhang2015character, maas2011learning, kiesel2019semeval, cohan2018discourse, sharma2019bigpatent} tasks. Similar to Longformer, BigBird's sparse attention cannot be used in an autoregressive fashion. However, it can be integrated in the encoder component of the original Transformer architecture for sequence-to-sequence tasks. The resulting model was evaluated on summarization tasks \citep{cohan2018discourse}. Furthermore, \citet{zaheer2020big} also leveraged BigBird's longer sequence capabilities for genomics applications \citep{dreos2013epd, zhou2015predicting}.

\paragraph{Reformer}

Rather than employing fixed patterns, Reformer \citep{kitaev2020reformer} uses learnable patterns that enable the model to learn the access pattern in a data-driven fashion. Learnable patterns facilitates a more global view of the sequence while maintaining the efficiency benefits of fixed patterns approaches. Reformer introduces \ac{LSH} attention, a novel attention mechanism that consists in sharing parameters between $\bm{Q}$ and $\bm{K}$, and clustering tokens into chunks. This concept is rooted in the idea that if the sequence is long, $\text{Softmax}(\bm{Q}\bm{K}^{\top})$ only puts significant weight on very few key vectors for each query vector. Hence, given a query $\bm{q}$, $\text{Softmax}(\bm{qK})$ can be approximated by using only the keys that have a high cosine similarity with $\bm{q}$. If $\bm{K} = \bm{Q}$ then only the similarity of query vectors to each other has to be computed. Using the \ac{LSH} algorithm, query vectors are hashed into buckets of similar vectors. Attention is then computed among each bucket. If the bucket size is appropriately selected, the time and memory complexity of Reformer is $\mathcal{O}(n \log n)$. The model can easily be trained on sequences as long as 64,000 tokens. Reformer can be used in both bidirectional and autoregressive settings. 

As a downstream application to evaluate Reformer, \citet{kitaev2020reformer} used image generation tasks \citep{parmar2018image}.

\paragraph{ETC}

The \ac{ETC} model \citep{ainslie2020etc} represents another iteration within the Sparse Transformer family. It introduces a novel \textit{global-local} attention mechanism, encompassing four distinctive components: \textit{global-to-global}, \textit{global-to-local}, \textit{local-to-global}, and \textit{local-to-local} attentions. In addition to the original input, \ac{ETC} integrates $n_g$ auxiliary tokens at the beginning of the sequence, functioning as global tokens for participating in global-to-* and *-to-global attention processes. The local-to-local component operates as a localized attention mechanism with a predefined radius of $k$. Notably, \ac{ETC}'s approach closely resembles that of Longformer in its incorporation of global auxiliary tokens, which function as trainable parameters and can be interpreted as a form of model memory that pools across the sequence to collect global sequence information. The memory complexity of \ac{ETC} is $\mathcal{O}(n_g^2 + n_n N)$. 

\ac{ETC} was evaluated on two \ac{NLP} tasks: question answering \citep{yang2018hotpotqa, welbl2018constructing, kwiatkowski2019natural} and information extraction \citep{xiong2019open}. 

Because of the global attention mechanism, \ac{ETC} is not suitable for tasks where preserving the temporal order of information is essential, such as in sequence generation tasks. Furthermore, \ac{ETC} substantially increases code complexity, attributed to the addition of many attention directions.

% \subsubsection{Recurrence and Compressed Memory}
\subsubsection{Recurrence}

Recurrence and compressed memory approaches incorporate \textit{segment-level recurrence} into Transformer models to lengthen their attention span. The underlying concept of segment-based recurrence methods is to consider blocks of local receptive fields by chunking the input sequence into segments, and then connect them via recurrence.

\paragraph{Transformer-XL} Rather than attempting to reduce the cost of self-attention, \citet{dai2019transformer} take inspiration from \acp{RNN} and propose Transformer-XL, a causal language model that introduces a segment-based recurrence mechanism to connect adjacent segments. In Transformer-XL, segments are sequentially fed to the model, and tokens within a segment attend to the rest of the segment \textit{and} to the hidden states of the previous segment. Hence, after the first segment, tokens in subsequent segments will always have an immediate context size of $n$. By stacking multiple attention layers, the receptive field can be increased to multiple previous segments. In addition, this recurrence mechanism provides context for tokens in the beginning of a new segment. 
 
%Rather than treating the inputs as a sum of content and absolute position embeddings, each layer’s attention operation is broken up into a portion that attends based on content and a portion that attends based on relative position – for the 512th token in a chunk to attend to the 511th, the embedding corresponding to relative position -1 is used. Absolute position embeddings are only considered while computing attention weights, where they can be replaced with relative position embeddings.

% Transformer-XL introduces novel relative position encodings. In this scheme, absolute positional encodings are not added to the content embeddings. Instead, they are only considered while computing attention weights where they can be replaced with relative position encodings. S

\paragraph{XLNet}

XLNet \citep{yang2019xlnet} leverages both autoregressive and bidirectional language modeling. Unlike traditional autoregressive models that rely on fixed forward/backward factorization orders, XLNet maximizes the expected log likelihood of a sequence across all possible permutations of factorization orders. This approach allows each position in the sequence to consider tokens from both left and right, creating a bidirectional context. Additionally, XLNet incorporates the segment recurrence mechanism and relative encoding scheme of Transformer-XL during pre-training. This integration empirically improves the model's performance, specifically for tasks involving long text sequences. 

XLNet was evaluated on several \ac{NLP} tasks involving reading comprehension \citep{lai2017race}, document ranking\footnote{\url{https://lemurproject.org/clueweb09/}}, question answering \citep{rajpurkar2016squad}, text classification \citep{maas2011learning, zhang2015character}, and \ac{NLU} \citep{wang2018glue}.

% \paragraph{Compressive Transformers}

% In contrast to Transformer-XL, which entirely discards past activations as it moves across segments, Compressive Transformers \citep{rae2019compressive} retain a more detailed and fine-grained memory of previous segment activations. In this model, past activations are stored and compressed, contributing to a more effective capture of relevant information from earlier segments and its subsequent utilization in further processing. Hence, Compressive Transformers have access to a broader context and are able to capture longer-range dependencies across segments. 

% Instead of discarding past activations entirely, Compressive Transformers store and compress this information. This allows the model to preserve a broader context and capture longer dependencies across segments. By maintaining this compressed memory, Compressive Transformers can better capture relevant information from earlier segments and utilize it in subsequent processing, leading to improved contextual understanding and performance, especially for tasks requiring a strong grasp of distant dependencies.

\subsubsection{Low-rank and Kernels}

Another approach to improve the efficiency of Transformer models is to approximate the self-attention mechanism using \textit{low-rank approximation} or \textit{kernelization}. The idea revolves around mathematically redefining the self-attention mechanism, which eliminates the need to explicitly compute the $n \times n$ matrix.

\paragraph{Linformer} In a high-rank matrix, no particular dimension has much more information than any other. Conversely, most of the information in a low-rank matrix is concentrated in very few dimensions, meaning that most of the dimensions are redundant. The core idea behind Linformer \citep{wang2020linformer} is to approximate the self-attention matrix with a lower rank matrix: the keys and values are projected to a lower-dimensional space $k \times d$, in which the attention matrix is computed. The projected matrices $k \times d$ can be viewed as producing a set of $k$ pseudo-tokens that summarize the sequence — each of these pseudo-tokens indicates how highly a given filter activates on average when dotted with the full sequence of corresponding representations. As $k$ does not depend on the sequence length, the time and memory complexity of Linformer is linear. There is only a minimal parameter costs of the Linformer due to the extra $nk$ length projections. If $k$ is sufficiently small, there is negligible parameter costs incurred. 

To evaluate Linformer, \citet{wang2020linformer} used sentiment classification \citep{maas2011learning, socher2013recursive}, \ac{NLI} \citep{wang2018glue} and textual similarity \citep{wang2017bilateral} tasks. 

Because projecting on the length dimension $n$ causes mixing of sequence information, it is non-trivial to maintain causal masking and/or prevent mixing of past and future information when computing attention scores. Hence, Linformer's attention approximation cannot be used in an autoregressive setting.


\paragraph{Performer} To estimate standard full-rank-attention Transformers without relying on any prior such as sparsity or low-rankness, \citet{choromanski2020rethinking} propose a kernel-based approach that uses a generalized attention framework to approximate any attention matrix. The attention matrix $\text{Softmax}(\bm{Q}\bm{K}^{\top})$ can be approximated using lower-rank randomized matrices $\bm{Q'}$ and $\bm{K'}$ where the rows encode positive-valued nonlinear functions of the original $\bm{Q}$ and $\bm{K}$. This approximation allows to store the implicit attention matrix $\bm{A}$ with
linear memory complexity. To obtain a linear time complexity, matrix multiplications are rearranged: instead of multiplying $\bm{A}$ with $\bm{V}$ to obtain the final $n \times d$ matrix, $\bm{K'}^{\top} \in \mathbb{R}^{k \times n}$ is first multiplied with $\bm{V} \in \mathbb{R}^{n \times d}$, and $\bm{Q'} \in \mathbb{R}^{n \times k}$ is multiplied with the resulting matrix $\bm{K'}^{\top} \bm{V} \in \mathbb{R}^{k \times d}$. This framework allows to create a broad class of attention mechanisms based on different similarity measures (kernels). 

Performer was evaluated as an encoder-only model on protein modeling \citep{uniprot2019uniprot} and image generation \citep{parmar2018image} tasks. 

Kernel-based self-attention, while offering computational advantages, introduces complexities that make it less amenable to enforcing causal masking. The randomized feature map and the approximations involved in the kernel-based approach might not inherently preserve the sequential order required for causal masking. 

% The model poses considerable challenges in meeting the requirements for achieving causal masking. 


\subsection{Benchmarking Long-range Models}

The broad array of efficient Transformers that has emerged to address the challenge of long-range modeling have been mostly evaluated using perplexity or differents sets of tasks and datasets. However, an increasing
amount of literature shows that predicting the next token is mostly a local task that does not require modeling long-range dependencies \citep{khandelwal2018sharp, sun2021long}. In addition, the large diversity of tasks and datasets used complicates the comparison of models and the assessment of their relative strenghts and weaknessses. In this subsection, we describe unified benchmarks designed to shed light on the ability of these architectures to reason in long-context scenarios, and study their performance in handling \ac{NLP} tasks.

\subsubsection{Long-Range Benchmarks}

\paragraph{Long-Range Arena} \citet{tay2020long} introduce a systematic and unified benchmark, \ac{LRA}, designed to evaluate the ability of a model to reason in long-context scenarios. This benchmark consists of five classification tasks with sequences ranging from 1,000 to 16K tokens, encompassing various data types and modalities (text, natural and synthetic images, mathematical expressions). This benchmark was created based on a set of specific requirements and criteria. First, \ac{LRA} is general: all long-range Transformer models are applicable to the tasks. The tasks are intentionally designed to promote simple models rather than cumbersome pipelined approaches. Furthermore, the tasks are crafted to be challenging enough to ensure there is room for improvement. The input sequences are reasonably long, and the set of tasks allows to assess different capabilities of models. Finally, \ac{LRA} is deliberately non-resource intensive and accessible. 

In \textit{Long ListOps}, sequences with a hierarchical structure and mathematical operators are given as input and the model has to predict the mathematical result of the sequence as a classification task. The goal is to evaluate the ability to model hierarchically structured data while handling long contexts. 
In the \textit{Character-level Sentiment Analysis} task, the model is provided with character-level IMDb reviews \citep{maas2011learning} and has to classify them into positive or negative. This task benchmarks the ability of the model to deal with compositionality as it is required to compose characters into words, and words into higher-level phrases.
Given two documents from the ACL Anthology Network \citep{radev2013acl} represented as character-level sequences, the \textit{Character-level Document Relatedness} task consists in predicting whether these documents are related. This task assesses the capability of a model to compress long sequences into representations suitable for similarity-based matching. As previously, the character level setup challenges the model to compose and aggregate information over long contexts.
Using the CIFAR-10 dataset \citep{krizhevsky2009learning}, the \textit{Image Classification on sequences of pixels} task requires the model to learn the 2D spatial relations between input pixels, while presented as a 1D sequence of symbols.
In \textit{Pathfinder}, the model is given a sequence of pixels and has to predict whether two points are connected by a path. A more challenging version with extreme lengths, \textit{Pathfinder-X}, evaluates if the same algorithmic challenges bear a different extent of diffculty when sequence lengths are much longer.

However, only two out of these five tasks involve natural language, constraining the relevance of \ac{LRA} for \ac{NLP}. Furthermore, \ac{LRA} artificially increases the length of natural language sequences through character tokenization, and truncates each example at 4,000 characters, equivalent to less than 1,000 words. This exempts models from dealing with the complex long-range interactions present in long texts.
% The tasks in the \ac{LRA} benchmark are specifically designed for the purpose of probing different aspects of long-range Transformer models. 

\paragraph{SCROLLS} To extend evaluation beyond sentences and paragraphs, \citet{shaham2022scrolls} propose \ac{SCROLLS}, an \ac{NLP} benchmark featuring datasets with long texts, ranging from an average of 1,700 words to over 51,000. \ac{SCROLLS} consists of seven datasets covering various domains and tasks: summarization from government reports (\textit{GovReport}), TV shows transcripts (\textit{SummScreenFD}), and meetings (\textit{QMSum}), question answering from scientific articles (\textit{Qasper}), books (\textit{NarrativeQA}), and stories (\textit{QuALITY}), and \ac{NLI} from non-disclosure agreements (\textit{Contract NLI}). Every task is reformulated as a sequence-to-sequence problem for a simple unified format. \ac{SCROLLS} presents a challenge not only due to the length of its inputs but also because it requires models to process long-range interactions across different sections. 

\subsubsection{On the Effectiveness of Long-range Models on NLP Tasks}

To validate the effectiveness and long-range ability of long-range Transformers on language tasks and uncover the underlying factors behind model behaviors, \citet{qin2022nlp} benchmark different long-range Transformer models on \ac{NLP} tasks characterized by long sequences. Five complex, long-text \ac{NLP} tasks are considered, covering a wide spectrum of typical language scenarios: token/span-level prediction, sequence-level classification, and sequence-to-sequence generation.

\paragraph{Sparse Pattern Models}

Longformer and BigBird are used to assess the performance of sparse pattern approaches. In coreference resolution, which consists in identifying mention spans and clustering them into entities, \citet{qin2022nlp} find that using larger sliding windows can be advantageous, but this advantage tends to level off or even decline after a certain point. In tasks where the amount of guiding text is limited, such as a query in question answering, setting it as global tokens can enhance its attention and substantially improve the overall performance. When there is no guiding text (\textit{e.g.}, in the case of coreference resolution), setting all tokens as global can have a detrimental impact on performance. Additionally, \citet{qin2022nlp} find a connection between long-range attention, global tokens, and the selectivity of sequence-to-sequence problems, which ultimately enhances the decoding process.

\paragraph{Recurrence Models} 

The effectiveness of recurrence-based methods is evaluated using XLNet. In various tasks, \citet{qin2022nlp} show that the memory of recurrence models tends to enhance performance, demonstrating the advantage of using past hidden states in Transformers. Nevertheless, XLNet falls short in maximizing the potential of past tokens, as it gives relatively less attention to distant information. This could be attributed to XLNet's pretraining objective of predicting masked tokens, which does not consistently require long-range context \citep{sun2021long}. Moreover, the application of the stop-gradient technique might impede the model's ability to efficiently focus on memories.

\paragraph{Kernel-based Models} 

Performer is used as a kernel-based model. It is found that the approximation technique of Performer demonstrates strong performance with shallow networks. However, when applied to deeply stacked Transformer layers, it encounters significant  error accumulation issues. This leads to a notable drop in performance, which is considered unacceptable even for the base version of Transformer encoders. \\

Drawing from their discoveries, \citet{qin2022nlp} offer a few recommendations. For typical tasks like sequence classification or token-level prediction, it remains effective to divide inputs into chunks and use short-range Transformer models. In cases where explicit guiding text such as queries is available, models based on sparse patterns and featuring a global token mechanism are preferable. For sequence-to-sequence problems, leveraging long-range Transformers with pre-trained checkpoints yields superior performance. \\

The exploration of long-range modeling has been marked by continuous efforts to reduce the cost of Transformers to efficiently model long texts. The quest for ideal long-range models demands finding an equilibrium. These models should address the quadratic issue of Transformers, showcase universality by performing well across most tasks, and remain simple without unnecessary hard-coding or engineering complexities. There should be no compromise between speed and memory efficiency, and they should be able to seamlessly integrate with TPUs and accomodate causal masking. 


\section{Document Understanding}

The majority of models, benchmarks, and tasks focus exclusively on a single source of information, namely plain text. However, disregarding the visual appearance of text is sub-optimal in real-world scenarios. Documents, ranging from webpages to digital-born PDF files and scanned document images, exhibit a diverse array of formats. These documents, including business forms, scholarly and news articles, invoices, financial reports and emails, convey information not only through language but also through \textit{visual} content (\textit{e.g.}, figures, text formatting) and \textit{layout} structure (\textit{i.e.}, text positioning). Extracting information from documents becomes a challenging task due to the diversity in layouts and formats, the presence of low-quality scanned document images, and the complexity of template structures. Manually extracting information is a time-consuming and labor-intensive process with low accuracy
and limited reusability. As such, Document Understanding, \textit{i.e.}, the process of automatically understanding, classifying and extracting information from \textit{visually-rich} documents, has emerged as a key research area. As the foundation of digital transformation, Document Understanding holds significant economic value and has experienced a surge in industrial demand in recent years. To reduce the time and cost of document workflows, more and more companies are shifting from labor-intensive, rule-based algorithms to deep learning-based entity recognition, document classification, semantic extraction, \textit{etc}. The study of Document Understanding spans multiple disciplines and involves developing models and techniques to comprehend the content and structure of complex, visually-rich documents. As such, Document Understanding is also crucial from an academic standpoint as it opens up avenues for various research and advancements in \ac{NLP} and Computer Vision. 

Over the last thirty years, the development of Document Undrstanding has undergone various stages — starting from rule-based heuristics to the rise of neural network approaches. In the early 1990s, researchers relied on rule-based heuristic methods constructed by manually observing document layout information \citep{wong1982document, fisher1990rule, lebourgeois1992fast}. However, these handcrafted rules proved to be non-scalable, and the adoption of rule-based approaches often resulted in high labor costs. As Machine Learning technology rose in the 2000s, models based on annotated data \citep{baechler2011multi, wei2013evaluation} became the predominant approach for document processing, marking a transition towards more scalable and data-driven approaches in Document Understanding. While offering a certain degree of performance enhancement, their general usability often falls short due to the lack of customized rules and limited training samples. Moreover, the adaptation costs for various document types are relatively high, rendering previous approaches impractical for widespread commercial use. In recent years, the advent of Deep Learning and the accumulation of vast amounts of unlabeled electronic documents, have propelled Document Understanding into a new era. This era, termed as \textit{deep learning-based Document Understanding}, embraces the “pre-training then fine-tuning” paradigm, resulting in a significant breakthrough in the field \citep{xu2020layoutlm, peng2022ernie}.

In this section, we begin by providing an overview of the representative tasks and datasets in Document Understanding. We then explore the most recent and significant advancements in the field driven by Deep Learning, focusing on two research directions. The first direction involves task-specific approaches that employ shallow fusion between textual and visual/layout information. On the other hand, the second axis explores the application of pre-training techniques for deep fusion of modalities, significantly advancing analysis performance and accuracy. 

% Définir enjeux, tâches, métriques

\subsection{Document Understanding Tasks and Datasets}

% These visual and layout aspects are prominent in tasks that could be much better solved when provided with not just text, but also multimodal information encompassing aspects such as text positioning, text formatting, and visual elements. 

The field of Document Understanding covers problems that involve reading and interpreting visually-rich documents (in contrast to plain texts), requiring comprehending the conveyed multimodal information. Real-world application scenarios can be classified into four types of tasks: Document Layout Analysis, Visual Information Extraction, Document Image Classification, and \ac{Document VQA}. While document layout analysis and document image classification are more centered around processing visual data (\textit{image-centric}), visual information extraction and \ac{Document VQA} put more focus on textual data (\textit{text-centric}).

\subsubsection{Document Layout Analysis}

\textit{Document layout analysis} consists in automatically locating and categorizing the components (\textit{e.g.}, text, tables, figures) of a document. This task includes two primary subtasks: \textit{page segmentation} and \textit{logical structural analysis} \citep{binmakhashen2019document}. Page segmentation consists in detecting the structure of the document and establishing a partition into distinct regions such as text, figures, images, and tables. On the other hand, logical structural analysis focuses on providing finer-grained semantic classifications within the previously detected regions, \textit{e.g.} identifying a region of text that is a paragraph. Document layout analysis plays a crucial role in parsing semi-structured documents into structured, machine-readable formats for downstream applications, such as \ac{OCR}. This task is challenging due to the varying layouts and formats of the documents. Several benchmark datasets have emerged for document layout analysis. The International Conference on Document Analysis and Recognition (ICDAR) have produced several gold-standard datasets from their annual competitions \citep{antonacopoulos2013icdar, gao2017icdar2017}. PubLayNet \citep{zhong2019publaynet} consists of biomedical and life sciences articles and involves detecting and classifying page regions into categories including caption, list, and paragraph. 
DocBank \citep{li2020docbank} comprises scientific research papers with fine-grained token-level annotations for better applicability to \ac{NLP} methods.

\textit{Table understanding} is a crucial and challenging subtask of document layout analysis. Tables serve as a concise and effective means of summarizing and presenting information in various types of documents. Unlike other document elements such as headings and paragraphs, tables display greater variability in format and a more complex structure. Consequently, significant research has been conducted on tables, focusing on two key subtasks: 1) Table detection, which aims to determine the boundaries of tables in the document, and 2) Table structure recognition, where the objective is to extract the layout structure of tables, including information about rows, columns, and cells. ICDAR held several competitions to evaluate both aspects of table understanding, using both modern and archival documents \citep{gobel2013icdar, gao2019icdar}. To address the need for larger datasets, \citet{li2019tablebank} proposed TableBank, a large-scale dataset built from Office Word and \LaTeX documents using weak supervision. With PubTabNet, \citet{zhong2020image} offer additional information on table structure and cell content to assist in table recognition.

% Traditionally, \ac{DLA} has been tackled by using models that largely rely on conventional rule-based or machine learning techniques. However, these approaches fail to generalize well due to their dependence on manually crafted features that may not withstand layout variations. Recently, the rapid advancement of deep learning in the field of Computer Vision has greatly propelled the use of data-driven image-based strategies for \ac{DLA}. Common \ac{DLA} datasets, such as PubLayNet \citep{zhong2019publaynet} and DocBank \citep{li2020docbank}, involve detecting and classifying page regions or tokens into categories such as caption, list, paragraph, \textit{etc}. 


\subsubsection{Document Image Classification}

\textit{Document image classification} refers to the process of classifying document images into various categories such as emails, invoices, scientific papers, and more. Unlike natural images, document images primarily consist of textual content displayed in diverse styles and layouts. Therefore, Document Image Classification is a special subtask of image classification that requires understanding both visual and textual aspects of documents. The Tobacco-3482 dataset \citep{kumar2013unsupervised} consists of 3,482 document images classified into 10 classes. Widely used for document image classification, the large-scale RVL-CDIP dataset \citep{harley2015evaluation} is a representative dataset for this task and encompasses 400,000 images distributed across 16 categories. 


\subsubsection{Visual Information Extraction}

\textit{Visual information extraction} consists in extracting semantic entities and their relationships from visually-rich documents, based on a set of pre-defined keys. In contrast to traditional text-only information extraction, the two-dimensional spatial arrangement of the text requires an understanding of page layouts for complete comprehension. Notable public datasets for this task include \ac{FUNSD} \citep{jaume2019funsd}, a form understanding dataset consisting of 199 real, noisy and scanned forms where each sample contains key-value pairs of form entities. For receipt understanding, \ac{SROIE} \citep{huang2019icdar2019} (973 English documents) and \ac{CORD} \citep{park2019cord} (1000 Indonesian documents) are widely used datasets. Kleister \citep{gralinski2020kleister} comprises non-disclosure agreements and financial reports, serving as a benchmark for advancing entity extraction from long and complex documents. EPHOIE \citep{wang2021towards} contains Chinese document images collected from scanned examination papers. XFUND is a multilingual extension of \ac{FUNSD} \citep{xu-etal-2022-xfund} and contains manually-labeled forms in 7 languages.

% Visual information extraction tasks can be framed as a Computer Vision problem, where semantically meaningful regions are detected through text box detection, and labeled using semantic segmentation. 

% Given a document and a set of keys, Key Information Extraction consists in extracting from the document the values of the given set of keys, e.g., the total amount in a receipt or the date in a form. In \ac{KIE} tasks, documents have a layout structure that is crucial for their interpretation. Notable public datasets in the field include the FUNSD (Form Understanding in Noisy Scanned Documents) dataset \citep{jaume2019funsd}, consisting of 199 real, noisy and fully annotated scanned forms. For receipt understanding, SROIE (Scanned Receipts OCR And Key Information Extraction) \citep{huang2019icdar2019} (973 documents) and CORD (Consolidated Receipt Dataset) \citep{park2019cord} (1000 documents) are widely used. \citet{gralinski2020kleister} elicit progress on deeper and more complex \ac{KIE} by introducing Kleister-NDA and Kleister-Charity, two collections of, respectively, non-disclosure agreements and financial reports with varying lengths. The objective is to help extending the understanding of documents with substantial lengths, various reasoning problems, complex layouts and OCR quality problems.


\subsubsection{Document Visual Question Answering}

% processing the multimodal information (\textit{i.e.}, text, layout, images) conveyed by the document

\textit{Document visual question answering} is a high-level understanding task that requires providing the correct answer to a question related to a visually-rich document, posed in natural language. This is achieved by jointly reasoning over the document layout (page structure, forms, tables), textual content (handwritten or typewritten), and visual elements (marks, tick boxes, diagrams). Unlike traditional \ac{VQA} tasks, textual information holds a pivotal role in \ac{Document VQA}. The DocVQA dataset \citep{mathew2021docvqa} contains more than 12,000 industry documents and 5,000 corresponding questions. InfographicsVQA \citep{mathew2022infographicvqa} comprises infographic images with questions that require elementary reasoning and basic arithmetic skills. To spurr further research on developping generation abilities, VisualMRC \citep{tanaka2021visualmrc} was built from multiple domains of webpages and requires producing long and abstractive answers. \\

% Document \ac{VQA} can be approached in different ways, depending on the characteristics of the dataset. In the case of the DocVQA dataset, where answers to questions often take the form of text fragments within the document, prevailing methods model the problem as a machine reading comprehension task. In this scenario, the model is provided with visual features and textual contents and is tasked with extracting relevant text fragments from the documents based on the questions. In contrast, for datasets like VisualMRC, answers usually do not appear in the document as text fragments, and a longer abstractive answer is required. An effective strategy for such cases involves using a text generation method to generate answers to the questions. 


The number of open-sourced benchmark datasets proposed for these tasks has significantly facilitated the development of new techniques and models. Particularly noteworthy is the recent surge in deep learning-based models that have achieve state-of-the-art performance across these tasks.

\subsection{Deep Learning-based Document Understanding}

Deep Learning marked a paradigm shift by enabling significant performance leaps across various research areas. In particular, the fields of \ac{NLP} and Computer Vision have undergone substantial advancements due to the emergence of Deep Learning. The development of Document Understanding also reflects a similar trend, where methodologies from \ac{NLP} and Computer Vision are integrated into modern document understanding systems. We first explore the early application of Deep Learning to improve Document Understanding, focusing on task-specific methods. We then delve into the more recent and successful strategy of general-purpose multimodal pre-training.

%  but the success of Deep Learning has put Computer Vision and \ac{NLP} models at the heart of contemporary approaches.

\subsubsection{Task-specific Deep Learning Models}

The initial approach to enhancing document understanding system with Deep Learning involves the use of task-specific methods. These methods leverage pre-trained Computer Vision and/or \ac{NLP} models to gather knowledge from the corresponding modality. Features are extracted from either a single modality or through a simple combination of modality features, and used for a specific downstream task. 


\paragraph{Document Layout Analysis} can be framed as an instance segmentation task for document images, where units such as text, figures, headings, paragraphs,
and captions are the objects that need to be detected and recognized. In this task, the model has to predict per-pixel labels to categorize regions of interest within the document. These methods offer flexibility and can adapt to both the coarser-grained task of page segmentation and the finer-grained task of logical structural analysis. \citet{yang2017learning} proposed an end-to-end \ac{CNN} that combines both text and visual features within an encoder-decoder architecture for pixel classification. The encoder-decoder architecture ensures that visual feature information at various levels of resolution is considered throughout the encoding and decoding process \citep{burt1987laplacian}. In addition to the resulting visual representation, text embeddings learned from a pre-trained \ac{NLP} model are supplied at the final decoding layer. Similarly, \citet{oliveira2018dhsegment} introduced a multi-task pixel-by-pixel prediction \ac{CNN}-based model to perform layout analysis on historical documents. \citet{soto2019visual} view layout analysis of scientific articles as an object detection task. By integrating contextual information into Faster R-CNN \citep{ren2015faster}, the local invariance of article elements is leveraged to improve region detection performance. 

Table detection in document images is often treated as an object detection task, leading to notable progress with diverse deep learning-based models. Faster R-CNN has been particularly successful when directly applied to table detection. CascadeTabNet \citep{prasad2020cascadetabnet} leverages the Cascade R-CNN \citep{cai2018cascade} model for simultaneous table detection and table structure recognition. Another notable contribution is TableSense \citep{dong2019tablesense}, which significantly enhances table detection by introducing cell features and employing advanced sampling algorithms.


\paragraph{Visual Information Extraction} 

For information extraction from visually-rich documents, many researchers and practitioners have framed the problem as an instance segmentation task. In this approach, semantically meaningful regions are identified through object detection and labeled via semantic segmentation. Given the pivotal role of text in visual information extraction, directly embedding textual information into the image simplifies comprehension of 2D textual relationships. Given $w$ the image width, $h$ the image height, and $d$ the embedding dimension, the standard framework treats document images as a pixel grid of $w \times h \times d$ where each pixel corresponds to a text embedding vector \citep{katti2018chargrid}. This grid representation preserves the 2D layout of documents, capturing details such as positioning, size, and alignment for textual components. Text embeddings are transposed into each pixel corresponding to the bounding box of the embedded text. Thereby, a textual element is encoded by a single vector rather than by a collection of pixels. A convolution-based encoder-decoder model is then applied to perform instance-level segmentation. Specifically, the model predicts a segmentation mask where each pixel is assigned a class label, and generates object bounding boxes to group multiple instances of the same class.  Depending on the granularity of textual information, this approach can work at the character level, word level, or word-piece level. Chargrid \citep{katti2018chargrid} employs a one-hot encoding for each character, constructing a 2D grid of characters by mapping each pixel intersecting with a character bounding box with the corresponding one-hot encoding. Expanding on Chargrid, VisualWordgrid \citep{kerroumi2021visualwordgrid} operates at the word level, incorporating word embeddings from Word2Vec or fastText. To improve the end-to-end accuracy, BERTgrid \citep{denk2019bertgrid} constructs a grid at the word-piece level, embedding each word piece with dense contextualized vectors obtained from \ac{BERT}. ViBERTgrid \citep{lin2021vibertgrid} takes a step further by concatenating a BERTgrid to an intermediate layer of a \ac{CNN} model, resulting in a more powerful grid-based document representation. 

Documents can also be represented as graph networks, wherein the nodes correspond to textual segments, and the relationships between text fragments are modeled as edges. \citet{liu2019graph} introduce a model based on \acp{GCN} to integrate both textual and visual information. In this model, a document takes the form of a graph where nodes represent textual segments. Each node comprises information about the position of the segment and the text it contains, while edges correspond to the relative distances between the corresponding segments and their aspect ratio. Graph convolution is employed to calculate graph embeddings for each text segment, which are then combined with text embeddings. The resulting embeddings are fed into a bidirectional \ac{LSTM} for information extraction from in-house invoices and receipts. This graph-based approach ensures that both local and global information can be learned. \citet{hwang2020spatial} model a document as a directed graph, extracting information through dependency analysis. \citet{yu2021pick} combines graph learning with graph convolution to achieve richer semantic representations. \\

\paragraph{Document Image Classification} constitutes a subtask of image classification. Consequently, classification models originally designed for natural images can be applied to tackle the challenges associated with document image classification. \citet{afzal2015deepdocclassifier} used a deep \ac{CNN} trained on the millions of examples from ImageNet \citep{deng2009imagenet}. \citet{das2018document} introduced a range of deep \acp{CNN} designed to classify specific regions within a document. These classifiers are combined using an effective ensembling technique for document image classification. \citet{dauphinee2019modular}  developed a model consisting of two distinct components — one dedicated to processing text and the other to handling images. This modular approach uses both visual information and textual content within a page to classify document images.

\paragraph{Document Visual Question Answering} can be approached in different ways, depending on the characteristics of the dataset. In the case of the DocVQA dataset, where answers to questions often take the form of text fragments within the document, prevailing methods model the problem as a machine reading comprehension task. In this scenario, the model is tasked with extracting relevant text fragments from the documents based on the questions. \citet{mathew2021docvqa} showed that BERT outperforms state-of-the-art text-augmented \ac{VQA} models \citep{singh2019towards, hu2020iterative} on DocVQA. In contrast, for datasets like VisualMRC, answers usually do not appear in the document as text fragments, and a longer abstractive answer is required. An effective strategy for such cases involves using a text generation method to generate answers to the questions. \citet{tanaka2021visualmrc} extended BART and T5 with bounding box information and layout features.

% However, these models are all designed for specific tasks and document types. Because the domain knowledge of one document type cannot be easily transferred into another, the models have to be re-trained when the document type is changed. Hereby, models based on shallow fusion cannot fully exploit the layout invariance among different document types (e.g. the arrangement of key-value pairs in forms is usually in the left-right order or the top-down order). Additionally, they rely on labeled data, yet many tasks related to Document Understanding are label-scarce. Following the current research trend in \ac{NLP}, a framework that can learn from unlabeled documents through pre-training and perform model fine-tuning for specific downstream applications is preferred over ones that require fully-annotated training data.


\subsection{Deep Fusion of Modalities via General-purpose Multimodal Pre-training}
\label{subsection:chapter2-deep-fusion}

While the aforementioned methods demonstrate good performance across document understanding tasks, they still face significant limitations. The majority of these models are all designed for specific tasks and document types. As such, these approaches rely on labeled data; however, most datasets related to Document Understanding are label-scarce. This scarcity arises from the labor-intensive and time-consuming nature of the human annotation process. Driven by data limitations, these task-specific models often rely solely on pre-trained Computer vision models and/or \ac{NLP} models, each trained independently. A common practice involves combining the knowledge gained from each modality through a shallow fusion of features, such as concatenation. However, this approach makes it challenging to easily transfer domain knowledge from one document type to another. This stems from the need to re-train models from scratch when the document type is changed.

% However, the correlation between different tasks, such as shared semantic representations between visual information extraction and document visual question answering, cannot be effectively leveraged. This stems from the need to re-train models from scratch when changing tasks, as the domain knowledge specific to one task cannot be easily transferred to another. 

% which cannot fully exploit the layout invariance among different document types (\textit{e.g.}, the arrangement of key-value pairs in forms is usually in the left-right order or the top-down order). This stems from the necessity of re-training models when the document type is changed. This is because the domain knowledge specific to one document type cannot be easily transferred to another. 

Visually-rich documents encompass three modalities that inherently align: text, layout, and visual information. Layout, \textit{i.e.}, the spatial relationship of text blocks within a document, often carries rich semantic information. For instance, the arrangement of key-value pairs in forms typically follows a left-right or top-down order. This layout invariance among document types is a crucial characteristic. In addition to spatial information, the visual elements presented with the text can offer global structural information (\textit{e.g.} there is a clear visual distinction between different document types) and help with downstream tasks (\textit{e.g.}, the title of documents is usually enlarged). Hence, it becomes vital to jointly learn text, layout, and visual information.

As visually-rich documents are widely used in real-world applications, there is a substantial volume of unlabeled documents. This provides an opportunity for leveraging self-supervised pre-training methods. The widespread success and popularity of pre-training techniques in transfer learning, notably with the Transformer architecture, emphasizes the critical role of deep contextualization for sequence modeling in both \ac{NLP} and Computer Vision problems. Following the current research trend, a general-purpose framework that can learn from unlabeled documents through pre-training and perform model fine-tuning for different types of downstream applications is preferred over ones that are task-specififc and require fully-annotated training data. This trend has prompted a shift in Document Understanding towards the “pre-training then fine-tuning” paradigm, marking a notable surge in the adoption of pre-training techniques in the field in recent years. In particular, researchers and practitioners have been leveraging the Transformer architecture to attain cross-modal alignment via joint pre-training of text, layout, and images from large amounts of unlabeled data. This approach enables pre-trained models to absorb cross-modal knowledge across various document types. Consequently, when the model is applied to a different domain with different document formats, only a small number of labeled samples are required to fine-tune the generalized model effectively. 

Next, we discuss general-purpose, multimodal pre-training methods for Document Understanding. We review these methods from several viewpoints, considering the various challenges in the field. This includes the integration of layout, image encoding, the pre-training tasks used, the incorporation of positional information, model initialization, and considerations for multilingual and long-range aspects. The models discussed are summarized in Table~\ref{table:document-understanding-models}.

% Hence, integrating layout and visual information into the pre-training stage allows for cross-modal alignment via self-supervised, joinHence, layout and visual information can be integrated into the pre-training stage to be jointly learned alongside the text. In particular, researchers and practitioners have been leveraging the Transformer architecture to attain cross-modal alignment via self-supervised, joint multimodal pre-training from large amounts of unlabeled data.

\begin{table}[h]
\centering
\small
\begin{adjustbox}{max width=\textwidth}
\renewcommand{\arraystretch}{1.25}
\begin{threeparttable}
\begin{tabular}{lcccccccc}
    \toprule
        & \textbf{Architecture} &                           & & \multicolumn{3}{c}{\textbf{Layout}} \\            
        & & \textbf{Visual Encoding}  & & \textbf{Bounding Box}  & \textbf{Encoding} & \textbf{Relative Bias} \\
        & & & & \textbf{Granularity} & & \\
    \midrule
    \rowcolor{lightgray}
    LayoutLM \citep{xu2020layoutlm} & Encoder & \xmark & & Word & Embedding Tables & \xmark \\
    LayoutLMv2 \citep{xu2020layoutlmv2} & Encoder & ResNeXt-FPN & & Word & Embedding Tables & \cmark \\
    \rowcolor{lightgray}
    LayoutXLM \citep{xu-etal-2022-xfund} & Encoder & ResNeXt-FPN & & Word & Embedding Tables & \cmark \\
    LayoutLMv3 \citep{huang2022layoutlmv3} & Encoder & Embedding Tables & & Block & Embedding Tables & \cmark \\
    \rowcolor{lightgray}
    \citet{pramanik2020towards} & Encoder & ResNet50-FPN & & Word & Embedding Tables & \xmark \\
    DocFormer \citep{appalaraju2021docformer} & Encoder & ResNet50 & & Word & Embedding Tables & \xmark \\
    \rowcolor{lightgray}
    ERNIE-Layout \citep{peng2022ernie} & Encoder & ResNeXt-FPN & & Word & Embedding Tables & \cmark  \\
    FormNet \citep{lee2022formnet} & Encoder & \xmark & & Word & \xmark & \cmark \\
    \rowcolor{lightgray}
    BROS \citep{hong2020bros} & Encoder & \xmark & & Word & Sinusoidal functions & \xmark \\
    StructuralLM \citep{li2021structurallm} & Encoder & \xmark & & Block & Embedding Tables & \xmark \\
    \rowcolor{lightgray}
    SelfDoc \citep{li2021selfdoc} & Dual-stream encoder & Faster R-CNN & & Block & \xmark & \xmark \\ 
    LiLT \citep{wang2022lilt} & Dual-stream encoder & \xmark & & Word & Embedding Tables & \xmark \\
    \rowcolor{lightgray} 
    DiT \citep{li2022dit} & Encoder + detection framework & Embedding Tables & & \xmark & \xmark & \xmark \\
    H-VILA \citep{shen2022vila} & Hierarchical encoder & \xmark & & Block & Embedding Tables & \xmark \\ 
    \rowcolor{lightgray}
    LAMPreT \citep{wu2021lampret} & Hierarchical encoder & CNN + Linear Layer & & Block & \xmark & \xmark \\
    TILT \citep{powalski2021going} & Encoder-decoder & U-Net & & Word &\xmark & \cmark \\
    \rowcolor{lightgray}
    LayoutT5 \citep{tanaka2021visualmrc} & Encoder-decoder & Faster R-CNN & & Word & Embedding Tables & \xmark \\
    LayoutBART \citep{tanaka2021visualmrc} & Encoder-decoder & Faster R-CNN & & Word & Embedding Tables & \xmark \\
    \rowcolor{lightgray}
    Donut \citep{kim2022ocr} & Encoder-decoder & Swin Transformer & & \xmark & \xmark & \xmark \\
\bottomrule
\end{tabular}
\end{threeparttable}
\end{adjustbox}
\caption{Summary of general-purpose, multimodal pre-training document understanding models.}
\label{table:document-understanding-models}
\end{table}

\paragraph{Incorporating Layout Information} 

As the first work to jointly learn text and layout information, LayoutLM \citep{xu2020layoutlm} stands out as the pioneer work in multimodal pre-training for document understanding tasks. Over time, it has become the building block for designing more complex document understanding sytems. LayoutLM encodes layout information in the form of 2D position embeddings built using embedding layers. A 2D position embedding carries information about the spatial position of a token within the document page. The spatial position of a token is represented by its delineating bounding box $(x_0, y_0, x_1, y_1)$ obtained by an \ac{OCR} system \citep{kay2007tesseract}, where $(x_0, y_0)$ and $(x_1, y_1)$ respectively denote the upper-left and lower-right corners. The coordinates are discretized and normalized to integers in $[0, \ldots, 1000]$. Four embedding tables are used to encode spatial positions: two for the coordinates axes ($x$ and $y$) and the other two for the bounding box size (width and height). The final layout embedding $\bell \in \mathbb{R}^{d_{\ell}}$, for a token located at position $(x_0, y_0, x_1, y_1)$, is defined by:

\begin{equation}
\begin{split}
    \bell & = \text{2DPosEmb}_x(x_0) + \text{2DPosEmb}_y(y_0) \\
    & + \text{2DPosEmb}_x(x_1) + \text{2DPosEmb}_y(y_1) \\
    & + \text{2DPosEmb}_w(x_1 - x_0) \\
    & + \text{2DPosEmb}_h(y_1 - y_0) \\
\end{split}
\end{equation}

\noindent These 2D position embeddings are added to the 1D positional and text embeddings of \ac{BERT}. The resulting input sequence is passed through an encoder similar to \ac{BERT}. Via the self-attention mechanism, encoding 2D position features into the language model improves alignment between layout information and semantic representation. 

Building on the groundwork laid by LayoutLM, many works have introduced alternative approaches to encode layout information. BROS \citep{hong2020bros} employs sinusoidal functions as an alternative to linear embedding layers to encode continuous values for the spatial positions of tokens on the page. In addition to spatial positions, DocFormer \citep{appalaraju2021docformer} incorporates the Euclidean distance from each corner of a bounding box to the corresponding corner in the bounding box to its right, as well as the distance between centroids of the bounding boxes. DocFormer creates separate layout features for visual and language modalities. Layout features are untied from text embeddings and introduced as residual connections to each layer of a Transformer encoder, as opposed to directly adding them to language features. This approach is driven by the consideration that 2D dependencies might differ across layers. Based on the assumption that words in a block typically convey the same semantic meaning, StructuralLM \citep{li2021structurallm} departs from word-level 2D positions and leverages block-level 2D positions derived from the bounding boxes obtained through \ac{OCR}. As such, words that belong to the same block share the same 2D position embeddings. This approach allows the model to discern which words belong to the same block, thereby enhancing contextual representations of cells. Similarly, LayoutLMv3 \citep{huang2022layoutlmv3} models the layout information of blocks by adopting block-level 2D positions.

% In \ac{LAMPreT} \citep{wu2021\ac{LAMPreT}}, the layout is obtained by parsing a document into content blocks, each being assigned a block position, a block type (\textit{e.g.}, header, paragraph, image), and block attributes (\textit{i.e.}, font size, boldness, underline, and italic appearance). The document layout is defined as the structural presentation of the content blocks, and the aforementioned features of the textual contents within a block. 

Rather than treating layout information as an additional feature, some research works have modified model architectures to align with the specific layout formulation of documents. Specifically, these works favor leveraging information from content blocks (such as header, paragraph, figure) rather than working at the word-level, as contextualization between every word may be redundant and overlook localized context. In \ac{LAMPreT} \citep{wu2021lampret}, the layout is obtained by parsing a document into content blocks using PDF parsing tools. The layout is then defined as the structural presentation of these content blocks, and processed by two cascaded Transformers. The first one deals with the contents of a block, while the second one focuses on how these blocks are spatially structured. Similarly, SelfDoc \citep{li2021selfdoc} uses content blocks obtained with an object detection model, Faster R-CNN \citep{ren2015faster}, trained on semantically meaningful components such as text blocks, titles, lists, tables, and figures. These content blocks are then processed by a Transformer architecture. H-VILA \citep{shen2022vila} parses the document to extract group of tokens, which are then encoded using a hierarchical Transformer.


\paragraph{Encoding Visual Elements} 

To capture appearance features, many research works have leveraged vision models to integrate visual elements into document understanding systems.

% By leveraging the bounding box information of each word obtained through \ac{OCR}, document images are divided into pieces, aligning one-to-one with the words. Token image embeddings are derived from these pieces using the image regions obtained by Faster R-CNN.

In the fine-tuning stage of LayoutLM \citep{xu2020layoutlm}, optional token embeddings can be added to capture appearance features, \textit{e.g.}, fonts, types, colors. Token visual embeddings are obtained by splitting the document image according to the bounding boxes obtained through \ac{OCR}, and feeding the resulting pieces to Faster-RCNN \citep{ren2015faster}. LayoutLMv2 \citep{xu2020layoutlmv2} extends LayoutLM by integrating visual embeddings in the pre-training stage. Following contextualized word embeddings, contextualized image embeddings are expected to capture each image region semantics in the context of its entire visual neighborhood. In addition to the text $(w_1, \ldots, w_n)$ extracted from a document page image via \ac{OCR}, the model introduces visual tokens $(v_1, \ldots, v_{WH})$. These visual tokens are prepended to the text sequence, creating an input sequence $(v_1, \ldots, v_{WH}, w_1, \ldots, w_n)$ that combines visual and textual tokens. Visual tokens are obtained by resizing the document page image and feeding it to a visual encoder, namely ResNeXt-FPN \citep{xie2017aggregated, lin2017feature}. The resulting feature map is average-pooled to a fixed size $W \times H$, then flattened into a visual embedding sequence of length $WH$. A linear projection layer is then applied to each visual token embedding to unify the dimensionality with the text embeddings. Finally, LayoutLMv2 integrates the embedding of
each textual and visual token with its corresponding layout embedding. 
% ERNIE-Layout \citep{peng2022ernie} follows the same procedure to encode visual information.

In DocFormer \citep{appalaraju2021docformer}, ResNet50 \citep{he2016deep} is used to obtain high resolution image features. These features are then flattened and projected to obtain sequence of $n$ $d$-dimensional vectors, where $n$ represents the number of tokens on the page. Unlike LayoutLMv2, where visual and text features are concatenated into a single sequence, DocFormer adopts a different strategy. Here, visual features are disentangled and introduced as residual connections to each layer of a Transformer encoder. This design choice aims to enforce the correlation between language and vision modalities. 

\ac{LAMPreT} \citep{wu2021lampret} feeds the image contents of each text block into a pre-trained \ac{CNN}, followed by a feed-forward network that projects the visual embedding to the same dimension as textual embeddings. In addition, the model incorporates the visual presentation of the text within each block. Font size is defined as a block attribute, whereas text formatting (bold, italic, underlined) is set as binary-typed attributes. 

SelfDoc \citep{li2021selfdoc} exploits visual and language features at the block-level. For each block detected with Faster R-CNN, visual features are obtained from the same model, while the text is extracted using an \ac{OCR} system and embedded via Sentence-BERT \citep{reimers2019sentence}. The language and visual features are separately passed through a language and a vision Transformer-based encoder, respectively. A cross-modal encoder with cross-attention functions is then used to fuse the language and visual features.  

LayoutLMv3 \citep{huang2022layoutlmv3} takes a different approach to visual feature extraction. Instead of relying on a pre-trained \ac{CNN} or Faster R-CNN backbone, each image patch, obtained by resizing and uniformly splitting the document image, is encoded using linear embeddings. This approach not only saves parameters but also eliminates the need for region annotations.

\ac{DiT} \citep{li2022dit} exclusively relies on visual features for downstream usage on tasks such as image classification and text detection. Document images are divided into patches, added to 1D position embeddings, and passed through a stack of Transformer layers. The resulting contextualized output vectors serve as the representation of image patches.

\paragraph{Layout-aware Attention Mechanisms}

In addition to 2D position embeddings, LayoutLMv2 \citep{xu2020layoutlmv2} extends LayoutLM by encoding the 2D relative position as a \textit{bias} term. This term is added to the attention scores to explicitly capture the relationship between tokens, defining a \textit{spatial-aware attention mechanism}. Formally, the pre-softmax attention scores are defined as follows:

\begin{equation}
    \alpha_{i,j} = \dfrac{1}{\sqrt{d}} \bm{q}_i \cdot \bm{k}_j + \bm{b}^{(1D)}_{j - i} + \bm{b}^{(2D_x)}_{x_j - x_i} + \bm{b}^{(2D_y)}_{y_j - y_i},
\end{equation}

\noindent where $\bm{b}^{(1D)}$, $\bm{b}^{(2D_x)}$, and $\bm{b}^{(2D_y)}$ correspond to the sequential, horizontal, and vertical relative position biases, respectively. Likewise, TILT \citep{powalski2021going} incorporates 2D relative positions while entirely discarding the use of absolute 2D positions in its encoding strategy. 

To prevent early merging of distinct types of relative (sequential and spatial) position information, ERNIE-Layout \citep{peng2022ernie} computes attention scores between tokens using \textit{disentangled} matrices on their semantics and relative sequential and spatial positions. Specifically, attention scores are disentangled into four components: language semantics (denoted by the subscript $S$), relative sequential positions ($1D$), relative horizontal positions ($x$), and relative vertical positions ($y$). Let $\delta_{1D}$, $\delta_x$, and $\delta_y$ be the sequential, horizontal, and vertical relative distance matrices between every pair of tokens in the sequence. Given the query and key projections $\bm{q}^\star, \bm{k}^\star$, where $\star \in \{s, 1D, x, y\}$, the disentangled unscaled pre-softmax attention scores are computed as follows:

\begin{equation}
\begin{aligned}
    \alpha^{S}_{ij} &= \bm{q}^{S}_i \cdot \bm{k}^{S}_j \\
    \alpha^{1D}_{ij} &= \bm{q}^{S}_i \cdot \bm{k}^{1D}_{\delta_{1D}(i, j)} + \bm{k}^{S}_j  \cdot \bm{q}^{1D}_{\delta_{1D}(j, i)} \\
    \alpha^{x}_{ij} &= \bm{q}^{S}_i \cdot \bm{k}^{x}_{\delta_{x}(i, j)} + \bm{k}^{S}_j  \cdot \bm{q}^{x}_{\delta_{x}(j, i)} \\
    \alpha^{y}_{ij} &= \bm{q}^{S}_i \cdot \bm{k}^{y}_{\delta_{y}(i, j)} + \bm{k}^{S}_j  \cdot \bm{q}^{y}_{\delta_{y}(j, i)} \\
\end{aligned}
\label{eq:ernie-layout-attention}
\end{equation}

\noindent These attention scores are summed up to get the final attention matrix.

Similarly, DocFormer \citep{appalaraju2021docformer} modifies the self-attention mechanism to obtain spatially-aware features. In each Transformer layer, self-attention processes language semantics (denoted by $S$) and visual (denoted by $V$) modalities separately. Given $\star \in \{S, V\}$, let $\bell^{\star} = (\bell^{\star}_1, \ldots, \bell^{\star}_n)$ be the modality-specific layout features, and $\bm{q}^\star$, $\bm{k}^\star$, the query and key projections. Let $\bm{b}^{(1D)}$ represent the sequential relative position bias vector, and $\bm{W}^{(2D)}_q$ and $\bm{W}^{(2D)}_k$ be the learnable projection matrices for layout-specific query and key, respectively. The modality-specific attention score\footnote{For the sake of clarity, we omit the scaling and softmax operations.} for tokens $i$ and $j$ is disentangled as follows:

\begin{equation}
    \alpha^{\star}_{ij} = \left(\bm{q}^{\star}_i \cdot \bm{k}^{\star}_j\right) + \left(\bm{q}^{\star}_i \cdot \bm{b}^{(1D)}_{j-i}\right) + \left(\bm{q}^{\star}_j \cdot \bm{b}^{(1D)}_{j-i}\right) + \left(\bell^{\star}_i \cdot \bm{W}^{(2D)}_q \right) \left(\bell^{\star}_j \cdot \bm{W}^{(2D)}_k \right).
\end{equation}

% In SelfDoc \citep{li2021selfdoc}, self-attention in the cross-modal encoder is replaced by two cross-modal attention functions. The first function identifies the alignment between language and visual information, \textit{e.g.}, if the font size in a text is confirmed by the semantic meaning of language features, these features should be amplified. Let $\bm{q}^\star$, $\bm{k}^\star$, and $\bm{v}^\star$ denote the modality-specific query, key and value projections, where $\star \in \{S, V\}$. For the language modality, the first cross-modal attention function is defined as follows:

% \begin{equation}
%     f_{1}(\bm{q}^S_i, \bm{k}^V_j, \bm{v}^S_j) = \softmax \left( \dfrac{\bm{q}^S_i \cdot \bm{k}^V_j}{\sqrt{d}}\right) \bm{v}^S_j.
% \end{equation}

% \noindent The second attention function uncovers inner-relationships between modalities, \textit{e.g.}, similarity in font style between content blocks can improve the understanding of semantic meaning between these blocks. Formally:

% \begin{equation}
%     f_{2}(\bm{q}^V_i, \bm{k}^V_j, \bm{v}^S_j) = \softmax \left( \dfrac{\bm{q}^V_i \cdot \bm{k}^V_j}{\sqrt{d}}\right) \bm{v}^S_j.
% \end{equation}

% \noindent The vision modality undergoes the same process by swapping $S$ and $V$ in the above equations.

In \ac{LiLT} \citep{wang2022lilt}, text embeddings and layout embeddings are fed into their respective modality-specific Transformer encoder to create contextualized features. To consider the cross-modal interactions between text and layout across the entire pipeline, a \textit{bi-directional attention complementation mechanism} is introduced. In each layer of each modality-specific encoder, the attention scores obtained by the other encoder at the corresponding layer are summed with the current attention scores.  

FormNet \citep{lee2022formnet} completely avoids the use of both absolute and relative (sequential and spatial) position embeddings. The underlying idea behind the attention mechanism in FormNet, Rich Attention, is that features such as the order two tokens are in, how many tokens separate them, or how many pixels apart they are, are often relevant to the decision of how strongly a token should attend to another one. Therefore, Rich Attention computes, for every pair of tokens, their order and log-distance with respect to the $x$ and $y$ axes on the grid. For an attention head at a certain layer $l$, the model computes the \textit{actual} \textit{order} $o_{ij}$ and \textit{log-distance} $o_{ij}$ between token representations $\bm{x}^{(l)}_i$ and $\bm{x}^{(l)}_j$:

\begin{align}
    o_{ij} &= \{i < j\} \\
    d_{ij} &= \text{ln}(1 + \mid i - j \mid).
\end{align}

\noindent Using two affine functions $f^p$ and $f^{\mu}$, it then calculates the \textit{ideal} order $p_{ij}$ and log-distance $\mu_{ij}$ the tokens should have if there was a meaningful relationship between them:

\begin{align}
    p_{ij} &= \text{Sigmoid}\left(f^p(\bm{x}^{(l)}_i \mathbin\Vert \bm{x}^{(l)}_j)\right)\\
    \mu_{ij} &= f^{\mu}(\bm{x}^{(l)}_i \mathbin\Vert \bm{x}^{(l)}_j).
\end{align}

\noindent The predicted and ground-truth orders and log-distances are then compared using binary cross-entropy and L2 loss functions, respectively. The corresponding losses are then added to the attention scores. By penalizing token pairs that violate these soft order/distance constraints, the ability to learn logicial implication rules is incorporated into the model.


\paragraph{Pre-training Tasks for Cross-modal Alignment}

LayoutLM \citep{xu2020layoutlm} is pre-trained using two self-supervised pre-training tasks, \textit{\ac{MVLM}} and \textit{Multi-Label Document Classification}. Drawing inspiration from the \ac{MLM} strategy, \ac{MVLM} aims to learn language representation by considering both semantics and spatial clues. \ac{MVLM} randomly masks some tokens while retaining their layout information, as well as layout information and semantics from the remaining tokens. The model is then trained to predict the masked tokens based on the contextual information. Therefore, LayoutLM not only understands language contexts, but also uses the corresponding 2D information, thereby bridging the gap between visual and language modalities. On the other hand, Multi-Label Document Classification improves document-level representations by supervising the pre-training process using the document tags, if available. The objective is for the model to group knowledge from diverse domains, resulting in improved document-level representations. 

On top of the \ac{MVLM} objective, two new pre-training strategies are added to enforce the alignment among modalities in LayoutLMv2 \citep{xu2020layoutlmv2}: \textit{Text-Image Alignment} and \textit{Text-Image Matching}. Text-Image Alignment is a fine-grained cross-modality alignment task, where the image regions of some randomly selected text tokens are covered, and the model has to predict, for each text token, whether it is covered. On the other hand, Text-Image Matching is a coarse-grained cross-modality alignment task, where the model is asked to predict whether an image and a text are from the same document page. 

In SelfDoc \citep{li2021selfdoc}, image features are randomly masked and the model is tasked to predict them. DocFormer \citep{appalaraju2021docformer} is trained to reconstruct the original image by passing its output features through a \ac{CNN}-based image decoder. While this task focuses on local features, DocFormer introduces another task to infuse global information into the model. In this task, DocFormer is trained to predict whether a given piece of text, represented by a pooled representation of the output features, describes a document image. 

In addition to \ac{MVLM} and Text-Image Matching, ERNIE-Layout  \citep{peng2022ernie} introduces two novel pre-training tasks. In ERNIE-Layout, documents are parsed into blocks using an in-house document layout analysis toolkit. Each block includes a series of words and their corresponding bounding boxes. Because there is no explicit boundary between blocks in the input fed to the Transformer, the models needs to learn the relationship between layout and reading order. The \textit{Reading Order Prediction} strategy aims to enhance interactions between tokens belonging to the same block. Suppose that the attention score $\alpha_{ij}$, obtained by summing the four attention components in Equation~\ref{eq:ernie-layout-attention}, carries information about the reading order. In other words, it represents the probability that the $j$-th token follows the $i$-th token. Given $g_{ij}$ as the ground-truth value, equal to 1 if the $j$-th token follows the $i$-th token and $0$ otherwise, the Reading Order Prediction loss is defined as:

\begin{equation}
    \mathcal{L}_{\text{ROP}} = - \sum_{1 \leq i \leq n}\sum_{1 \leq j \leq n} g_{ij} \log(\alpha_{ij}).
\end{equation}

\noindent The Text-Image Matching task introduced in LayoutLMv2 primarily focuses on aligning content at the whole image-text level. However, instances where the image and text are completely unrelated tend to be too easy for the model. Therefore, ERNIE-Layout adopts the \textit{Replaced Regions Prediction} strategy, designed to strengthen alignment between modalities at a fine-grained level. The original image is split into patches, and a percentage of image patches are randomly selected and replaced with a random patch from another image. The processed image is encoded by the visual encoder and fed to ERNIE-Layout. The \texttt{[CLS]} output representation is then used to predict which patches are replaced. 

Inspired by SpanBERT \citep{joshi2020spanbert}, BROS \citep{hong2020bros} extends spans of a one-dimensional text to consecutive text bounding boxes in a two-dimensional space. This approach involves selecting a few regions in the document layout, masking all tokens within bounding boxes in the selected regions, and predicting the masked tokens. 

\ac{LAMPreT} \citep{wu2021lampret} introduces three higher-level objectives designed to exploit the structural interactions among blocks. The training of the model involves predicting whether two blocks are swapped, allowing it to learn the spatial order of blocks. Additionally, given a set of masked blocks, the model has to select the most suitable block for each masked block. Finally, when provided with masked images, the model must choose the most suitable images.

Most visually-augmented Transformer-based pre-trained models diverge in their pre-training objectives for the vision modality. This discrepancy in pre-training objectives for the vision modality makes multimodal representation learning more challenging. Besides, learning cross-modal alignment, which is essential for effective multimodal representation learning, becomes more difficult due to the differing granularities in images (dense image pixels or contiguous region features) and text (discrete tokens) objectives. To overcome the discrepancy between language and visual representation learning, and facilitate multimodal representation learning, LayoutLMv3 \citep{huang2022layoutlmv3} proposes unified discrete token reconstructive objectives. In addition to the \ac{MVLM} strategy, LayoutLMv3 uses \textit{Masked Image Modeling}, a symmetry to \ac{MVLM} that consists in randomly masking image tokens, and reconstructing them using their surrounding context. To learn a fine-grained aligment between text words and image patches, LayoutLMv3 is also pre-trained with a \textit{Word-Patch Alignment} objective. This objective involves predicting whether the corresponding image patches of a text word are masked. 


\paragraph{Incorporating 1D Position Information}

All document pre-training techniques operate on serialized text. An \ac{OCR} system or PDF parser is used to extract text from a document and serialize it according to a raster-scan order, which aligns tokens in a sequence from the top-left to the bottom-right corner. However, this arrangement does not always conform to human reading patterns, particularly for documents with complex layouts such as multicolumn texts, tables, and forms. This misalignment with human reading habits can result in suboptimal performance in document understanding tasks. To alleviate this issue, ERNIE-Layout \citep{peng2022ernie} uses an in-house document layout analysis toolkit that provides an appropriate reading order based on the spatial distribution of words, pictures, and tables. Enhanced with this knowledge, the token sequence can be rearranged in a way that yields a lower perplexity compared to the raster-scan order. This translates into a serialization that aligns better with human reading patterns. Another strategy to mitigate reading order errors involves using more robust 1D position encodings. \ac{LSPE} \citep{wang2022simple} is a learnable sinusoidal positional encoding method based on feed-forward networks. \ac{LSPE} can be integrated into any Transformer-based model and demonstrates improved performance and robustness on noisy data with unreliable reading order information. 

% Likewise, the model introduced by \citet{pham2022understanding} is driven by the observation that, in real-world documents, word relationships extend beyond the sequential nature of texts, and are also influenced by the arrangement of words in the two-dimensional space. In these scenarios, spatial information becomes necessary, complementing textual information. 

\paragraph{Model Initialization} 

Several models take advantage of existing powerful pre-trained language models and adapt them to document understanding tasks. LayoutLM \citep{xu2020layoutlm} is initialized with the weights of \ac{BERT} \citep{devlin2018bert}, LayoutLMv2 \citep{xu2020layoutlmv2} leverages UniLMv2 \citep{bao2020unilmv2}, while ERNIE-Layout \citep{peng2022ernie} and StructuralLM \citep{li2021structurallm} are initialized from RoBERTa \citep{liu2019roberta}.

Certain studies unify document understanding tasks under one framework by casting them as sequence-to-sequence problems, thereby expanding the language generation capabilities of the models. Donut \citep{kim2022ocr} is an OCR-free encoder-decoder model that consists of a visual encoder initialized with Swin Transformer \citep{liu2021swin}, and a decoder that uses the weights of mBART \citep{liu2020multilingual}. TILT \citep{powalski2021going} adds layout and visual information into \ac{T5} \citep{raffel2020exploring}. LayoutT5 and LayoutBART \citep{tanaka2021visualmrc} add 2D position embeddings and visual features to \ac{T5} and \ac{BART} \citep{lewis2019bart} in the fine-tuning stage.

\paragraph{Multilinguality} 

These models have demonstrated successful application on English Documents. However, visually-rich documents typically exhibit varying formats and layouts based on the country, and this diversity even extends to different regions within the same country. To bridge the language barriers for real-world document understanding, LayoutXLM \citep{xu2021layoutxlm} carries out multilingual pre-training by expanding the language support of LayoutLMv2 to a total of 53 languages.

\paragraph{Capturing Long-range Dependencies}

The majority of multimodal pre-trained models focus on short documents due to the computational requirements and memory limitations of the Transformer. Furthermore, incorporating layout and visual information renders them more resource-intensive compared to pre-trained models that exclusively deal with text. As such, long document understanding remains largely under-explored. However, real-world documents, such as business documents, can be very dense and long. The conventional approach for handling long documents involves truncating them into short segments and processing each segment independently. This approach presents a significant challenge for multi-page and cross-page understanding of long documents, where useful information is usually distributed across their lengths. While the component-level formulation \citep{li2021structurallm, li2021selfdoc} can reduce the input sequence length for a document, it does not allow capturing long-range dependencies. Therefore, processing long documents requires a suitable method to connect information across pages.

To handle multi-page documents, \citet{pramanik2020towards}'s model encodes the page number of each token and uses the Longformer architecture as its backbone. \citet{pham2022understanding}'s approach introduces a plug-able approach to integrating spatial input into self-attention, removing the need for extra embeddings. Drawing inspiration from the sliding window approach in Longformer \citep{beltagy2020longformer}, the model introduces attention masks based on spatial information. In this approach, each context window for a bounding box is defined by calculating its spatial neighbors, rather than relying on neighboring words determined by the sequential order obtained from an \ac{OCR} system. \\


% encouraging research on understanding documents with complex interplay of text, layout and graphical elements.

Research on understanding documents with complex interplay of text, layout and visual elements has garnered significant attention for document understanding tasks. Notably, general-purposed multimodal pre-trained language models have  experienced a surge in usage due to their remarkable performance. Yet, the academic literature on methodologies addressing Document Understanding remains relatively scarce when compared to fields with abundant publicly available data, such as image classification and translation. While advancements in Deep Learning has led to significant improvements in document understanding tasks, the field faces several challenges in practical applications. Firstly, the limited input length of pre-trained language models poses difficulties in processing long documents. Secondly, real-world documents, originating from scanning equipment and affected by issues like crumpled paper, deviate in quality from annotated training data, resulting in suboptimal performance. Addressing this issue involves data synthesis and augmentation techniques. Thirdly, existing document understanding tasks are often treated independently from each other, and there is a lack of effective leveraging of correlations between different tasks. Fourthly, pre-trained document understanding models encounter challenges due to insufficient computing resources and labeled training samples in practical applications. This emphasizes the importance of research directions such as model compression, few-shot learning, and zero-shot learning. Finally, the relationship between \ac{OCR} and document understanding tasks is crucial. Given that document understanding systems typically receive input from \ac{OCR} systems, the accuracy of text recognition and the ordering of words play pivotal roles in downstream tasks. Addressing these challenges and research directions is essential for advancing the field of Document Understanding in practical scenarios.

\acresetall

